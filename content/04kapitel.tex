%!TEX root = ../dokumentation.tex

\chapter{Implementierung}
In dem folgenden Kapitel wird der entwickelte Code und die Umsetzung der Schach-Künstlichen Intelligenz besprochen. Hierbei wird jedoch nicht jede einzelne Zeile Code explizit erklärt, sondern der Fokus auf die Schlüsselmerkmale gelegt werden, sowie kompliziertere Abschnitte erläutert werden. Ein besonderes Augenmerk wird dabei auf die Umsetzung, der in Kapitel \ref{theoretische_hintergründe} eingeführten Algorithmen und Techniken gelegt. Zuvor wird jedoch noch ein Benutzerhandbuch eingeführt, indem beschrieben wird, wie das Programm zu verwenden ist. Zudem wird ein zu Kapitel 3.3 ergänzender Überblick über die Verzeichnisstruktur gegeben.

   \section{Benutzerhandbuch}\label{benutzerhandbuch}

In diesem Kapitel werden die zwei Möglichkeiten beschrieben, wie das Programm ausgeführt werden kann - entweder direkt über Python mit dem Quellcode oder über einen Docker Container. Zudem wird erklärt, wie das Programm zu verwenden ist.

\subsection{Verwenden der Schach KI mittels ``python
run''}\label{verwenden-der-schach-ki-mittels-python-run}

Zum lokalen Starten des Programmes muss zunächst das entsprechende
GitHub Repository runtergeladen werden. Dazu ist die Installation von
\texttt{git} nötig. Dieses Programm inklusive Installationsanleitungen kann hier
gefunden werden:

\href{https://git-scm.com/download/}{git-scm.com/download}

Daraufhin muss in dem Verzeichnis, in das das Programm heruntergeladen werden
soll navigiert werden und in der Konsole der Befehl

\texttt{git\ clone\ https://github.com/StudienarbeitLarsPascal/PythonChessAI.git}

ausgeführt werden.

Anschließend müssen die nötigen Packages installiert werden. Dazu ist
zunächst die Installation von Python notwendig:

\href{https://www.python.org/downloads/}{python.org/downloads}

Nun kann in der Konsole in das Verzeichnis
\texttt{PythonChessAI/python\_chess\_ai} navigiert werden und
anschließend der Befehl

\texttt{pip\ install\ -r\ requirements.txt}

ausgeführt werden.

Jetzt ist das Programm startbereit. Dazu muss in eben genannten
Verzeichnis der Befehl

\texttt{python\ run\ main.py}

ausgeführt werden. Dazu können Startparameter verwendet werden wie in
Kapitel \ref{befehlszeilenargumente-zum-starten-der-schach-ki} erklärt. Wie dies bedient
wird, ist in Kapitel \ref{bedienen-des-programms} beschrieben.

\subsection{Verwenden der Schach KI mittels Ausführen des Docker
Containers}\label{verwenden-der-schach-ki-mittels-ausfuxfchren-des-docker-containers}

Um das Programm in einem Docker Container laufen zu lassen, ist zunächst
die Installation von Docker nötig:

\href{https://www.docker.com/get-started}{docker.com/get-started}

Dann kann der Docker Container heruntergeladen werden, indem per Konsole
in das gewünschte Verzeichnis navigiert wird und anschließend der Befehl

\texttt{docker\ pull\ pascalschroeder/python\_chess\_ai}

ausgeführt wird.

Nach dem erfolgreichen Herunterladen kann das Programm gestartet werden,
indem der Befehl

\texttt{docker\ run\ -it\ pascalschroeder/python\_chess\_ai}

durchlaufen wird. Dadurch gelangt man direkt in den Programmablauf. Wie
dies bedient werden kann, ist in Kapitel \ref{bedienen-des-programms} beschrieben.

\subsection{Verwenden der Schach KI direkt im Jupyter Notebook}\label{verwenden-der-schach-ki-direkt-im-jupyter-notebook}

Die Schach KI ist in einer simplifizierten Version, in der Erstellen der Spieler und dynamische Abfragen wegfallen, auch direkt im Jupyter Notebook ausführbar. Dazu muss das Jupyter Notebook 'Chess-AI.ipynb' gestartet werden und jede Zelle von oben bis unten durchlaufen werden. Sobald die letzte Zelle erreicht ist wird das Spiel gestartet. Dabei wird zunächst der Spieler zu einer Eingabe aufgefordert, ehe die KI den 2. Zug übernimmt. Dies wiederholt sich so lange bis das Spiel vorbei ist. Wie diese vereinfachte Verwaltung aufgebaut ist, ist in Kapitel \ref{starten-des-spiels-im-jupyter-notebook} zu sehen.

\subsection{Bedienen des Programms}\label{bedienen-des-programms}

Nach Start des Programmes folgen, wenn keine Startparameter
mitgegeben wurden, die Einstellungen.

Dabei wird zunächst der Name des ersten Spielers abgefragt. Dabei ist
eine beliebige Eingabe möglich, die mit 'Enter' bestätigt werden muss.
Nun wird der Typ des Spielers abgefragt. Hierbei muss eine Zahl zwischen 1
und 4 eingeben werden. 1 steht dabei für einen menschlichen Spieler, 2
für die KI, 3 für eine API zu einer anderen Schach-KI, die bisher noch
nicht implementiert wurde, jedoch in zukünftigen Versionen geplant ist,
und 4 für einen "Dummy", der stets einen zufälligen Zug spielt.

Wird beim Typ des Spielers die KI gewählt, wird zusätzlich noch
abgefragt, welchen Schwierigkeitsgrad die KI haben soll. Dabei sind
Werte von 1 bis 3 möglich, wobei 1 für leicht, 2 für mittel und 3 für
schwer steht.

Dieses Prozedere wird noch für den 2. Spieler wiederholt.

Nach Durchlauf für beide Spieler beginnt das eigentliche Spiel. Dazu
wird zunächst das aktuelle Schachbrett in der Konsole ausgegeben. Ist
der Spieler, der an der Reihe ist, ein manueller Spieler, so wird
zusätzlich eine Liste aller legalen Züge ausgegeben. Dann wird der
Spieler zur Eingabe einer dieser Züge aufgefordert. Die Eingabe muss in
den ersten zwei Zeichen das Feld enthalten, von dem eine Figur aus
bewegt werden soll, und bei Zeichen drei und vier die Felder, auf die
die Figur bewegt werden soll. Ist der eingegebene Zug möglich, so wird
dieser auf dem Schachfeld durchgeführt und erneut ausgegeben. Dann ist
der nächste Spieler am Zug. Andernfalls wird die Aufforderung zur
Eingabe eines Zuges wiederholt, bis eine legale Eingabe
vorliegt und der Zug durchgeführt werden kann.

Sobald das Spiel zu Ende ist, wird der Sieger des Spiels genannt. Danach
wird der Spieler gefragt, ob er mit den gleichen Einstellungen das Spiel
wiederholen möchte. Mit Eingabe einer "1" und bestätigen durch
die "Enter" Taste kann dies bejaht werden. So wird das Spiel nochmal mit
den gleichen Einstellungen wiederholt. Andernfalls wird das Programm
beendet.

Jederzeit kann mit der Tastenkombination "Ctrl" + "C" das Programm und somit auch das Spiel vorläufig beendet werden.

    \section{Verzeichnisstruktur}\label{verzeichnisstruktur}

Um die Entwicklung des Programms möglichst modular und somit erweiterbar
zu gestalten und zugleich ein übersichtliches Lesen des Codes zu
ermöglichen, wurden die einzelnen Funktionen in verschiedene Klassen
aufgeteilt und teils mittels Schnittstellen vereinheitlicht.

Die Hauptdatei, die für das Starten und Erstellen des Spiels
verantwortlich ist, ist die \texttt{main.py}. In dieser werden die
Spieler mit ihren Attributen (Spielertyp, Name, ggf. Schwierigkeitsgrad)
erstellt, die an der Schach Partie teilnehmen, sowie der
\texttt{chess\_master} initialisiert. Zudem wird hinterlegt, ob das
Spiel als GUI oder in der Konsole starten soll.

Zum Abfragen der Einstellungen wird die Datei
\texttt{settings/interface.py} zur Hilfe genommen, die alle nicht beim
Starten mit übermittelten Parameter der Spieler abfragt und entsprechend
an die \texttt{main.py} zurückgibt, die die Spieler auf Basis dieser
Informationen erstellt. Dazu sind zwei Implementierungen dieser
Schnittstelle vorhanden - einmal für eine konsolenbasierte
(\texttt{settings/terminal.py}) Abfrage der Einstellungen und einmal für
eine GUI basierte Darstellung (\texttt{settings/gui.py}).

Der eben angesprochene \texttt{chess\_master} befindet sich in der
\texttt{chess\_master.py}. Dieser verwaltet das Schachspiel bis dies zu
Ende ist. Dabei wird über die Liste der teilnehmenden Spieler iteriert
und jeder nach dem von ihm gewünschten Zug gefragt. Diese Züge werden
dann auf dem virtuellen Schachbrett vorgenommen. Zudem lässt der
\texttt{chess\_master} die einzelnen Spieler stets den aktuellen
Spielzustand ausgeben. Zusätzlich ist der \texttt{chess\_master} auch
für das Pflegen der Spielverläufe zuständig. Dies bedeutet, dass der
\texttt{chess\_master} nach jedem getätigten Zug diesen der Liste aller
Züge hinzufügt und am Ende des Spiels diese in den vorhandenen
Spielverlauf zusammen mit dem Ausgang des Spiels migriert.

Zum Abfragen der Spieler nach ihren Zügen, das Forcieren der Ausgabe des
Spielzustands sowie alle weiteren Interaktionen mit den Spielern steht
mit der \texttt{player/interface.py} eine weiter Schnittstelle zur
Verfügung. Diese ist für menschliche Spieler (\texttt{player/user.py}),
eine künstliche Intelligenz (\texttt{player/ai.py}), eine Schnittstelle
zu einer vorhandenen Intelligenz (\texttt{player/api.py}) sowie zu einem
sogenannten "Dummy" (\texttt{player/dummy.py}), der stets einen
zufälligen Zug auswählt, implementiert. Diese beinhalten allesamt die
Funktionen \texttt{get\_move}, \texttt{submit\_move} und
\texttt{print\_board}. Welche der Dateien für das Spiel genommen wird
ist abhängig von den getätigten Einstellungen über den Spielertypen.

Beim Spieler \texttt{player/user.py} existiert zudem eine weitere
Schnittstelle \texttt{player/user\_input/interface.py}, die ebenfalls
als GUI sowie als konsolenbasierte Version existiert, um den
menschlichen Spieler beide Optionen zu ermöglichen.

Die Datei \texttt{player/ai.py} nimmt sich zum Evaluieren der
Spielzustände die Datei \texttt{misc/ai\_evaluation\_library} zur Hilfe.
Zudem sind einige wiederkehrende Funktionen für alle Dateien in der
\texttt{misc/tools.py} gespeichert, so dass diese stets aufrufbar sind.

Eine genaue Erläuterung der Architektur kann Kapitel \ref{architektur} entnommen
werden. Im folgenden Teil wird Wissen über Verzeichnisstruktur sowie
Architektur vorrausgesetzt, um die Abläufe und Kommunikationsflüsse
unter den Dateien und Funktionen zu verstehen.

    \section{Befehlszeilenargumente zum Starten der
Schach-KI}\label{befehlszeilenargumente-zum-starten-der-schach-ki}

Eine erste Option, um Einfluss auf den Ablauf des Programmes zu
gewinnen, ist das Mitliefern von Argumenten direkt beim Start der
Anwendung. Diese Argumente ermöglichen es Programme und Scripts mit
Speziellen Eigenschaften aufrufen zu können.

Diese Art der Spezialisierung von Programmen wird von der Sprache Python
mittels des Moduls \texttt{argparse} unterstützt.

    \begin{Verbatim}[commandchars=\\\{\}]
{\color{incolor}In [{\color{incolor}1}]:} \PY{k+kn}{import} \PY{n+nn}{argparse}
\end{Verbatim}

    Wie im Kapitel \ref{bedienen-des-programms} bereits beschrieben müssen zum Start der Schach-KI
einige Einstellungen vorgenommen werden. Hierunter fallen sowohl die
Spieler, ihr Spielertyp und falls es sich um das Spiel gegen eine KI
handelt um deren Schwierigkeitsgrad. Durch die Verwendung von
\texttt{argparse} lassen sich diese Einstellungen als
Befehlszeilenargumente vornehmen und somit einer Abfrage von
Informationen entgehen.

Im folgenden Code wird neben des Nutzers und der Spieleinstellung mittels \texttt{argparse} auch die Möglichkeit gegeben auszuwählen, ob das
Graphical User Interface geöffnet werden soll, oder das Spiel in der
Konsole stattfinden soll. Ebenfalls lässt sich durch die Argumente die
aktuelle Entwicklungsversion der Schach-KI ausgeben.

Für die Implementierung und Verwendung von Befehlszeilenargumente wurde
die Funktion \texttt{intialize\_parser} entwickelt in der die benötigten
Einstellungen des Argument-Parsers vorgenommen werden. Aufgerufen wird
diese Funktion bei der Verwendung der Schach-KI bevor die Hauptfunktion
aufgerufen wird.

Dazu wird zu Beginn ein neuer Parser erzeugt und diesem werden zwei
verschiedene Argumentgruppen zugewiesen. Die beiden Gruppen sind dafür
zuständig weitere Argumente zu enthalten. Der Unterschied dieser liegt
darin, dass in der Gruppe \texttt{ui\_group} nur eins der angehörigen
Argumente ausgewählt werden dürfen. Dies hat den Grund, dass sich der
Nutzer zwischen der Verwendung des Graphical User Interface und der
Konsole entscheiden muss und nicht beides gleichzeitig verwenden kann.

Jedes Argument, das hinzugefügt wird, kann dabei durch zwei verschiedene
Argumente angesprochen werden. Die Kurzform dieses Arguments wird mit
einem Spiegelstrich angesprochen und die längere Variante mit zwei
Spiegelstrichen. Durch den Parameter \texttt{help} lässt sich dem Nutzer
eine Hilfestellung geben was die Aufgabe dieses Arguments ist. Diese
Hilfestellung kann aufgerufen werden indem das Programm mit dem
Parameter \texttt{-h} oder \texttt{-\/-help} aufgerufen wird.

Ebenfalls lässt sich bei der Initialisierung eines Arguments der
Parameter \texttt{action} angeben, der bei den Argumenten
\texttt{-\/-terminal} und \texttt{-\/-gui} im Parser den Boolean-Wert
\texttt{True} speichert falls diese durch den Nutzer aufgerufen wurden.

Die zweite Argumentengruppe enthält Argumente die gleichzeitig während
des Startens aufgerufen werden können. Hierunter fällt beispielsweise
der Spielername, dessen Typ und die Schwierigkeit einer KI.

Bei allen drei Argumenten, die der Gruppe
\texttt{player\_name\_type\_group} angehören wurde die Eigenschaft
festgelegt, dass genau zwei Werte bei Aufruf dieser Argumente angegeben
werden müssen. Eine weitere Spezialisierung wird bei den Argumenten zum
Spielertyp und der Spielschwierigkeit vorgenommen. Bei diesen Argumenten
hat der User nur die Auswahl zwischen vorgegebenen Antworten und kann
dabei nur diese auswählen.

Ingesamt wurde bei dieser Argumentengruppe die Hilfestellung durch
\texttt{help=argparse.SUPPRESS} deaktiviert, da dies Einstellungen sind,
die hauptsächlich für den wiederholten Aufruf durch Scripts interessant
sind und nicht für den allgemeinen Nutzer. Somit werden diese Argumente
bei der Verwendung des Parameters \texttt{-h}/\texttt{-\/-help} nicht
aufgelistet.

Als letztes Argument wird ein Parameter initialisiert, der für die
Version der Schach-KI zuständig ist. Nachdem alle Argumente
initialisiert wurden wird der Parser, zu dem diese Argumente gehören,
zurückgegeben.

    \begin{Verbatim}[commandchars=\\\{\}]
{\color{incolor}In [{\color{incolor}2}]:} \PY{k}{def} \PY{n+nf}{intialize\PYZus{}parser}\PY{p}{(}\PY{p}{)}\PY{p}{:}
            \PY{n}{parser} \PY{o}{=} \PY{n}{argparse}\PY{o}{.}\PY{n}{ArgumentParser}\PY{p}{(}\PY{p}{)}
            \PY{n}{ui\PYZus{}group} \PY{o}{=} \PY{n}{parser}\PY{o}{.}\PY{n}{add\PYZus{}mutually\PYZus{}exclusive\PYZus{}group}\PY{p}{(}\PY{p}{)}
            \PY{n}{ui\PYZus{}group}\PY{o}{.}\PY{n}{add\PYZus{}argument}\PY{p}{(}\PY{l+s+s2}{\PYZdq{}}\PY{l+s+s2}{\PYZhy{}t}\PY{l+s+s2}{\PYZdq{}}\PY{p}{,} \PY{l+s+s2}{\PYZdq{}}\PY{l+s+s2}{\PYZhy{}\PYZhy{}terminal}\PY{l+s+s2}{\PYZdq{}}\PY{p}{,} \PY{n}{help}\PY{o}{=}\PY{l+s+s2}{\PYZdq{}}\PY{l+s+s2}{starts the terminal ui}\PY{l+s+s2}{\PYZdq{}}\PY{p}{,} 
		\PY{n}{action}\PY{o}{=}\PY{l+s+s2}{\PYZdq{}}\PY{l+s+s2}{store\PYZus{}true}\PY{l+s+s2}{\PYZdq{}}\PY{p}{)}
            \PY{n}{ui\PYZus{}group}\PY{o}{.}\PY{n}{add\PYZus{}argument}\PY{p}{(}\PY{l+s+s2}{\PYZdq{}}\PY{l+s+s2}{\PYZhy{}g}\PY{l+s+s2}{\PYZdq{}}\PY{p}{,} \PY{l+s+s2}{\PYZdq{}}\PY{l+s+s2}{\PYZhy{}\PYZhy{}gui}\PY{l+s+s2}{\PYZdq{}}\PY{p}{,} \PY{n}{help}\PY{o}{=}\PY{l+s+s2}{\PYZdq{}}\PY{l+s+s2}{starts the GUI}\PY{l+s+s2}{\PYZdq{}}\PY{p}{,} 
		\PY{n}{action}\PY{o}{=}\PY{l+s+s2}{\PYZdq{}}\PY{l+s+s2}{store\PYZus{}true}\PY{l+s+s2}{\PYZdq{}}\PY{p}{)}
        
            \PY{n}{player\PYZus{}name\PYZus{}type\PYZus{}group} \PY{o}{=} \PY{n}{parser}\PY{o}{.}\PY{n}{add\PYZus{}argument\PYZus{}group}\PY{p}{(}\PY{p}{)}
            \PY{n}{player\PYZus{}name\PYZus{}type\PYZus{}group}\PY{o}{.}\PY{n}{add\PYZus{}argument}\PY{p}{(}\PY{l+s+s2}{\PYZdq{}}\PY{l+s+s2}{\PYZhy{}p}\PY{l+s+s2}{\PYZdq{}}\PY{p}{,} \PY{l+s+s2}{\PYZdq{}}\PY{l+s+s2}{\PYZhy{}\PYZhy{}player}\PY{l+s+s2}{\PYZdq{}}\PY{p}{,} \PY{n}{nargs}\PY{o}{=}\PY{l+m+mi}{2}\PY{p}{,} 
		\PY{n}{help}\PY{o}{=}\PY{n}{argparse}\PY{o}{.}\PY{n}{SUPPRESS}\PY{p}{)} 
		\PY{c+c1}{\PYZsh{} help=\PYZdq{}set name player 1 and 2\PYZdq{}}
            \PY{n}{player\PYZus{}name\PYZus{}type\PYZus{}group}\PY{o}{.}\PY{n}{add\PYZus{}argument}\PY{p}{(}\PY{l+s+s2}{\PYZdq{}}\PY{l+s+s2}{\PYZhy{}pT}\PY{l+s+s2}{\PYZdq{}}\PY{p}{,}\PY{l+s+s2}{\PYZdq{}}\PY{l+s+s2}{\PYZhy{}\PYZhy{}player\PYZus{}type}\PY{l+s+s2}{\PYZdq{}}\PY{p}{,} \PY{n}{nargs}\PY{o}{=}\PY{l+m+mi}{2}\PY{p}{,} 
		\PY{n}{choices}\PY{o}{=}\PY{p}{[}\PY{l+s+s2}{\PYZdq{}}\PY{l+s+s2}{User}\PY{l+s+s2}{\PYZdq{}}\PY{p}{,}\PY{l+s+s2}{\PYZdq{}}\PY{l+s+s2}{AI}\PY{l+s+s2}{\PYZdq{}}\PY{p}{,} \PY{l+s+s2}{\PYZdq{}}\PY{l+s+s2}{Player}\PY{l+s+s2}{\PYZdq{}}\PY{p}{,} \PY{l+s+s2}{\PYZdq{}}\PY{l+s+s2}{Dummy}\PY{l+s+s2}{\PYZdq{}}\PY{p}{]}\PY{p}{,} \PY{n}{help}\PY{o}{=}\PY{n}{argparse}\PY{o}{.}\PY{n}{SUPPRESS}\PY{p}{)} 
		\PY{c+c1}{\PYZsh{} help=\PYZdq{}enter player type for player 1 and 2\PYZdq{}}
            \PY{n}{player\PYZus{}name\PYZus{}type\PYZus{}group}\PY{o}{.}\PY{n}{add\PYZus{}argument}\PY{p}{(}\PY{l+s+s2}{\PYZdq{}}\PY{l+s+s2}{\PYZhy{}pD}\PY{l+s+s2}{\PYZdq{}}\PY{p}{,}\PY{l+s+s2}{\PYZdq{}}\PY{l+s+s2}{\PYZhy{}\PYZhy{}player\PYZus{}difficulty}\PY{l+s+s2}{\PYZdq{}}\PY{p}{,} 
		\PY{n}{nargs}\PY{o}{=}\PY{l+m+mi}{2}\PY{p}{,} \PY{n+nb}{type}\PY{o}{=}\PY{n+nb}{int}\PY{p}{,} \PY{n}{choices}\PY{o}{=}\PY{n+nb}{range}\PY{p}{(}\PY{l+m+mi}{0}\PY{p}{,}\PY{l+m+mi}{4}\PY{p}{)}\PY{p}{,} \PY{n}{help}\PY{o}{=}\PY{n}{argparse}\PY{o}{.}\PY{n}{SUPPRESS}\PY{p}{)} 
		\PY{c+c1}{\PYZsh{} help=\PYZdq{}enter ai difficulty for player 1 and 2,\PYZdq{}}
 		 \PY{c+c1}{use 0 if you are not using ai\PYZdq{}}
        
            \PY{n}{parser}\PY{o}{.}\PY{n}{add\PYZus{}argument}\PY{p}{(}\PY{l+s+s2}{\PYZdq{}}\PY{l+s+s2}{\PYZhy{}v}\PY{l+s+s2}{\PYZdq{}}\PY{p}{,} \PY{l+s+s2}{\PYZdq{}}\PY{l+s+s2}{\PYZhy{}\PYZhy{}version}\PY{l+s+s2}{\PYZdq{}}\PY{p}{,} \PY{n}{help}\PY{o}{=}\PY{l+s+s2}{\PYZdq{}}\PY{l+s+s2}{print the version number\PYZdq{}}
		\PY{c+c1}{and exit}\PY{l+s+s2}{\PYZdq{}}\PY{p}{,} \PY{n}{action}\PY{o}{=}\PY{l+s+s2}{\PYZdq{}}\PY{l+s+s2}{store\PYZus{}true}\PY{l+s+s2}{\PYZdq{}}\PY{p}{)}
            \PY{k}{return} \PY{n}{parser}
\end{Verbatim}

    Nach der Initialisierung des Parsers wird die Hauptfunktion
\texttt{main} mit dem Rückgabewert der Funktion
\texttt{intialize\_parser} aufgerufen. Dieser Rückgabewert enthält die
Befehlszeilenargumente, die in der \texttt{main} durch If-Verzweigungen
verwendet werden.

Falls der Nutzer die Schach-KI startet ohne ein Übergabeparameter oder
mit dem Parameter \texttt{-g}/\texttt{-\/-gui} wird der Variable
\texttt{ui\_status} der Wert \texttt{1} zugewiesen. Diese Variable wird
im weiteren Verlauf des Codes dazu verwendet, um entscheiden zu können
welche Visualisierungsmöglichkeit ausgewählt wurde.

Der aktuelle Stand ist, dass nur die Nachricht für den Nutzer ausgegeben
wird, dass das Graphical User Interface gestartet wird. Da die
Entwicklung einer weiteren Benutzerschnittstellen den Umfang dieser
wissenschaftlichen Arbeit sprengen würde, ist die Möglichkeit vorhanden
zukünftig eine weitere Schnittstelle hinzuzufügen, jedoch dies in der
aktuellen Version nicht umgesetzt.

Falls der Nutzer den Parameter \texttt{-t}/\texttt{-\/-terminal}
verwendet hat, wird die Funktion
\texttt{start\_chess\_maste(ui\_status)} mit dem Übergabeparameter
\texttt{ui\_status} und dessen Wert \texttt{0} aufgerufen. Sollte der
Nutzer den Parameter \texttt{-v}/\texttt{-\/-version} verwenden erhält
die aktuelle Versionsnummer.

    \begin{Verbatim}[commandchars=\\\{\}]
{\color{incolor}In [{\color{incolor}3}]:} \PY{k}{def} \PY{n+nf}{main}\PY{p}{(}\PY{p}{)}\PY{p}{:}
            \PY{k}{if} \PY{n}{args}\PY{o}{.}\PY{n}{terminal} \PY{o+ow}{or} \PY{p}{(}\PY{o+ow}{not} \PY{p}{(}\PY{n}{args}\PY{o}{.}\PY{n}{gui}\PY{p}{)} \PY{o+ow}{and} \PY{o+ow}{not} \PY{p}{(}\PY{n}{args}\PY{o}{.}\PY{n}{terminal}\PY{p}{)} 
	     \PY{o+ow}{and} \PY{o+ow}{not} \PY{p}{(}\PY{n}{args}\PY{o}{.}\PY{n}{version}\PY{p}{)}\PY{p}{)}\PY{p}{:}
                \PY{n}{ui\PYZus{}status} \PY{o}{=} \PY{l+m+mi}{0}
                \PY{n}{start\PYZus{}chess\PYZus{}master}\PY{p}{(}\PY{n}{ui\PYZus{}status}\PY{p}{)}
            \PY{k}{elif} \PY{n}{args}\PY{o}{.}\PY{n}{terminal}\PY{p}{:}
                \PY{n}{ui\PYZus{}status} \PY{o}{=} \PY{l+m+mi}{1}
                \PY{n}{start\PYZus{}chess\PYZus{}master}\PY{p}{(}\PY{n}{ui\PYZus{}status}\PY{p}{)}
            \PY{k}{elif} \PY{n}{args}\PY{o}{.}\PY{n}{version}\PY{p}{:}
                \PY{n+nb}{print}\PY{p}{(}\PY{n}{\PYZus{}\PYZus{}version\PYZus{}\PYZus{}}\PY{p}{)}
\end{Verbatim}

    Die zusätzlichen Argumente, die der Nutzer angeben kann, werden in der
Funktion \texttt{start\_chess\_master(ui\_status)} verwendet um damit
steuern zu können, ob die Einstellungen des Spiels gesetzt wurden, oder
diese abgefragt werden müssen.

    \section{Erstellen der Spieler}\label{erstellen-der-spieler}

Vor dem Start eines Schachspiels müssen zunächst zusätzlich noch Spiel
generell sowie die partizpierenden Spieler der gewünschten Spielertypen
erstellt werden. Dazu muss der Spieler dem Programm entweder Parameter
mitgeben, wie in Kapitel \ref{befehlszeilenargumente-zum-starten-der-schach-ki} beschrieben, die das Spiel direkt mit
den gewünschten Spielern startet, oder aber der Spieler wird während des
Erstellvorgangs nach den fehlenden Einstellungen gefragt.

Dazu wird eine Liste an Spielern anhand der gegebenen Informationen
erstellt, mittels welcher das Spiel stattfinden kann. Dies findet
in der Funktion \texttt{interrogate\_settings} aus der Schnittstelle
\texttt{settings/interface.py} statt.

Wird das Spiel mit Parametern gestartet werden diese mit in die
\texttt{interrogate\_settings} Funktion übermittelt. Andernfalls werden
alle fehlenden Informationen in der Funktion über die
Nutzerschnittstelle abgefragt. Dies sieht wie folgt aus:

    \begin{Verbatim}[commandchars=\\\{\}]
{\color{incolor}In [{\color{incolor}4}]:} \PY{k}{def} \PY{n+nf}{interrogate\PYZus{}settings}\PY{p}{(}\PY{n+nb+bp}{self}\PY{p}{,} \PY{n}{player\PYZus{}names}\PY{o}{=}\PY{k+kc}{None}\PY{p}{,} \PY{n}{player\PYZus{}types}\PY{o}{=}\PY{k+kc}{None}\PY{p}{,} 
	  \PY{n}{player\PYZus{}difficulty} \PY{o}{=} \PY{k+kc}{None}\PY{p}{)}\PY{p}{:}
            \PY{n}{players} \PY{o}{=} \PY{p}{[}\PY{p}{]}
            \PY{k}{for} \PY{n}{i} \PY{o+ow}{in} \PY{n+nb}{range}\PY{p}{(}\PY{l+m+mi}{2}\PY{p}{)}\PY{p}{:}
                \PY{n}{num} \PY{o}{=} \PY{n}{i}\PY{o}{+}\PY{l+m+mi}{1}
                \PY{k}{if} \PY{n}{player\PYZus{}names} \PY{o+ow}{is} \PY{k+kc}{None}\PY{p}{:}
                    \PY{n}{name} \PY{o}{=} \PY{n+nb+bp}{self}\PY{o}{.}\PY{n}{get\PYZus{}player\PYZus{}name}\PY{p}{(}\PY{n}{num}\PY{p}{)}
                \PY{k}{else}\PY{p}{:}
                    \PY{n}{name} \PY{o}{=} \PY{n}{player\PYZus{}names}\PY{p}{[}\PY{n}{i}\PY{p}{]}
        
                \PY{k}{if} \PY{n}{player\PYZus{}types} \PY{o+ow}{is} \PY{k+kc}{None}\PY{p}{:}
                    \PY{n}{player\PYZus{}type} \PY{o}{=} \PY{n+nb+bp}{self}\PY{o}{.}\PY{n}{get\PYZus{}player\PYZus{}type}\PY{p}{(}\PY{n}{num}\PY{p}{)}
                \PY{k}{else}\PY{p}{:}
                    \PY{n}{player\PYZus{}type} \PY{o}{=} \PY{n}{player\PYZus{}types}\PY{p}{[}\PY{n}{i}\PY{p}{]}
        
                \PY{n}{difficulty} \PY{o}{=} \PY{k+kc}{None}
                \PY{k}{if} \PY{n}{player\PYZus{}type} \PY{o}{==} \PY{l+m+mi}{2} \PY{o+ow}{and} \PY{n}{player\PYZus{}difficulty} \PY{o+ow}{is} \PY{k+kc}{None}\PY{p}{:}
                    \PY{n}{difficulty} \PY{o}{=} \PY{n+nb+bp}{self}\PY{o}{.}\PY{n}{get\PYZus{}difficulty}\PY{p}{(}\PY{n}{num}\PY{p}{)}
                \PY{k}{elif} \PY{n}{player\PYZus{}type} \PY{o}{==} \PY{l+m+mi}{2} \PY{o+ow}{and} \PY{n}{player\PYZus{}difficulty} \PY{o+ow}{is} \PY{o+ow}{not} \PY{k+kc}{None}\PY{p}{:}
                    \PY{n}{difficulty} \PY{o}{=} \PY{n}{player\PYZus{}difficulty}\PY{p}{[}\PY{n}{i}\PY{p}{]}
        
                \PY{n}{new\PYZus{}player} \PY{o}{=} \PY{n}{PlayerSettings}\PY{p}{(}\PY{n}{num}\PY{p}{,} \PY{n}{name}\PY{p}{,} \PY{n}{player\PYZus{}type}\PY{p}{,} \PY{n}{difficulty}\PY{p}{)}
                \PY{n}{players}\PY{o}{.}\PY{n}{append}\PY{p}{(}\PY{n}{new\PYZus{}player}\PY{p}{)}
        
            \PY{k}{return} \PY{n}{players}
\end{Verbatim}

    Dabei werden für jeden Spieler einmal ein Block an Abfragen
durchgegangen, der alle Einstellungen abfragt und diese entsprechend
speichert.

Nach der Abfrage der Informationen "Name" und "Spielertyp" wird im
Falle, dass dieser Spielertyp "2", der der KI entspricht, auch
noch der Schwierigkeitsgrad abgefragt, sollte dieser nicht bereits als
Startparameter übergeben worden sein.

Die Funktionen zur Abfrage dieser Informationen müssen in der
Implementierung der Schnittstelle definiert werden. Dies kann also
sowohl in der GUI als auch in der Terminal-Implementierung stattfinden. Die
Implementierung dieser Informationsabfrage ist in der
\texttt{settings/terminal.py} Datei vorzufinden und sieht wie folgt aus:

    \begin{Verbatim}[commandchars=\\\{\}]
{\color{incolor}In [{\color{incolor}5}]:} \PY{k}{def} \PY{n+nf}{get\PYZus{}player\PYZus{}type}\PY{p}{(}\PY{n+nb+bp}{self}\PY{p}{,} \PY{n}{player\PYZus{}num}\PY{p}{)}\PY{p}{:}
            \PY{n+nb}{super}\PY{p}{(}\PY{p}{)}\PY{o}{.}\PY{n}{get\PYZus{}player\PYZus{}type}\PY{p}{(}\PY{n}{player\PYZus{}num}\PY{p}{)}
            \PY{n}{legal\PYZus{}range} \PY{o}{=} \PY{n+nb}{range}\PY{p}{(}\PY{n}{PLAYER\PYZus{}TYPE\PYZus{}MIN}\PY{p}{,} \PY{n}{PLAYER\PYZus{}TYPE\PYZus{}MAX} \PY{o}{+} \PY{l+m+mi}{1}\PY{p}{)}
            \PY{n}{input\PYZus{}msg} \PY{o}{=} \PY{n}{PLAYER\PYZus{}TYPE\PYZus{}INPUT\PYZus{}MESSAGE}\PY{o}{.}\PY{n}{format}\PY{p}{(}\PY{n}{player\PYZus{}num}\PY{p}{)}
            \PY{n}{input\PYZus{}player\PYZus{}type} \PY{o}{=} \PY{n}{Tools}\PY{o}{.}\PY{n}{check\PYZus{}legal\PYZus{}input\PYZus{}int}\PY{p}{(}\PY{n}{legal\PYZus{}range}\PY{p}{,} \PY{n}{input\PYZus{}msg}\PY{p}{)}
            \PY{k}{return} \PY{n}{input\PYZus{}player\PYZus{}type}
        
        \PY{k}{def} \PY{n+nf}{get\PYZus{}player\PYZus{}name}\PY{p}{(}\PY{n+nb+bp}{self}\PY{p}{,} \PY{n}{player\PYZus{}num}\PY{p}{)}\PY{p}{:}
            \PY{n+nb}{super}\PY{p}{(}\PY{p}{)}\PY{o}{.}\PY{n}{get\PYZus{}player\PYZus{}name}\PY{p}{(}\PY{n}{player\PYZus{}num}\PY{p}{)}
            \PY{k}{return} \PY{n+nb}{input}\PY{p}{(}\PY{n}{PLAYER\PYZus{}NAME\PYZus{}INPUT\PYZus{}MESSAGE}\PY{o}{.}\PY{n}{format}\PY{p}{(}\PY{n}{player\PYZus{}num}\PY{p}{)}\PY{p}{)}
        
        \PY{k}{def} \PY{n+nf}{get\PYZus{}difficulty}\PY{p}{(}\PY{n+nb+bp}{self}\PY{p}{,} \PY{n}{player\PYZus{}num}\PY{p}{)}\PY{p}{:}
            \PY{n+nb}{super}\PY{p}{(}\PY{p}{)}\PY{o}{.}\PY{n}{get\PYZus{}difficulty}\PY{p}{(}\PY{n}{player\PYZus{}num}\PY{p}{)}
            \PY{n}{legal\PYZus{}range} \PY{o}{=} \PY{n+nb}{range}\PY{p}{(}\PY{n}{DIFFICULTY\PYZus{}MIN}\PY{p}{,} \PY{n}{DIFFICULTY\PYZus{}MAX} \PY{o}{+} \PY{l+m+mi}{1}\PY{p}{)}
            \PY{n}{input\PYZus{}msg} \PY{o}{=} \PY{n}{DIFFICULTY\PYZus{}INPUT\PYZus{}MESSAGE}\PY{o}{.}\PY{n}{format}\PY{p}{(}\PY{n}{player\PYZus{}num}\PY{p}{)}
            \PY{n}{input\PYZus{}difficulty} \PY{o}{=} \PY{n}{Tools}\PY{o}{.}\PY{n}{check\PYZus{}legal\PYZus{}input\PYZus{}int}\PY{p}{(}\PY{n}{legal\PYZus{}range}\PY{p}{,} \PY{n}{input\PYZus{}msg}\PY{p}{)}
            \PY{k}{return} \PY{n}{input\PYZus{}difficulty}
\end{Verbatim}

    Anschließend wird ein neues Einstellungsparameterschema für Spieler,
genannt \texttt{PlayerSettings}, erstellt und der Liste hinzugefügt.
Diese Liste wird nach Durchgang der Schleife für beide Spieler zurückgegeben.

Nachdem die Einstellungen für alle Spieler bekannt sind, werden die
Spieler erstellt. Dies funktioniert wie in nachfolgendem Code-Snippet zu
sehen.

    \begin{Verbatim}[commandchars=\\\{\}]
{\color{incolor}In [{\color{incolor}6}]:} \PY{k}{def} \PY{n+nf}{start\PYZus{}chess\PYZus{}master}\PY{p}{(}\PY{n}{ui\PYZus{}status}\PY{p}{)}\PY{p}{:}    
            \PY{k}{try}\PY{p}{:}
                \PY{n}{settings\PYZus{}ui} \PY{o}{=} \PY{n}{ui\PYZus{}switcher}\PY{p}{(}\PY{n}{ui\PYZus{}status}\PY{p}{)}\PY{o}{.}\PY{n}{Settings}\PY{p}{(}\PY{p}{)}
                \PY{n}{player\PYZus{}settings} \PY{o}{=} \PY{n}{settings\PYZus{}ui}\PY{o}{.}\PY{n}{interrogate\PYZus{}settings}\PY{p}{(}\PY{p}{)}
                \PY{n}{players} \PY{o}{=} \PY{p}{[}\PY{p}{]}
                \PY{k}{for} \PY{n}{player\PYZus{}setting} \PY{o+ow}{in} \PY{n}{player\PYZus{}settings}\PY{p}{:}
                    \PY{n+nb}{type} \PY{o}{=} \PY{n}{type\PYZus{}switcher}\PY{p}{(}\PY{n}{player\PYZus{}setting}\PY{o}{.}\PY{n}{type}\PY{p}{)}
                    \PY{n}{players}\PY{o}{.}\PY{n}{append}\PY{p}{(}\PY{n+nb}{type}\PY{o}{.}\PY{n}{Player}\PY{p}{(}\PY{n}{player\PYZus{}setting}\PY{o}{.}\PY{n}{num}\PY{p}{,} 
			\PY{n}{player\PYZus{}setting}\PY{o}{.}\PY{n}{name}\PY{p}{,} \PY{n}{ui\PYZus{}status}\PY{p}{,} \PY{n}{player\PYZus{}setting}\PY{o}{.}\PY{n}{difficulty}\PY{p}{)}\PY{p}{)}
        
                \PY{n}{chess\PYZus{}master} \PY{o}{=} \PY{n}{ChessMaster}\PY{p}{(}\PY{p}{)}
                \PY{n}{chess\PYZus{}master}\PY{o}{.}\PY{n}{start\PYZus{}chess\PYZus{}game}\PY{p}{(}\PY{n}{players}\PY{p}{)}
            \PY{k}{except} \PY{n+ne}{KeyboardInterrupt}\PY{p}{:}
                \PY{n+nb}{print}\PY{p}{(}\PY{l+s+s2}{\PYZdq{}}\PY{l+s+se}{\PYZbs{}n}\PY{l+s+s2}{You}\PY{l+s+s2}{\PYZsq{}}\PY{l+s+s2}{ve quit the game.}\PY{l+s+s2}{\PYZdq{}}\PY{p}{)}
            \PY{k}{except} \PY{n+ne}{Exception}\PY{p}{:}
                \PY{n}{traceback}\PY{o}{.}\PY{n}{print\PYZus{}exc}\PY{p}{(}\PY{n}{file}\PY{o}{=}\PY{n}{sys}\PY{o}{.}\PY{n}{stdout}\PY{p}{)}
            \PY{n}{sys}\PY{o}{.}\PY{n}{exit}\PY{p}{(}\PY{l+m+mi}{0}\PY{p}{)}
\end{Verbatim}

    Dabei werden zunächst wie gehabt die Einstellungen abgefragt, ehe eine
Liste aller Spieler erstellt wird. Dann wird über jeden Eintrag in der
Einstellungsliste iteriert und für diesen zunächst der Spielertyp
berechnet. Hierzu wird der \texttt{type\_switcher} zur Hilfe genommen,
welcher je nach angegebenen Typ die entsprechende Klasse des Spielertyps
zurückgibt:

    \begin{Verbatim}[commandchars=\\\{\}]
{\color{incolor}In [{\color{incolor}7}]:} \PY{k}{def} \PY{n+nf}{type\PYZus{}switcher}\PY{p}{(}\PY{n}{player\PYZus{}type}\PY{p}{)}\PY{p}{:}
            \PY{k}{return} \PY{p}{\PYZob{}}
                \PY{l+m+mi}{1}\PY{p}{:} \PY{n}{user}\PY{p}{,}
                \PY{l+m+mi}{2}\PY{p}{:} \PY{n}{ai}\PY{p}{,}
                \PY{l+m+mi}{3}\PY{p}{:} \PY{n}{api}\PY{p}{,}
                \PY{l+m+mi}{4}\PY{p}{:} \PY{n}{dummy}
            \PY{p}{\PYZcb{}}\PY{p}{[}\PY{n}{player\PYZus{}type}\PY{p}{]}
\end{Verbatim}

    Mittels dieser wird dann ein neuer Spieler erstellt, wobei beim
Initialisierren alle weiteren Daten mitgegeben werden. Dieser neue
Spieler wird dann der Spielerliste hinzugefügt. Mit dieser Liste an
Spielern wird dann der \texttt{ChessMaster} initialisiert, der im
Kapitel \ref{verwalten-des-schachspiels-und-pflege-des-spielverlaufs} beschrieben wird.

Zusätzlich kann man das Spiel an dieser Stelle auch mittels eines
Keyboard Interrupts (CTRL + C) beenden.

    \section{Verwalten des Schachspiels und Pflege des
Spielverlaufs}\label{verwalten-des-schachspiels-und-pflege-des-spielverlaufs}

Ein essentieller Bestandteil im Erstellen eines Schachprogrammes ist das
Verwalten des Schachspiels. Dabei muss garantiert werden, dass solange
das Spiel nicht vorbei ist, die Spieler abwechselnd einen Zug auswählen
können, dieser Zug auf dem vorhandenen Schachbrett durchgeführt wird und das
daraus entstehende Brett dem Spieler sichtbar gemacht wird.

Dies ist Aufgabe der \texttt{chess\_master.py} und ist mit folgendem
Code umgesetzt:

    \begin{Verbatim}[commandchars=\\\{\}]
{\color{incolor}In [{\color{incolor}8}]:} \PY{k}{def} \PY{n+nf}{start\PYZus{}chess\PYZus{}game}\PY{p}{(}\PY{n+nb+bp}{self}\PY{p}{,} \PY{n}{players}\PY{p}{)}\PY{p}{:}
            \PY{n}{board} \PY{o}{=} \PY{n}{chess}\PY{o}{.}\PY{n}{Board}\PY{p}{(}\PY{p}{)}
            \PY{k}{while} \PY{o+ow}{not} \PY{n}{board}\PY{o}{.}\PY{n}{is\PYZus{}game\PYZus{}over}\PY{p}{(}\PY{p}{)}\PY{p}{:}
                \PY{n}{current\PYZus{}player} \PY{o}{=} \PY{n}{players}\PY{p}{[}\PY{n+nb}{int}\PY{p}{(}\PY{o+ow}{not} \PY{n}{board}\PY{o}{.}\PY{n}{turn}\PY{p}{)}\PY{p}{]}
                \PY{n}{current\PYZus{}player}\PY{o}{.}\PY{n}{print\PYZus{}board}\PY{p}{(}\PY{n}{current\PYZus{}player}\PY{o}{.}\PY{n}{name}\PY{p}{,} \PY{n}{board}\PY{p}{)}
        
                \PY{n}{move} \PY{o}{=} \PY{n}{current\PYZus{}player}\PY{o}{.}\PY{n}{get\PYZus{}move}\PY{p}{(}\PY{n}{board}\PY{p}{)}
                \PY{n}{board}\PY{o}{.}\PY{n}{push}\PY{p}{(}\PY{n}{move}\PY{p}{)}
                \PY{n}{current\PYZus{}player}\PY{o}{.}\PY{n}{submit\PYZus{}move}\PY{p}{(}\PY{n}{move}\PY{p}{)}
        
            \PY{n}{current\PYZus{}player} \PY{o}{=} \PY{n}{players}\PY{p}{[}\PY{n+nb}{int}\PY{p}{(}\PY{o+ow}{not} \PY{n}{board}\PY{o}{.}\PY{n}{turn}\PY{p}{)}\PY{p}{]}
            \PY{n}{current\PYZus{}player}\PY{o}{.}\PY{n}{print\PYZus{}board}\PY{p}{(}\PY{n}{current\PYZus{}player}\PY{o}{.}\PY{n}{name}\PY{p}{,} \PY{n}{board}\PY{p}{)}
\end{Verbatim}

    Dabei wird zunächst ein neues Schachbrett - hier board genannt -
erstellt. Solange das Spiel auf diesem nicht vorbei ist, das mittels der
\texttt{board.is\_game\_over()} Funktiong geprüft werden kann, wird dann
stets die gleiche Schleife durchlaufen.

In dieser wird zuallererst der aktuelle Spieler ermittelt und
referenziert. Dazu wird aus einer vorhandenen Liste aller Spieler
derjenige gewählt, dessen Position in der Liste der gespeicherten
Zugnummer des boards entspricht. Diese ist entweder 0, wenn der weiße
Spieler an der Reihe ist, oder 1, wenn der schwarze Spieler den nächsten
Zug auswählen kann.

Der daraus berechnete Spieler wurde zuvor beim Spielstart ein Spielertyp
zugewiesen, der auf einer Schnittstelle basiert und somit alle nötigen
Funktionen implementiert. Im Verwalter des Schachspiels werden diese nun nach für nach aufgerufen.

Zunächst wird das board für den Spieler ausgegeben mittels der
\texttt{print\_board} Funktion. Ist der Spieler ein Nutzer, so wird
diese für gewöhnlich am Nutzerinterface ausgegeben. Dies wird in Kapitel
\ref{ausgabe-des-schachbretts-und-eingabe-von-schachzuxfcgen} näher erläutert. Andernfalls ist dies nicht nötig und die
Funktion kann leer bleiben, ohne etwas auszugeben.

Nun kommt es zum wichtigsten Teil - dem Berechnen des nächsten Zuges.
Dazu wird der Spieler aufgefordert anhand eines gegebenen boards den
nächsten Zug zu nennen. In diesem Teil übernimmt beispielsweise die KI
ihre Berechnungen für den nächsten Schachzug. Dies wird in Kapitel
\ref{zugfindung-durch-ki} erklärt. Der Nutzer dagegen gibt diesen mittels eines
Eingabefeldes ein, das beispielsweise über die Nutzer Interface
Schnittstelle \texttt{player/user\_input/terminal.py} implementiert ist
wie hier zu sehen ist:

    \begin{Verbatim}[commandchars=\\\{\}]
{\color{incolor}In [{\color{incolor}9}]:} \PY{k}{def} \PY{n+nf}{get\PYZus{}move}\PY{p}{(}\PY{n+nb+bp}{self}\PY{p}{,} \PY{n}{legal\PYZus{}moves}\PY{p}{)}\PY{p}{:}
            \PY{n+nb}{super}\PY{p}{(}\PY{p}{)}\PY{o}{.}\PY{n}{get\PYZus{}move}\PY{p}{(}\PY{n}{legal\PYZus{}moves}\PY{p}{)}
            \PY{n}{input\PYZus{}msg} \PY{o}{=} \PY{n}{ASK\PYZus{}FOR\PYZus{}MOVE\PYZus{}MESSAGE}\PY{o}{.}\PY{n}{format}\PY{p}{(}\PY{n}{legal\PYZus{}moves}\PY{p}{)}
            \PY{n}{move} \PY{o}{=} \PY{n}{Tools}\PY{o}{.}\PY{n}{check\PYZus{}legal\PYZus{}input\PYZus{}string}\PY{p}{(}\PY{n}{legal\PYZus{}moves}\PY{p}{,}\PY{n}{input\PYZus{}msg}\PY{p}{,} 
		\PY{n}{WRONG\PYZus{}INPUT\PYZus{}MESSAGE}\PY{p}{)}
            \PY{k}{return} \PY{n}{move}
\end{Verbatim}

    Nachdem der Zug vom entsprechenden Spieler berechnet und zurückgegeben
wurde, wird dieser dem aktuellen board hinzugefügt. Dadurch wechselt
auch automatisch der Spieler, der an der Reihe ist, wodurch dieser im
nächsten Schleifendurchlauf nach dessen Zug gefragt wird.

Abschließend wird das neue board nochmal abgeschickt. Dabei kann zum
Beispiel eine erneute Ausgabe des Schachbretts mit dem aktualisierten
Zustand stattfinden oder beispielsweise bei der Nutzung einer Schach-API
der gewählte Zug an die Schnittstelle gesendet werden.

Nach dem Ende des Spiels wird noch das Ergebnis ausgegeben. Diese
Ausgabe erfolgt zur aktuellen Version ausschließlich über die Konsole.
Dies sieht wie folgt aus:

    \begin{Verbatim}[commandchars=\\\{\}]
{\color{incolor}In [{\color{incolor}10}]:} \PY{k}{def} \PY{n+nf}{start\PYZus{}chess\PYZus{}game}\PY{p}{(}\PY{n+nb+bp}{self}\PY{p}{,} \PY{n}{players}\PY{p}{)}\PY{p}{:}
             \PY{n}{board} \PY{o}{=} \PY{n}{chess}\PY{o}{.}\PY{n}{Board}\PY{p}{(}\PY{p}{)}
             \PY{k}{while} \PY{o+ow}{not} \PY{n}{board}\PY{o}{.}\PY{n}{is\PYZus{}game\PYZus{}over}\PY{p}{(}\PY{p}{)}\PY{p}{:}
                 \PY{n}{current\PYZus{}player} \PY{o}{=} \PY{n}{players}\PY{p}{[}\PY{n+nb}{int}\PY{p}{(}\PY{o+ow}{not} \PY{n}{board}\PY{o}{.}\PY{n}{turn}\PY{p}{)}\PY{p}{]}
                 \PY{n}{current\PYZus{}player}\PY{o}{.}\PY{n}{print\PYZus{}board}\PY{p}{(}\PY{n}{current\PYZus{}player}\PY{o}{.}\PY{n}{name}\PY{p}{,} \PY{n}{board}\PY{p}{)}
         
                 \PY{n}{move} \PY{o}{=} \PY{n}{current\PYZus{}player}\PY{o}{.}\PY{n}{get\PYZus{}move}\PY{p}{(}\PY{n}{board}\PY{p}{)}
                 \PY{n}{board}\PY{o}{.}\PY{n}{push}\PY{p}{(}\PY{n}{move}\PY{p}{)}
                 \PY{n}{current\PYZus{}player}\PY{o}{.}\PY{n}{submit\PYZus{}move}\PY{p}{(}\PY{n}{move}\PY{p}{)}
         
             \PY{n}{current\PYZus{}player} \PY{o}{=} \PY{n}{players}\PY{p}{[}\PY{n+nb}{int}\PY{p}{(}\PY{o+ow}{not} \PY{n}{board}\PY{o}{.}\PY{n}{turn}\PY{p}{)}\PY{p}{]}
             \PY{n}{current\PYZus{}player}\PY{o}{.}\PY{n}{print\PYZus{}board}\PY{p}{(}\PY{n}{current\PYZus{}player}\PY{o}{.}\PY{n}{name}\PY{p}{,} \PY{n}{board}\PY{p}{)}
         
             \PY{n}{result} \PY{o}{=} \PY{n}{Tools}\PY{o}{.}\PY{n}{get\PYZus{}board\PYZus{}result}\PY{p}{(}\PY{n}{board}\PY{p}{)}
             \PY{k}{if} \PY{n}{result} \PY{o+ow}{is} \PY{l+m+mi}{1}\PY{p}{:}
                 \PY{n+nb}{print}\PY{p}{(}\PY{l+s+s2}{\PYZdq{}}\PY{l+s+si}{\PYZob{}\PYZcb{}}\PY{l+s+s2}{ (White) has won}\PY{l+s+s2}{\PYZdq{}}\PY{o}{.}\PY{n}{format}\PY{p}{(}\PY{n}{players}\PY{p}{[}\PY{l+m+mi}{0}\PY{p}{]}\PY{o}{.}\PY{n}{name}\PY{p}{)}\PY{p}{)}
             \PY{k}{elif} \PY{n}{result} \PY{o+ow}{is} \PY{o}{\PYZhy{}}\PY{l+m+mi}{1}\PY{p}{:}
                 \PY{n+nb}{print}\PY{p}{(}\PY{l+s+s2}{\PYZdq{}}\PY{l+s+si}{\PYZob{}\PYZcb{}}\PY{l+s+s2}{ (Black) has won}\PY{l+s+s2}{\PYZdq{}}\PY{o}{.}\PY{n}{format}\PY{p}{(}\PY{n}{players}\PY{p}{[}\PY{l+m+mi}{1}\PY{p}{]}\PY{o}{.}\PY{n}{name}\PY{p}{)}\PY{p}{)}
             \PY{k}{else}\PY{p}{:}
                 \PY{n+nb}{print}\PY{p}{(}\PY{l+s+s2}{\PYZdq{}}\PY{l+s+s2}{Draw}\PY{l+s+s2}{\PYZdq{}}\PY{p}{)}
\end{Verbatim}

    Dabei wird zuerst das Ergebnis an Hand der \texttt{get\_board\_result}
Funktion aus der Hilfsklasse "Tools" ermittelt. Diese sieht wie folgt
aus:

    \begin{Verbatim}[commandchars=\\\{\}]
{\color{incolor}In [{\color{incolor}11}]:} \PY{k}{def} \PY{n+nf}{get\PYZus{}board\PYZus{}result}\PY{p}{(}\PY{n}{board}\PY{p}{)}\PY{p}{:}
             \PY{k}{if} \PY{n}{board}\PY{o}{.}\PY{n}{is\PYZus{}variant\PYZus{}loss}\PY{p}{(}\PY{p}{)}\PY{p}{:}
                 \PY{k}{return} \PY{o}{\PYZhy{}}\PY{l+m+mi}{1} \PY{k}{if} \PY{n}{board}\PY{o}{.}\PY{n}{turn} \PY{o}{==} \PY{n}{chess}\PY{o}{.}\PY{n}{WHITE} \PY{k}{else} \PY{l+m+mi}{1}
             \PY{k}{elif} \PY{n}{board}\PY{o}{.}\PY{n}{is\PYZus{}variant\PYZus{}win}\PY{p}{(}\PY{p}{)}\PY{p}{:}
                 \PY{k}{return} \PY{l+m+mi}{1} \PY{k}{if} \PY{n}{board}\PY{o}{.}\PY{n}{turn} \PY{o}{==} \PY{n}{chess}\PY{o}{.}\PY{n}{WHITE} \PY{k}{else} \PY{o}{\PYZhy{}}\PY{l+m+mi}{1}
             \PY{k}{elif} \PY{n}{board}\PY{o}{.}\PY{n}{is\PYZus{}variant\PYZus{}draw}\PY{p}{(}\PY{p}{)}\PY{p}{:}
                 \PY{k}{return} \PY{l+m+mi}{0}
         
             \PY{c+c1}{\PYZsh{} Checkmate.}
             \PY{k}{if} \PY{n}{board}\PY{o}{.}\PY{n}{is\PYZus{}checkmate}\PY{p}{(}\PY{p}{)}\PY{p}{:}
                 \PY{k}{return} \PY{o}{\PYZhy{}}\PY{l+m+mi}{1} \PY{k}{if} \PY{n}{board}\PY{o}{.}\PY{n}{turn} \PY{o}{==} \PY{n}{chess}\PY{o}{.}\PY{n}{WHITE} \PY{k}{else} \PY{l+m+mi}{1}
         
             \PY{c+c1}{\PYZsh{} Draw claimed.}
             \PY{k}{if} \PY{n}{board}\PY{o}{.}\PY{n}{can\PYZus{}claim\PYZus{}draw}\PY{p}{(}\PY{p}{)}\PY{p}{:}
                 \PY{k}{return} \PY{l+m+mi}{0}
         
             \PY{c+c1}{\PYZsh{} Seventyfive\PYZhy{}move rule or fivefold repetition.}
             \PY{k}{if} \PY{n}{board}\PY{o}{.}\PY{n}{is\PYZus{}seventyfive\PYZus{}moves}\PY{p}{(}\PY{p}{)} \PY{o+ow}{or} \PY{n}{board}\PY{o}{.}\PY{n}{is\PYZus{}fivefold\PYZus{}repetition}\PY{p}{(}\PY{p}{)}\PY{p}{:}
                 \PY{k}{return} \PY{l+m+mi}{0}
         
             \PY{c+c1}{\PYZsh{} Insufficient material.}
             \PY{k}{if} \PY{n}{board}\PY{o}{.}\PY{n}{is\PYZus{}insufficient\PYZus{}material}\PY{p}{(}\PY{p}{)}\PY{p}{:}
                 \PY{k}{return} \PY{l+m+mi}{0}
         
             \PY{c+c1}{\PYZsh{} Stalemate.}
             \PY{k}{if} \PY{o+ow}{not} \PY{n+nb}{any}\PY{p}{(}\PY{n}{board}\PY{o}{.}\PY{n}{generate\PYZus{}legal\PYZus{}moves}\PY{p}{(}\PY{p}{)}\PY{p}{)}\PY{p}{:}
                 \PY{k}{return} \PY{l+m+mi}{0}
             
             \PY{k}{return} \PY{l+m+mi}{0}
\end{Verbatim}

    Dabei wird jede mögliche Option, wie das Spiel zum Ende gekommen sein
kann, durchgegangen und anschließend das jeweilige Ergebnis
zurückgegeben. Eine 1 steht für einen Sieg für weiß, eine -1 für einen
Sieg für schwarz und eine 0 für ein Unentschieden beziehungsweise einen
Remis.

Nach Abfrage des Ergebnisses werden beide Spieler abhängig von diesem
entweder dazu aufgefordert eine Siegesbenachrichtigung auszugeben oder
aber eine Benachrichtigung über die Niederlage oder ein Unentschieden.

Zusätzlich zu dem Verwalten des Spiels ist es auch Aufgabe des
\texttt{ChessMaster}s die Historie aller Spiele zu pflegen. Dazu wird
zunächst eine Liste von Schachbrettern der Klasse \texttt{chess.Board}
angelegt. Nach jedem durchgeführten Zug wird der neue Zustand des boards
zu dieser hinzugefügt. Dabei wird das Schachbrett in der FEN-Notation
gespeichert, die in einem String den exakten Zustand des Schachbretts
wiedergeben kann. Dabei wird jedoch nur der erste Teil dieser Notation
gespeichert, da dieser alleine bereits Aufschluss über die
Positionierungen gibt. Die darauffolgenden Teile sind zum Speichern des
Spielers, der am Zug ist, wie viele Züge bereits durchgeführt wurden und
weitere Informationen, die zum Bewerten in der Historie nicht notwendig
sind.

Zum Speichern der Züge in einem Verlauf wird die Funktion zum einen um
eine Liste \texttt{turn\_list} erweitert, in die die Züge dann als
FEN Notation abgespeichert werden, die in \ref{FEN-Notation} erklärt wurde. Dazu wird die Funktion wie
folgt angepasst.

    \begin{Verbatim}[commandchars=\\\{\}]
{\color{incolor}In [{\color{incolor}12}]:} \PY{k}{def} \PY{n+nf}{start\PYZus{}chess\PYZus{}game}\PY{p}{(}\PY{n+nb+bp}{self}\PY{p}{,} \PY{n}{players}\PY{p}{)}\PY{p}{:}
             \PY{n}{board} \PY{o}{=} \PY{n}{chess}\PY{o}{.}\PY{n}{Board}\PY{p}{(}\PY{p}{)}
             \PY{n}{turn\PYZus{}list} \PY{o}{=} \PY{n+nb}{list}\PY{p}{(}\PY{p}{)}
             \PY{k}{while} \PY{o+ow}{not} \PY{n}{board}\PY{o}{.}\PY{n}{is\PYZus{}game\PYZus{}over}\PY{p}{(}\PY{p}{)}\PY{p}{:}
                 \PY{n}{current\PYZus{}player} \PY{o}{=} \PY{n}{players}\PY{p}{[}\PY{n+nb}{int}\PY{p}{(}\PY{o+ow}{not} \PY{n}{board}\PY{o}{.}\PY{n}{turn}\PY{p}{)}\PY{p}{]}
                 \PY{n}{current\PYZus{}player}\PY{o}{.}\PY{n}{print\PYZus{}board}\PY{p}{(}\PY{n}{current\PYZus{}player}\PY{o}{.}\PY{n}{name}\PY{p}{,} \PY{n}{board}\PY{p}{)}
         
                 \PY{n}{move} \PY{o}{=} \PY{n}{current\PYZus{}player}\PY{o}{.}\PY{n}{get\PYZus{}move}\PY{p}{(}\PY{n}{board}\PY{p}{)}
                 \PY{n}{board}\PY{o}{.}\PY{n}{push}\PY{p}{(}\PY{n}{move}\PY{p}{)}
                 \PY{n}{current\PYZus{}player}\PY{o}{.}\PY{n}{submit\PYZus{}move}\PY{p}{(}\PY{n}{move}\PY{p}{)}
         
                 \PY{n}{turn\PYZus{}list}\PY{o}{.}\PY{n}{append}\PY{p}{(}\PY{n}{board}\PY{o}{.}\PY{n}{fen}\PY{p}{(}\PY{p}{)}\PY{o}{.}\PY{n}{split}\PY{p}{(}\PY{l+s+s2}{\PYZdq{}}\PY{l+s+s2}{ }\PY{l+s+s2}{\PYZdq{}}\PY{p}{)}\PY{p}{[}\PY{l+m+mi}{0}\PY{p}{]}\PY{p}{)}
         
             \PY{n}{current\PYZus{}player} \PY{o}{=} \PY{n}{players}\PY{p}{[}\PY{n+nb}{int}\PY{p}{(}\PY{o+ow}{not} \PY{n}{board}\PY{o}{.}\PY{n}{turn}\PY{p}{)}\PY{p}{]}
             \PY{n}{current\PYZus{}player}\PY{o}{.}\PY{n}{print\PYZus{}board}\PY{p}{(}\PY{n}{current\PYZus{}player}\PY{o}{.}\PY{n}{name}\PY{p}{,} \PY{n}{board}\PY{p}{)}
         
             \PY{n}{result} \PY{o}{=} \PY{n}{Tools}\PY{o}{.}\PY{n}{get\PYZus{}board\PYZus{}result}\PY{p}{(}\PY{n}{board}\PY{p}{)}
             \PY{k}{if} \PY{n}{result} \PY{o+ow}{is} \PY{l+m+mi}{1}\PY{p}{:}
                 \PY{n+nb}{print}\PY{p}{(}\PY{l+s+s2}{\PYZdq{}}\PY{l+s+si}{\PYZob{}\PYZcb{}}\PY{l+s+s2}{ (White) has won}\PY{l+s+s2}{\PYZdq{}}\PY{o}{.}\PY{n}{format}\PY{p}{(}\PY{n}{players}\PY{p}{[}\PY{l+m+mi}{0}\PY{p}{]}\PY{o}{.}\PY{n}{name}\PY{p}{)}\PY{p}{)}
             \PY{k}{elif} \PY{n}{result} \PY{o+ow}{is} \PY{o}{\PYZhy{}}\PY{l+m+mi}{1}\PY{p}{:}
                 \PY{n+nb}{print}\PY{p}{(}\PY{l+s+s2}{\PYZdq{}}\PY{l+s+si}{\PYZob{}\PYZcb{}}\PY{l+s+s2}{ (Black) has won}\PY{l+s+s2}{\PYZdq{}}\PY{o}{.}\PY{n}{format}\PY{p}{(}\PY{n}{players}\PY{p}{[}\PY{l+m+mi}{1}\PY{p}{]}\PY{o}{.}\PY{n}{name}\PY{p}{)}\PY{p}{)}
             \PY{k}{else}\PY{p}{:}
                 \PY{n+nb}{print}\PY{p}{(}\PY{l+s+s2}{\PYZdq{}}\PY{l+s+s2}{Draw}\PY{l+s+s2}{\PYZdq{}}\PY{p}{)}
         
             \PY{n+nb+bp}{self}\PY{o}{.}\PY{n}{groom\PYZus{}board\PYZus{}history}\PY{p}{(}\PY{n}{board}\PY{p}{,} \PY{n}{turn\PYZus{}list}\PY{p}{)}
\end{Verbatim}

    Nach Ende des Spiels muss diese Liste noch in die bestehende Historie
eingepflegt werden. Dies ist in folgendem Code-Snippet zu sehen.

    \begin{Verbatim}[commandchars=\\\{\}]
{\color{incolor}In [{\color{incolor}13}]:} \PY{k}{def} \PY{n+nf}{groom\PYZus{}board\PYZus{}history}\PY{p}{(}\PY{n+nb+bp}{self}\PY{p}{,} \PY{n}{final\PYZus{}board}\PY{p}{,} \PY{n}{turn\PYZus{}list}\PY{p}{)}\PY{p}{:}
             \PY{n}{victory\PYZus{}status} \PY{o}{=} \PY{n}{Tools}\PY{o}{.}\PY{n}{get\PYZus{}board\PYZus{}result}\PY{p}{(}\PY{n}{final\PYZus{}board}\PY{p}{)}
         
             \PY{n}{new\PYZus{}turn\PYZus{}dict} \PY{o}{=} \PY{n+nb}{dict}\PY{o}{.}\PY{n}{fromkeys}\PY{p}{(}\PY{n}{turn\PYZus{}list}\PY{p}{,} \PY{n}{victory\PYZus{}status}\PY{p}{)}
         
             \PY{n}{history} \PY{o}{=} \PY{n}{pd}\PY{o}{.}\PY{n}{read\PYZus{}csv}\PY{p}{(}\PY{n}{HISTORY\PYZus{}FILE\PYZus{}LOC}\PY{p}{)}
             \PY{n}{history\PYZus{}dict} \PY{o}{=} \PY{n+nb}{dict}\PY{p}{(}\PY{n+nb}{zip}\PY{p}{(}\PY{n+nb}{list}\PY{p}{(}\PY{n}{history}\PY{o}{.}\PY{n}{board}\PY{p}{)}\PY{p}{,} \PY{n+nb}{list}\PY{p}{(}\PY{n}{history}\PY{o}{.}\PY{n}{value}\PY{p}{)}\PY{p}{)}\PY{p}{)}
         
             \PY{n}{merged\PYZus{}history\PYZus{}dict} \PY{o}{=} \PY{p}{\PYZob{}} \PY{n}{k}\PY{p}{:} \PY{n}{new\PYZus{}turn\PYZus{}dict}\PY{o}{.}\PY{n}{get}\PY{p}{(}\PY{n}{k}\PY{p}{,} \PY{l+m+mi}{0}\PY{p}{)} \PY{o}{+} 
		\PY{n}{history\PYZus{}dict}\PY{o}{.}\PY{n}{get}\PY{p}{(}\PY{n}{k}\PY{p}{,} \PY{l+m+mi}{0}\PY{p}{)} \PY{k}{for} \PY{n}{k} \PY{o+ow}{in} \PY{n+nb}{set}\PY{p}{(}\PY{n}{new\PYZus{}turn\PYZus{}dict}\PY{p}{)} 
		\PY{o}{|} \PY{n+nb}{set}\PY{p}{(}\PY{n}{history\PYZus{}dict}\PY{p}{)} \PY{p}{\PYZcb{}}
             \PY{n}{merged\PYZus{}history} \PY{o}{=} \PY{n}{pd}\PY{o}{.}\PY{n}{DataFrame}\PY{p}{(}\PY{n+nb}{list}\PY{p}{(}\PY{n}{merged\PYZus{}history\PYZus{}dict}\PY{o}{.}\PY{n}{items}\PY{p}{(}\PY{p}{)}\PY{p}{)}\PY{p}{,}
		 \PY{n}{columns}\PY{o}{=}\PY{p}{[}\PY{l+s+s1}{\PYZsq{}}\PY{l+s+s1}{board}\PY{l+s+s1}{\PYZsq{}}\PY{p}{,}\PY{l+s+s1}{\PYZsq{}}\PY{l+s+s1}{value}\PY{l+s+s1}{\PYZsq{}}\PY{p}{]}\PY{p}{)}
         
             \PY{n}{merged\PYZus{}history}\PY{o}{.}\PY{n}{to\PYZus{}csv}\PY{p}{(}\PY{n}{HISTORY\PYZus{}FILE\PYZus{}LOC}\PY{p}{)}
\end{Verbatim}

    Dabei wird zuerst das Ergebnis ermittelt mittels der bereits
vorgestellten Funktion \texttt{get\_board\_result}. Danach wird ein
Dictionary angelegt, das auf der einen Seite jeden gespeicherten Zustand
des einzupflegenden Spiels als Schlüssel enthält und auf der anderen
Seite als Wert den Ausgang des Spiels. Zur Erinnerung - dieser beträgt 1
bei Sieg von Weiß, 0 bei Unentschieden/Remis und -1 bei Sieg von Schwarz.

Nachdem das Dictionary der neu einzupflegenden Züge angelegt ist, wird
die vorhandene Historie aus der entsprechenden Datei ausgelesen. Dies
wird mittels der in \texttt{panda} enthaltenden Funktion
\texttt{pd.read\_csv} durchgeführt. Anschließend wird auch daraus ein
Dictionary erstellt, indem die Zeile "board" als Schlüssel und die Zeile
"value" als Wert verwendet wird.

Nun werden die beiden Dictionaries zusammengefügt. Dazu wird für jeden
Schlüssel aus dem Dictionary der neu einzupflegenden Spielzüge oder dem
vorhandenen Historie-Dictionary der berechnete Wert aus ersterem auf den
vorhandenen Wert in der Spielhistorie aufaddiert. Dadurch erhalten wird
ein Dictionary, das alle Schachzustände aus der Historie sowie den neuen
Spielzuständen vereint und dessen Werte durch Addition kombiniert,
wodurch der neu berechnete Wert Aufschluss über den wahrscheinlichen
Sieger ausgehend von einem bestimmten Zustand geben kann.

Anschließend wird dieses Dictionary in ein Panda Dataframe umgewandelt,
wobei der Schlüssel für die Zeile "board" und der Wert für die Zeile
"value" verwendet wird, um das DataDrame anschließend wieder zu der
CSV-Datei zu speichern.

Zusätzlich wurde noch ein Parameter \texttt{repeat} eingeführt, um das
Spiel mit den gleichen Einstellungen nochmal wiederholen zu können. Dazu
wird am Ende des Spiels abgefragt, ob das Spiel wiederholt werden soll.
Solange dies mit einer "1" bestätigt wird, wird das Spiel von vorne
gestartet. Andernfalls wird das Spiel und somit das Programm beendet.

    \begin{Verbatim}[commandchars=\\\{\}]
{\color{incolor}In [{\color{incolor}14}]:} \PY{k}{def} \PY{n+nf}{start\PYZus{}chess\PYZus{}game}\PY{p}{(}\PY{n+nb+bp}{self}\PY{p}{,} \PY{n}{players}\PY{p}{)}\PY{p}{:}
             \PY{n}{repeat} \PY{o}{=} \PY{l+m+mi}{1}
             \PY{k}{while} \PY{n}{repeat} \PY{o}{==} \PY{l+m+mi}{1}\PY{p}{:}
         
                 \PY{n}{board} \PY{o}{=} \PY{n}{chess}\PY{o}{.}\PY{n}{Board}\PY{p}{(}\PY{p}{)}
                 \PY{n}{turn\PYZus{}list} \PY{o}{=} \PY{n+nb}{list}\PY{p}{(}\PY{p}{)}
                 \PY{k}{while} \PY{o+ow}{not} \PY{n}{board}\PY{o}{.}\PY{n}{is\PYZus{}game\PYZus{}over}\PY{p}{(}\PY{p}{)}\PY{p}{:}
                     \PY{n}{current\PYZus{}player} \PY{o}{=} \PY{n}{players}\PY{p}{[}\PY{n+nb}{int}\PY{p}{(}\PY{o+ow}{not} \PY{n}{board}\PY{o}{.}\PY{n}{turn}\PY{p}{)}\PY{p}{]}
                     \PY{n}{current\PYZus{}player}\PY{o}{.}\PY{n}{print\PYZus{}board}\PY{p}{(}\PY{n}{current\PYZus{}player}\PY{o}{.}\PY{n}{name}\PY{p}{,} \PY{n}{board}\PY{p}{)}
         
                     \PY{n}{move} \PY{o}{=} \PY{n}{current\PYZus{}player}\PY{o}{.}\PY{n}{get\PYZus{}move}\PY{p}{(}\PY{n}{board}\PY{p}{)}
                     \PY{n}{board}\PY{o}{.}\PY{n}{push}\PY{p}{(}\PY{n}{move}\PY{p}{)}
                     \PY{n}{current\PYZus{}player}\PY{o}{.}\PY{n}{submit\PYZus{}move}\PY{p}{(}\PY{n}{move}\PY{p}{)}
         
                     \PY{n}{turn\PYZus{}list}\PY{o}{.}\PY{n}{append}\PY{p}{(}\PY{n}{board}\PY{o}{.}\PY{n}{fen}\PY{p}{(}\PY{p}{)}\PY{o}{.}\PY{n}{split}\PY{p}{(}\PY{l+s+s2}{\PYZdq{}}\PY{l+s+s2}{ }\PY{l+s+s2}{\PYZdq{}}\PY{p}{)}\PY{p}{[}\PY{l+m+mi}{0}\PY{p}{]}\PY{p}{)}
         
                 \PY{n}{current\PYZus{}player} \PY{o}{=} \PY{n}{players}\PY{p}{[}\PY{n+nb}{int}\PY{p}{(}\PY{o+ow}{not} \PY{n}{board}\PY{o}{.}\PY{n}{turn}\PY{p}{)}\PY{p}{]}
                 \PY{n}{current\PYZus{}player}\PY{o}{.}\PY{n}{print\PYZus{}board}\PY{p}{(}\PY{n}{current\PYZus{}player}\PY{o}{.}\PY{n}{name}\PY{p}{,} \PY{n}{board}\PY{p}{)}
         
                 \PY{n}{result} \PY{o}{=} \PY{n}{Tools}\PY{o}{.}\PY{n}{get\PYZus{}board\PYZus{}result}\PY{p}{(}\PY{n}{board}\PY{p}{)}
                 \PY{k}{if} \PY{n}{result} \PY{o+ow}{is} \PY{l+m+mi}{1}\PY{p}{:}
                     \PY{n+nb}{print}\PY{p}{(}\PY{l+s+s2}{\PYZdq{}}\PY{l+s+si}{\PYZob{}\PYZcb{}}\PY{l+s+s2}{ (White) has won}\PY{l+s+s2}{\PYZdq{}}\PY{o}{.}\PY{n}{format}\PY{p}{(}\PY{n}{players}\PY{p}{[}\PY{l+m+mi}{0}\PY{p}{]}\PY{o}{.}\PY{n}{name}\PY{p}{)}\PY{p}{)}
                 \PY{k}{elif} \PY{n}{result} \PY{o+ow}{is} \PY{o}{\PYZhy{}}\PY{l+m+mi}{1}\PY{p}{:}
                     \PY{n+nb}{print}\PY{p}{(}\PY{l+s+s2}{\PYZdq{}}\PY{l+s+si}{\PYZob{}\PYZcb{}}\PY{l+s+s2}{ (Black) has won}\PY{l+s+s2}{\PYZdq{}}\PY{o}{.}\PY{n}{format}\PY{p}{(}\PY{n}{players}\PY{p}{[}\PY{l+m+mi}{1}\PY{p}{]}\PY{o}{.}\PY{n}{name}\PY{p}{)}\PY{p}{)}
                 \PY{k}{else}\PY{p}{:}
                     \PY{n+nb}{print}\PY{p}{(}\PY{l+s+s2}{\PYZdq{}}\PY{l+s+s2}{Draw}\PY{l+s+s2}{\PYZdq{}}\PY{p}{)}
         
                 \PY{n+nb+bp}{self}\PY{o}{.}\PY{n}{groom\PYZus{}board\PYZus{}history}\PY{p}{(}\PY{n}{board}\PY{p}{,} \PY{n}{turn\PYZus{}list}\PY{p}{)}
                 \PY{n}{repeat} \PY{o}{=} \PY{n+nb}{int}\PY{p}{(}\PY{n+nb}{input}\PY{p}{(}\PY{n}{GAME\PYZus{}FINISHED\PYZus{}MESSAGE}\PY{p}{)}\PY{p}{)}
\end{Verbatim}

    Dies ermöglicht nicht nur eine Verwaltung des Spiels, sondern
gleichzeitig auch eine Speicherung aller möglichen Zustände, die bei den
Evaluierungsfunktionen (beschrieben in Kapitel \ref{evaluierung-eines-gegebenen-schachbretts}) zur Hilfe genommen werden können

    \section{Zugfindung durch KI}\label{zugfindung-durch-ki}

Die zentrale Aufgabe der KI ist es, den bestmöglichen Zug für eine
gegebene Situation zu finden. Dazu wird es
von dem \texttt{ChessMaster} aufgefordert und dazu das aktuelle
board mit übergeben. Die Findung des Zuges findet dann in der Datei
\texttt{player/ai.py} statt:

  \begin{Verbatim}[commandchars=\\\{\}]
{\color{incolor}In [{\color{incolor}18}]:} \PY{n}{OPENING\PYZus{}MAX\PYZus{}FULLMOVE\PYZus{}NUM} \PY{o}{=} \PY{l+m+mi}{6}
         \PY{n}{FINISHING\PYZus{}MAX\PYZus{}PIECES} \PY{o}{=} \PY{l+m+mi}{13}
         \PY{n}{MAX\PYZus{}BOARD\PYZus{}VALUE} \PY{o}{=} \PY{n+nb}{float}\PY{p}{(}\PY{l+s+s2}{\PYZdq{}}\PY{l+s+s2}{inf}\PY{l+s+s2}{\PYZdq{}}\PY{p}{)}
         
         \PY{k}{def} \PY{n+nf}{get\PYZus{}move}\PY{p}{(}\PY{n}{board}\PY{p}{)}\PY{p}{:}
                 \PY{n}{game\PYZus{}status} \PY{o}{=} \PY{l+m+mi}{2}
                 \PY{n}{evaluation\PYZus{}func} \PY{o}{=} \PY{n}{evaluate\PYZus{}board}
                 
                 \PY{n}{opening\PYZus{}book} \PY{o}{=} \PY{n}{import\PYZus{}opening\PYZus{}book}\PY{p}{(}\PY{n}{OPENING\PYZus{}BOOK\PYZus{}LOC}\PY{p}{)}
                 
                 \PY{k}{if} \PY{n}{board}\PY{o}{.}\PY{n}{fullmove\PYZus{}number} \PY{o}{\PYZlt{}}\PY{o}{=} \PY{n}{OPENING\PYZus{}MAX\PYZus{}FULLMOVE\PYZus{}NUM}\PY{p}{:}
                     \PY{n}{game\PYZus{}status} \PY{o}{=} \PY{l+m+mi}{1}
                     \PY{n}{move} \PY{o}{=} \PY{n}{get\PYZus{}opening\PYZus{}move}\PY{p}{(}\PY{n}{board}\PY{p}{,} \PY{n}{opening\PYZus{}book}\PY{p}{)}
                     \PY{k}{if} \PY{o+ow}{not} \PY{n}{move} \PY{o+ow}{is} \PY{k+kc}{None}\PY{p}{:}
                         \PY{k}{return} \PY{n}{move}
                 
                 \PY{n}{white\PYZus{}material} \PY{o}{=} \PY{n}{get\PYZus{}value\PYZus{}by\PYZus{}color}\PY{p}{(}\PY{n}{board}\PY{p}{,} \PY{n}{chess}\PY{o}{.}\PY{n}{WHITE}\PY{p}{,} \PY{k+kc}{False}\PY{p}{)}
                 \PY{n}{black\PYZus{}material} \PY{o}{=} \PY{n}{get\PYZus{}value\PYZus{}by\PYZus{}color}\PY{p}{(}\PY{n}{board}\PY{p}{,} \PY{n}{chess}\PY{o}{.}\PY{n}{BLACK}\PY{p}{,} \PY{k+kc}{False}\PY{p}{)}
                 \PY{k}{if} \PY{n}{white\PYZus{}material} \PY{o}{\PYZlt{}}\PY{o}{=} \PY{n}{FINISHING\PYZus{}MAX\PYZus{}PIECES} \PY{o+ow}{and} \PY{n}{black\PYZus{}material} \PY{o}{\PYZlt{}}\PY{o}{=} \PY{n}{FINISHING\PYZus{}MAX\PYZus{}PIECES}\PY{p}{:}
                     \PY{n}{game\PYZus{}status} \PY{o}{=} \PY{l+m+mi}{3}
                     
                 \PY{n}{pieces\PYZus{}counter\PYZus{}white} \PY{o}{=} \PY{n+nb}{sum}\PY{p}{(}\PY{n+nb}{len}\PY{p}{(}\PY{n}{board}\PY{o}{.}\PY{n}{pieces}\PY{p}{(}\PY{n}{piece\PYZus{}type}\PY{p}{,} \PY{n}{chess}\PY{o}{.}\PY{n}{WHITE}\PY{p}{)}\PY{p}{)} \PY{k}{for} \PY{n}{piece\PYZus{}type} \PY{o+ow}{in} \PY{n}{chess}\PY{o}{.}\PY{n}{PIECE\PYZus{}TYPES}\PY{p}{)}
                 \PY{n}{pieces\PYZus{}counter\PYZus{}black} \PY{o}{=} \PY{n+nb}{sum}\PY{p}{(}\PY{n+nb}{len}\PY{p}{(}\PY{n}{board}\PY{o}{.}\PY{n}{pieces}\PY{p}{(}\PY{n}{piece\PYZus{}type}\PY{p}{,} \PY{n}{chess}\PY{o}{.}\PY{n}{BLACK}\PY{p}{)}\PY{p}{)} \PY{k}{for} \PY{n}{piece\PYZus{}type} \PY{o+ow}{in} \PY{n}{chess}\PY{o}{.}\PY{n}{PIECE\PYZus{}TYPES}\PY{p}{)}
                 
                 \PY{k}{if} \PY{n}{pieces\PYZus{}counter\PYZus{}white} \PY{o}{+} \PY{n}{pieces\PYZus{}counter\PYZus{}black} \PY{o}{\PYZlt{}}\PY{o}{=} \PY{l+m+mi}{5}\PY{p}{:}
                     \PY{n}{board}\PY{o}{.}\PY{n}{evaluation\PYZus{}func} \PY{o}{=} \PY{n}{get\PYZus{}dtz\PYZus{}value}
                     \PY{n}{evaluation\PYZus{}func} \PY{o}{=} \PY{n}{get\PYZus{}dtz\PYZus{}value}
                 
                 \PY{n}{evaluation\PYZus{}funcs\PYZus{}dict} \PY{o}{=} \PY{n}{get\PYZus{}evaluation\PYZus{}funcs\PYZus{}by\PYZus{}dif}\PY{p}{(}\PY{n}{game\PYZus{}status}\PY{p}{,} \PY{l+m+mi}{3}\PY{p}{)}
         
                 \PY{k}{return} \PY{n}{iterative\PYZus{}deepening}\PY{p}{(}\PY{n}{board}\PY{p}{,} \PY{n}{get\PYZus{}max\PYZus{}depth\PYZus{}by\PYZus{}game\PYZus{}status}\PY{p}{(}\PY{n}{game\PYZus{}status}\PY{p}{)}\PY{p}{,} \PY{n}{evaluation\PYZus{}func}\PY{p}{,} \PY{n}{evaluation\PYZus{}funcs\PYZus{}dict}\PY{p}{)}
\end{Verbatim}

    Dabei gibt es 3 erreichbare Status, in der sich das Spiel befinden kann:

\begin{itemize}
\tightlist
\item
  Eröffnungsspiel
\item
  Mittelspiel
\item
  Endspiel
\end{itemize}

Für alle 3 Status unterscheidet sich die Art, wie die KI einen Zug
finden soll. Standardmäßig befindet sich das Spiel im Zeitfenster des
Mittelspiels. Mittels dieser wird im Minimax
Algorithmus (Kapitel \ref{implementierung-des-minimax-algorithmus-mit-alpha-beta-pruning}) nach dem besten Zug gesucht. Dies ändert sich, wenn
eine der zwei folgenden Bedingungen zutrifft:

Die erste ist für den Fall, dass sich das Spiel noch innerhalb der
ersten 10 Züge (5 Züge pro Spieler) befindet. In dem Fall wird es noch
als Eröffnungsspiel gewertet und es wird zuerst ein
Eröffnungsspielzugbuch geprüft, ob dies einen passenden Zug zur
Verfügung hat. Dies ist in Kapitel \ref{einbinden-und-verwendung-von-opening-books} beschrieben. Andernfalls wird
standardmäßig in die ``Iterative Deepening`` Funktion gegeben, die in
Kapitel \ref{implementierung-des-iterative-deepening-algorithmus} beschrieben wird.

Die zweite Möglichkeit ist, dass sich das Spiel im Endspiel befindet.
Dazu wird von beiden Seiten ein Wert der Materialien berechnet. Dies
wird später noch bei den Evaluierungsfunktionen in Kapitel \ref{evaluierung-eines-gegebenen-schachbretts} näher
beschrieben. Ist für beide Spieler der Wert unter 13, so gilt das Spiel
als Endspiel. Dafür werden auch die Evaluierungsfunktionen geändert, um
statt der üblichen Evaluationen den DTZ Wert zu nehmen, der in Kapitel
\ref{einbinden-und-verwendung-von-endspiel-datenbanken} erklärt wird, um per Minimax Algorithmus (Kapitel \ref{implementierung-des-minimax-algorithmus-mit-alpha-beta-pruning}) nach dem besten
Zug zu suchen.

In beiden Fällen wird der \texttt{game\_status} geändert. Anhand dieser
bestimmt die Funktion \texttt{get\_evaluation\_funcs\_by\_dif} die
Gewichtungsfaktoren, die den einzelnen Evaluierungsfunktionen zugewiesen
werden, um diese je nach Spielsituation optimal miteinander zu
kombinieren. Diese Funktion sieht wie folgt aus:

  \begin{Verbatim}[commandchars=\\\{\}]
{\color{incolor}In [{\color{incolor}19}]:} \PY{n}{MAX\PYZus{}DEPTH\PYZus{}START} \PY{o}{=} \PY{l+m+mi}{4}
         \PY{n}{BOARD\PYZus{}VALUE\PYZus{}FACTOR\PYZus{}START} \PY{o}{=} \PY{l+m+mi}{50}
         \PY{n}{ATTACKED\PYZus{}PIECES\PYZus{}FACTOR\PYZus{}START} \PY{o}{=} \PY{l+m+mi}{10}
         \PY{n}{BOARD\PYZus{}POSITIONS\PYZus{}FACTOR\PYZus{}START} \PY{o}{=} \PY{l+m+mi}{50}
         \PY{n}{OPP\PYZus{}BOARD\PYZus{}POSITIONS\PYZus{}FACTOR\PYZus{}START} \PY{o}{=} \PY{l+m+mi}{50}
         \PY{n}{KING\PYZus{}SAFETY\PYZus{}FACTOR\PYZus{}START} \PY{o}{=} \PY{l+m+mi}{5}
         \PY{n}{OPP\PYZus{}KING\PYZus{}SAFETY\PYZus{}FACTOR\PYZus{}START} \PY{o}{=} \PY{l+m+mi}{5}
         \PY{n}{MOBILITY\PYZus{}FACTOR\PYZus{}START} \PY{o}{=} \PY{l+m+mi}{4}
         \PY{n}{HISTORY\PYZus{}FACTOR\PYZus{}START} \PY{o}{=} \PY{l+m+mi}{1}
         
         \PY{n}{MAX\PYZus{}DEPTH\PYZus{}MID} \PY{o}{=} \PY{l+m+mi}{4}
         \PY{n}{BOARD\PYZus{}VALUE\PYZus{}FACTOR\PYZus{}MID} \PY{o}{=} \PY{l+m+mi}{50}
         \PY{n}{ATTACKED\PYZus{}PIECES\PYZus{}FACTOR\PYZus{}MID} \PY{o}{=} \PY{l+m+mi}{10}
         \PY{n}{BOARD\PYZus{}POSITIONS\PYZus{}FACTOR\PYZus{}MID} \PY{o}{=} \PY{l+m+mi}{25}
         \PY{n}{OPP\PYZus{}BOARD\PYZus{}POSITIONS\PYZus{}FACTOR\PYZus{}MID} \PY{o}{=} \PY{l+m+mi}{25}
         \PY{n}{KING\PYZus{}SAFETY\PYZus{}FACTOR\PYZus{}MID} \PY{o}{=} \PY{l+m+mi}{5}
         \PY{n}{OPP\PYZus{}KING\PYZus{}SAFETY\PYZus{}FACTOR\PYZus{}MID} \PY{o}{=} \PY{l+m+mi}{5}
         \PY{n}{MOBILITY\PYZus{}FACTOR\PYZus{}MID} \PY{o}{=} \PY{l+m+mi}{4}
         \PY{n}{HISTORY\PYZus{}FACTOR\PYZus{}MID} \PY{o}{=} \PY{l+m+mi}{0}
         
         \PY{n}{MAX\PYZus{}DEPTH\PYZus{}END} \PY{o}{=} \PY{l+m+mi}{10}
         \PY{n}{BOARD\PYZus{}VALUE\PYZus{}FACTOR\PYZus{}END} \PY{o}{=} \PY{l+m+mi}{50}
         \PY{n}{ATTACKED\PYZus{}PIECES\PYZus{}FACTOR\PYZus{}END} \PY{o}{=} \PY{l+m+mi}{20}
         \PY{n}{BOARD\PYZus{}POSITIONS\PYZus{}FACTOR\PYZus{}END} \PY{o}{=} \PY{l+m+mi}{10}
         \PY{n}{OPP\PYZus{}BOARD\PYZus{}POSITIONS\PYZus{}FACTOR\PYZus{}END} \PY{o}{=} \PY{l+m+mi}{10}
         \PY{n}{KING\PYZus{}SAFETY\PYZus{}FACTOR\PYZus{}END} \PY{o}{=} \PY{l+m+mi}{5}
         \PY{n}{OPP\PYZus{}KING\PYZus{}SAFETY\PYZus{}FACTOR\PYZus{}END} \PY{o}{=} \PY{l+m+mi}{10}
         \PY{n}{MOBILITY\PYZus{}FACTOR\PYZus{}END} \PY{o}{=} \PY{l+m+mi}{10}
         \PY{n}{HISTORY\PYZus{}FACTOR\PYZus{}END} \PY{o}{=} \PY{l+m+mi}{5}
         
         \PY{k}{def} \PY{n+nf}{get\PYZus{}evaluation\PYZus{}funcs\PYZus{}by\PYZus{}dif}\PY{p}{(}\PY{n}{game\PYZus{}status}\PY{p}{,} \PY{n}{difficulty}\PY{p}{)}\PY{p}{:}
             \PY{n}{factor\PYZus{}dict} \PY{o}{=} \PY{n}{get\PYZus{}factors\PYZus{}by\PYZus{}game\PYZus{}status}\PY{p}{(}\PY{n}{game\PYZus{}status}\PY{p}{)}
             \PY{n}{funcs\PYZus{}by\PYZus{}deg\PYZus{}of\PYZus{}dif} \PY{o}{=} \PY{p}{\PYZob{}}
                 \PY{l+m+mi}{1}\PY{p}{:} \PY{p}{\PYZob{}}\PY{n}{get\PYZus{}board\PYZus{}value}\PY{p}{:} \PY{n}{factor\PYZus{}dict}\PY{o}{.}\PY{n}{get}\PY{p}{(}\PY{l+s+s2}{\PYZdq{}}\PY{l+s+s2}{board\PYZus{}value}\PY{l+s+s2}{\PYZdq{}}\PY{p}{)}\PY{p}{\PYZcb{}}\PY{p}{,}
                 \PY{l+m+mi}{2}\PY{p}{:} \PY{p}{\PYZob{}}\PY{n}{get\PYZus{}board\PYZus{}value}\PY{p}{:} \PY{n}{factor\PYZus{}dict}\PY{o}{.}\PY{n}{get}\PY{p}{(}\PY{l+s+s2}{\PYZdq{}}\PY{l+s+s2}{board\PYZus{}value}\PY{l+s+s2}{\PYZdq{}}\PY{p}{)}\PY{p}{,} \PY{n}{get\PYZus{}attacked\PYZus{}pieces\PYZus{}value}\PY{p}{:} \PY{n}{factor\PYZus{}dict}\PY{o}{.}\PY{n}{get}\PY{p}{(}\PY{l+s+s2}{\PYZdq{}}\PY{l+s+s2}{attacked\PYZus{}pieces}\PY{l+s+s2}{\PYZdq{}}\PY{p}{)}\PY{p}{,} \PY{n}{get\PYZus{}board\PYZus{}value\PYZus{}by\PYZus{}history}\PY{p}{:} \PY{n}{factor\PYZus{}dict}\PY{o}{.}\PY{n}{get}\PY{p}{(}\PY{l+s+s2}{\PYZdq{}}\PY{l+s+s2}{history}\PY{l+s+s2}{\PYZdq{}}\PY{p}{)}\PY{p}{\PYZcb{}}\PY{p}{,}
                 \PY{l+m+mi}{3}\PY{p}{:} \PY{p}{\PYZob{}}\PY{n}{get\PYZus{}board\PYZus{}value}\PY{p}{:} \PY{n}{factor\PYZus{}dict}\PY{o}{.}\PY{n}{get}\PY{p}{(}\PY{l+s+s2}{\PYZdq{}}\PY{l+s+s2}{board\PYZus{}value}\PY{l+s+s2}{\PYZdq{}}\PY{p}{)}\PY{p}{,} \PY{n}{get\PYZus{}attacked\PYZus{}pieces\PYZus{}value}\PY{p}{:} \PY{n}{factor\PYZus{}dict}\PY{o}{.}\PY{n}{get}\PY{p}{(}\PY{l+s+s2}{\PYZdq{}}\PY{l+s+s2}{attacked\PYZus{}pieces}\PY{l+s+s2}{\PYZdq{}}\PY{p}{)}\PY{p}{,} \PY{n}{get\PYZus{}board\PYZus{}positions\PYZus{}value}\PY{p}{:} \PY{n}{factor\PYZus{}dict}\PY{o}{.}\PY{n}{get}\PY{p}{(}\PY{l+s+s2}{\PYZdq{}}\PY{l+s+s2}{board\PYZus{}position}\PY{l+s+s2}{\PYZdq{}}\PY{p}{)}\PY{p}{,} \PY{n}{get\PYZus{}opp\PYZus{}board\PYZus{}positions\PYZus{}value}\PY{p}{:} \PY{n}{factor\PYZus{}dict}\PY{o}{.}\PY{n}{get}\PY{p}{(}\PY{l+s+s2}{\PYZdq{}}\PY{l+s+s2}{opp\PYZus{}board\PYZus{}position}\PY{l+s+s2}{\PYZdq{}}\PY{p}{)}\PY{p}{,} \PY{n}{calculate\PYZus{}king\PYZus{}zone\PYZus{}safety}\PY{p}{:} \PY{n}{factor\PYZus{}dict}\PY{o}{.}\PY{n}{get}\PY{p}{(}\PY{l+s+s2}{\PYZdq{}}\PY{l+s+s2}{king\PYZus{}safety}\PY{l+s+s2}{\PYZdq{}}\PY{p}{)}\PY{p}{,} \PY{n}{calculate\PYZus{}opp\PYZus{}king\PYZus{}zone\PYZus{}safety}\PY{p}{:} \PY{n}{factor\PYZus{}dict}\PY{o}{.}\PY{n}{get}\PY{p}{(}\PY{l+s+s2}{\PYZdq{}}\PY{l+s+s2}{opp\PYZus{}king\PYZus{}safety}\PY{l+s+s2}{\PYZdq{}}\PY{p}{)}\PY{p}{,} \PY{n}{calculate\PYZus{}mobility\PYZus{}value}\PY{p}{:} \PY{n}{factor\PYZus{}dict}\PY{o}{.}\PY{n}{get}\PY{p}{(}\PY{l+s+s2}{\PYZdq{}}\PY{l+s+s2}{mobility}\PY{l+s+s2}{\PYZdq{}}\PY{p}{)}\PY{p}{\PYZcb{}}
             \PY{p}{\PYZcb{}}
             \PY{k}{return} \PY{n}{funcs\PYZus{}by\PYZus{}deg\PYZus{}of\PYZus{}dif}\PY{o}{.}\PY{n}{get}\PY{p}{(}\PY{n}{difficulty}\PY{p}{)}
         
         \PY{k}{def} \PY{n+nf}{get\PYZus{}factors\PYZus{}by\PYZus{}game\PYZus{}status}\PY{p}{(}\PY{n}{game\PYZus{}status}\PY{p}{)}\PY{p}{:}
             \PY{k}{return} \PY{p}{\PYZob{}}
                 \PY{l+m+mi}{1}\PY{p}{:} \PY{p}{\PYZob{}}
                     \PY{l+s+s2}{\PYZdq{}}\PY{l+s+s2}{board\PYZus{}value}\PY{l+s+s2}{\PYZdq{}}\PY{p}{:} \PY{n}{BOARD\PYZus{}VALUE\PYZus{}FACTOR\PYZus{}START}\PY{p}{,}
                     \PY{l+s+s2}{\PYZdq{}}\PY{l+s+s2}{attacked\PYZus{}pieces}\PY{l+s+s2}{\PYZdq{}}\PY{p}{:} \PY{n}{ATTACKED\PYZus{}PIECES\PYZus{}FACTOR\PYZus{}START}\PY{p}{,}
                     \PY{l+s+s2}{\PYZdq{}}\PY{l+s+s2}{board\PYZus{}position}\PY{l+s+s2}{\PYZdq{}}\PY{p}{:} \PY{n}{BOARD\PYZus{}POSITIONS\PYZus{}FACTOR\PYZus{}START}\PY{p}{,}
                     \PY{l+s+s2}{\PYZdq{}}\PY{l+s+s2}{opp\PYZus{}board\PYZus{}position}\PY{l+s+s2}{\PYZdq{}}\PY{p}{:} \PY{n}{OPP\PYZus{}BOARD\PYZus{}POSITIONS\PYZus{}FACTOR\PYZus{}START}\PY{p}{,}
                     \PY{l+s+s2}{\PYZdq{}}\PY{l+s+s2}{king\PYZus{}safety}\PY{l+s+s2}{\PYZdq{}}\PY{p}{:} \PY{n}{KING\PYZus{}SAFETY\PYZus{}FACTOR\PYZus{}START}\PY{p}{,}
                     \PY{l+s+s2}{\PYZdq{}}\PY{l+s+s2}{opp\PYZus{}king\PYZus{}safety}\PY{l+s+s2}{\PYZdq{}}\PY{p}{:} \PY{n}{OPP\PYZus{}KING\PYZus{}SAFETY\PYZus{}FACTOR\PYZus{}START}\PY{p}{,}
                     \PY{l+s+s2}{\PYZdq{}}\PY{l+s+s2}{mobility}\PY{l+s+s2}{\PYZdq{}}\PY{p}{:} \PY{n}{MOBILITY\PYZus{}FACTOR\PYZus{}START}\PY{p}{,}
                     \PY{l+s+s2}{\PYZdq{}}\PY{l+s+s2}{history}\PY{l+s+s2}{\PYZdq{}}\PY{p}{:} \PY{n}{HISTORY\PYZus{}FACTOR\PYZus{}START}
                 \PY{p}{\PYZcb{}}\PY{p}{,}
                 \PY{l+m+mi}{2}\PY{p}{:} \PY{p}{\PYZob{}}
                     \PY{l+s+s2}{\PYZdq{}}\PY{l+s+s2}{board\PYZus{}value}\PY{l+s+s2}{\PYZdq{}}\PY{p}{:} \PY{n}{BOARD\PYZus{}VALUE\PYZus{}FACTOR\PYZus{}MID}\PY{p}{,}
                     \PY{l+s+s2}{\PYZdq{}}\PY{l+s+s2}{attacked\PYZus{}pieces}\PY{l+s+s2}{\PYZdq{}}\PY{p}{:} \PY{n}{ATTACKED\PYZus{}PIECES\PYZus{}FACTOR\PYZus{}MID}\PY{p}{,}
                     \PY{l+s+s2}{\PYZdq{}}\PY{l+s+s2}{board\PYZus{}position}\PY{l+s+s2}{\PYZdq{}}\PY{p}{:} \PY{n}{BOARD\PYZus{}POSITIONS\PYZus{}FACTOR\PYZus{}MID}\PY{p}{,}
                     \PY{l+s+s2}{\PYZdq{}}\PY{l+s+s2}{opp\PYZus{}board\PYZus{}position}\PY{l+s+s2}{\PYZdq{}}\PY{p}{:} \PY{n}{OPP\PYZus{}BOARD\PYZus{}POSITIONS\PYZus{}FACTOR\PYZus{}MID}\PY{p}{,}
                     \PY{l+s+s2}{\PYZdq{}}\PY{l+s+s2}{king\PYZus{}safety}\PY{l+s+s2}{\PYZdq{}}\PY{p}{:} \PY{n}{KING\PYZus{}SAFETY\PYZus{}FACTOR\PYZus{}MID}\PY{p}{,}
                     \PY{l+s+s2}{\PYZdq{}}\PY{l+s+s2}{opp\PYZus{}king\PYZus{}safety}\PY{l+s+s2}{\PYZdq{}}\PY{p}{:} \PY{n}{OPP\PYZus{}KING\PYZus{}SAFETY\PYZus{}FACTOR\PYZus{}MID}\PY{p}{,}
                     \PY{l+s+s2}{\PYZdq{}}\PY{l+s+s2}{mobility}\PY{l+s+s2}{\PYZdq{}}\PY{p}{:} \PY{n}{MOBILITY\PYZus{}FACTOR\PYZus{}MID}\PY{p}{,}
                     \PY{l+s+s2}{\PYZdq{}}\PY{l+s+s2}{history}\PY{l+s+s2}{\PYZdq{}}\PY{p}{:} \PY{n}{HISTORY\PYZus{}FACTOR\PYZus{}MID}
                 \PY{p}{\PYZcb{}}\PY{p}{,}
                 \PY{l+m+mi}{3}\PY{p}{:} \PY{p}{\PYZob{}}
                     \PY{l+s+s2}{\PYZdq{}}\PY{l+s+s2}{board\PYZus{}value}\PY{l+s+s2}{\PYZdq{}}\PY{p}{:} \PY{n}{BOARD\PYZus{}VALUE\PYZus{}FACTOR\PYZus{}END}\PY{p}{,}
                     \PY{l+s+s2}{\PYZdq{}}\PY{l+s+s2}{attacked\PYZus{}pieces}\PY{l+s+s2}{\PYZdq{}}\PY{p}{:} \PY{n}{ATTACKED\PYZus{}PIECES\PYZus{}FACTOR\PYZus{}END}\PY{p}{,}
                     \PY{l+s+s2}{\PYZdq{}}\PY{l+s+s2}{board\PYZus{}position}\PY{l+s+s2}{\PYZdq{}}\PY{p}{:} \PY{n}{BOARD\PYZus{}POSITIONS\PYZus{}FACTOR\PYZus{}END}\PY{p}{,}
                     \PY{l+s+s2}{\PYZdq{}}\PY{l+s+s2}{opp\PYZus{}board\PYZus{}position}\PY{l+s+s2}{\PYZdq{}}\PY{p}{:} \PY{n}{OPP\PYZus{}BOARD\PYZus{}POSITIONS\PYZus{}FACTOR\PYZus{}END}\PY{p}{,}
                     \PY{l+s+s2}{\PYZdq{}}\PY{l+s+s2}{king\PYZus{}safety}\PY{l+s+s2}{\PYZdq{}}\PY{p}{:} \PY{n}{KING\PYZus{}SAFETY\PYZus{}FACTOR\PYZus{}END}\PY{p}{,}
                     \PY{l+s+s2}{\PYZdq{}}\PY{l+s+s2}{opp\PYZus{}king\PYZus{}safety}\PY{l+s+s2}{\PYZdq{}}\PY{p}{:} \PY{n}{OPP\PYZus{}KING\PYZus{}SAFETY\PYZus{}FACTOR\PYZus{}END}\PY{p}{,}
                     \PY{l+s+s2}{\PYZdq{}}\PY{l+s+s2}{mobility}\PY{l+s+s2}{\PYZdq{}}\PY{p}{:} \PY{n}{MOBILITY\PYZus{}FACTOR\PYZus{}END}\PY{p}{,}
                     \PY{l+s+s2}{\PYZdq{}}\PY{l+s+s2}{history}\PY{l+s+s2}{\PYZdq{}}\PY{p}{:} \PY{n}{HISTORY\PYZus{}FACTOR\PYZus{}END}
                 \PY{p}{\PYZcb{}}
             \PY{p}{\PYZcb{}}\PY{p}{[}\PY{n}{game\PYZus{}status}\PY{p}{]}
\end{Verbatim}

    Dabei wird erst über die Funktion
\texttt{get\_factors\_by\_game\_status} ein passendes Dictionary mit den
dem Spielstatus entsprechenden Gewichtungsfaktoren für jede einzelne
Evaluierungsfunktion erstellt. Dann werden je nach Schwierigkeitsgrad
die durchzuführenden Evaluierungen mitsamt dem aus dem Dictionary
passenden Gewichtungsfaktoren zurückgegeben.

Nachdem diese Schritte durchgeführt wurden und nicht schon bereits
vorher ein Zug durch die Eröffnungs- oder Endspieldatenbank
zurückgegeben wurde, wird die Funktion des ``Iterative Deepening``, die in
Kapitel \ref{implementierung-des-iterative-deepening-algorithmus} erklärt wird, aufgerufen und das der Rückgabewert dieser
Funktion als durchzuführender Zug ausgewählt und somit zurückgegeben.
Dazu wird zunächst allerdings noch die maximale Tiefe anhand des
Spielstatus ermittelt:

    \begin{Verbatim}[commandchars=\\\{\}]
{\color{incolor}In [{\color{incolor}20}]:} \PY{k}{def} \PY{n+nf}{get\PYZus{}max\PYZus{}depth\PYZus{}by\PYZus{}game\PYZus{}status}\PY{p}{(}\PY{n}{game\PYZus{}status}\PY{p}{)}\PY{p}{:}
             \PY{k}{return} \PY{p}{\PYZob{}}
                 \PY{l+m+mi}{1}\PY{p}{:} \PY{n}{MAX\PYZus{}DEPTH\PYZus{}START}\PY{p}{,} 
                 \PY{l+m+mi}{2}\PY{p}{:} \PY{n}{MAX\PYZus{}DEPTH\PYZus{}MID}\PY{p}{,}
                 \PY{l+m+mi}{3}\PY{p}{:} \PY{n}{MAX\PYZus{}DEPTH\PYZus{}END}
             \PY{p}{\PYZcb{}}\PY{p}{[}\PY{n}{game\PYZus{}status}\PY{p}{]}
\end{Verbatim}

    Der erhaltene Wert wird dann zusammen mit dem aktuellen Spielzustand in
die \texttt{iterative\_deepening} Funktion gegeben, die dann den
optimalen Zug ermitteln soll.

    \section{Einbinden und Verwendung von
Opening-Books}\label{einbinden-und-verwendung-von-opening-books}

Im Fall, dass sich das Spiel noch im Anfangszustand befindet, kann ein
Opening Book zur Hilfe genommen werden, um einen passenden Zug zu
finden.

Opening-Books sind ein wesentlicher Bestandteil der Schach-KI, da diese
aus einer großen Anzahl von Eröffnungszügen bestehen. Hierbei wurden die
Opening-Books bereits analysiert und konnten durch Überprüfungen als
gute Eröffnungen identifiziert werden.

Zur Verwendung dieser Bücher müssen diese Dateien, die im .bin-Format
vorliegen, in eine Variable geladen werden, da später auf diese Bücher
mittels Funktionen zugegriffen werden soll. Damit die Eröffnungszüge in
eine Variable gespeichert werden können, muss der Pfad zu dem Buch
vorliegen, der im folgenden Codeausschnitt in eine globale Konstante
\texttt{OPENING\_BOOK\_LOC} geschrieben wurde. Dies findet ebenfalls in
der Datei \texttt{player/ai.py}statt.

    \begin{Verbatim}[commandchars=\\\{\}]
{\color{incolor}In [{\color{incolor}21}]:} \PY{n}{OPENING\PYZus{}BOOK\PYZus{}LOC} \PY{o}{=} \PY{l+s+s2}{\PYZdq{}}\PY{l+s+s2}{res/polyglot/Performance.bin}\PY{l+s+s2}{\PYZdq{}}
\end{Verbatim}

    Für das Laden eines Buches in eine Variable wird die Funktion
\texttt{import\_opening\_book(self,\ book\_location)} genutzt. Diese hat
die Eigenschaft, dass sie als Übergabeparameter eine Konstante erhält,
die den Pfad zu dem zu importierenden Opening-Book enthält.

Innerhalb der \texttt{import\_opening\_book} Funktion wird mittels einer
If-Verzweigung überprüft, ob der angegebene Pfad eine Datei ist. Sollte
dies zutreffen, dann wird, inklusive des Buchpfades, die Funktion
\texttt{polyglot.open\_reader} aus der ``chess`` Bibliothek aufgerufen.
Diese Funktion liefert als Rückgabewert das Opening-Book, welches an die
Stelle zurückgegeben wird, an der die Funktion
\texttt{import\_opening\_book} aufgerufen wird.

Sollte der Pfad in der Variable \texttt{book\_location} keine Datei
sein, so wird ein \emph{File Not Found}-Fehler geworfen, der dabei den
vermeintlichen Pfad mit übergibt.

    \begin{Verbatim}[commandchars=\\\{\}]
{\color{incolor}In [{\color{incolor}22}]:} \PY{k}{def} \PY{n+nf}{import\PYZus{}opening\PYZus{}book}\PY{p}{(}\PY{n}{book\PYZus{}location}\PY{p}{)}\PY{p}{:}
                 \PY{l+s+sd}{\PYZsq{}\PYZsq{}\PYZsq{}}
         \PY{l+s+sd}{        load an opening book}
         \PY{l+s+sd}{        raise an error if system cannot find the opening\PYZhy{}book file}
         \PY{l+s+sd}{        \PYZsq{}\PYZsq{}\PYZsq{}}
                 \PY{k}{if} \PY{n}{os}\PY{o}{.}\PY{n}{path}\PY{o}{.}\PY{n}{isfile}\PY{p}{(}\PY{n}{book\PYZus{}location}\PY{p}{)}\PY{p}{:}
                     \PY{k}{return} \PY{n}{chess}\PY{o}{.}\PY{n}{polyglot}\PY{o}{.}\PY{n}{open\PYZus{}reader}\PY{p}{(}\PY{n}{book\PYZus{}location}\PY{p}{)}
                 \PY{k}{else}\PY{p}{:}
                     \PY{k}{raise} \PY{n+ne}{FileNotFoundError}\PY{p}{(}
                         \PY{n}{errno}\PY{o}{.}\PY{n}{ENOENT}\PY{p}{,} \PY{n}{os}\PY{o}{.}\PY{n}{strerror}\PY{p}{(}\PY{n}{errno}\PY{o}{.}\PY{n}{ENOENT}\PY{p}{)}\PY{p}{,} \PY{n}{book\PYZus{}location}\PY{p}{)}
\end{Verbatim}

    Die Funktion zum Importieren des Opening-Books wird im Konstruktor der
KI-Klasse aufgerufen, da das Opening-Book bei jeder Verwendung direkt zu
Beginn von der KI benötigt wird und somit direkt verfügbar sein muss.
Das bringt den Vorteil, dass eine Klassenvariable vorliegt, die von
allen Funktionen der Klasse KI verwendet werden kann, sobald der
Konstruktor ausgeführt wurde.

Als Übergabewert wird die zuvor definierte Konstante
\texttt{OPENING\_BOOK\_LOC} übergeben, die den Pfad zu einem
Opening-Book enthält.

    Um das Opening-Book verwenden zu können und aus diesem mögliche
Eröffnungsstrategien verwenden zu können wird die Funktion
\texttt{get\_opening\_move(self,\ board,\ opening\_book)} benötigt, die
als Übergabewerte das aktuelle Schachbrett und das importierte Book
erhält.

Der aktuelle Spielstand in Form des Schachbretts wird benötigt damit das
Opening-Book weiß auf welche Situation reagiert werden muss.

Der Sinn der Funktion \texttt{get\_opening\_move} ist es einen Schachzug
als Objekt \emph{chess.Move} zurückzugeben, der laut des Opening-Books
in dieser Situation angebracht ist. Dafür wird zuerst überprüft, ob die
übergebene Variable \texttt{opening\_book} durch den Konstruktor und dem
damit einhergehenden Aufruf der Funktion \texttt{import\_opening\_book}
korrekt initialisiert wurde.

Sollte dies nicht der Fall sein wird von der Funktion ein \texttt{None}
zurückgegeben. Der Wert \texttt{None} ist in der Sprache Python der
leere Zustand.

Falls das Opening-Book korrekt geladen werden konnte wird durch ein
try-except versucht einen passenden Schachzug zu finden. Hierbei wird
ein try-except verwendet, da die aufgerufene Funktion
\texttt{opening\_book.find(board)} einen \emph{Index Error} wirft, falls
das Opening-Book keinen passenden Zug kennt. Wenn dieser Fall eintreten
sollte wird ebenfalls ein \texttt{None} zurückgegeben. Sollte jedoch das
Buch einen passenden Zug haben, dann wird dieser Zug in eine Variable
\texttt{main\_entry} geladen und aus dieser der Zug extrahiert.

Nachdem dies durchgeführt wurde, muss das Opening-Book geschlossen
werden, damit es bei der nächsten Verwendung korrekt genutzt werden
kann. Ebenfalls wird der erfolgreich extrahierte Zug zurückgegeben.

\begin{Verbatim}[commandchars=\\\{\}]
{\color{incolor}In [{\color{incolor}23}]:} \PY{k}{def} \PY{n+nf}{get\PYZus{}opening\PYZus{}move}\PY{p}{(}\PY{n}{board}\PY{p}{,} \PY{n}{opening\PYZus{}book}\PY{p}{)}\PY{p}{:}
                 \PY{l+s+sd}{\PYZsq{}\PYZsq{}\PYZsq{}}
         \PY{l+s+sd}{        get the current board and return move, as string, for this situation}
         \PY{l+s+sd}{        \PYZsq{}\PYZsq{}\PYZsq{}}
                 \PY{k}{if} \PY{o+ow}{not} \PY{p}{(}\PY{n}{opening\PYZus{}book} \PY{o+ow}{is} \PY{k+kc}{None}\PY{p}{)}\PY{p}{:}
                     \PY{k}{try}\PY{p}{:}
                         \PY{n}{main\PYZus{}entry} \PY{o}{=} \PY{n}{opening\PYZus{}book}\PY{o}{.}\PY{n}{find}\PY{p}{(}\PY{n}{board}\PY{p}{)}
                         \PY{n}{move} \PY{o}{=} \PY{n}{main\PYZus{}entry}\PY{o}{.}\PY{n}{move}\PY{p}{(}\PY{p}{)}
                         \PY{n}{opening\PYZus{}book}\PY{o}{.}\PY{n}{close}\PY{p}{(}\PY{p}{)}
                         \PY{k}{return} \PY{n}{move}
                     \PY{k}{except} \PY{n+ne}{IndexError}\PY{p}{:}
                         \PY{k}{return} \PY{k+kc}{None}
                 \PY{k}{else}\PY{p}{:}
                     \PY{k}{return} \PY{k+kc}{None}
\end{Verbatim}

    \section{Einbinden und Verwendung von Endspiel
Datenbanken}\label{einbinden-und-verwendung-von-endspiel-datenbanken}

Analog dazu kann im Fall, dass sich das Spiel dem Ende neigt, eine
Endspiel Datenbank zur Zugfindung zur Hilfe genommen werden. Auch die
Einbindung einer solchen Datenbank wird direkt in der Datei
\texttt{player/ai.py} vorgenommen.

Dabei befinden sich auf dem Schachfeld nur noch wenige Figuren. Analog
zu den Opening-Books wurden bereits viele Endspiele analysiert und für
Spielsituationen die optimalen Züge evaluiert. Diese Informationen sind
in den Endspiel Datenbanken persistiert, die eine Auskunft darüber
geben welche der spielenden Parteien am Gewinnen bzw. Verlieren ist und
wie weit diese von einem Schlagzug entfernt sind. Näheres dazu ist in
Kapitel \ref{einbinden-und-verwendung-von-endspiel-datenbanken} zu finden.

Durch diese Informationen gepaart mit dem Minimax-Algorithmus lässt sich
der optimale Zug im Endspiel finden.

Das Einbinden der Endspiel Datenbank ähnelt sehr stark dem der
Opening-Books. Ebenfalls muss zuerst in einer Konstanten der Pfad zu der
Endspiel Datenbank angegeben werden.

    \begin{Verbatim}[commandchars=\\\{\}]
{\color{incolor}In [{\color{incolor}24}]:} \PY{n}{SYZYGY\PYZus{}LOC} \PY{o}{=} \PY{l+s+s2}{\PYZdq{}}\PY{l+s+s2}{res/syzygy}\PY{l+s+s2}{\PYZdq{}}
\end{Verbatim}

    Ebenfalls analog zu dem Vorgehen bei Opening-Books wird in der Funktion
\texttt{import\_syzygy(self,\ syzygy\_location)}, zum Importieren der
Datenbank, der Pfad übergeben, der zu den benötigten Daten führt. Der
Unterschied in dieser Funktion zu der bei den Opening-Books liegt daran,
dass die Daten der Datenbank sich in mehreren Dateien befinden und diese
sich in einem Ordner befinden. Dafür muss überprüft werden, ob der
übergebene Pfad tatsächlich auf ein Verzeichnis verweist. Falls dies
gegeben ist, dann kann die Funktion
\texttt{syzygy.open\_tablebase(syzygy\_location)} der ``chess``-Bibliothek
aufgerufen werden, die die Datenbank importiert. Nachdem die Daten
innerhalb dieser Funktion vorliegen können sie zurückgegeben werden.

Sollte der übergebene Pfad nicht auf ein verfügbares Verzeichnis
verweisen, dann wird ebenfalls ein \emph{File Not Found Error} geworfen,
der den angegebenen Pfad enthält.

    \begin{Verbatim}[commandchars=\\\{\}]
{\color{incolor}In [{\color{incolor}25}]:} \PY{k}{def} \PY{n+nf}{import\PYZus{}syzygy}\PY{p}{(}\PY{n}{syzygy\PYZus{}location}\PY{p}{)}\PY{p}{:}
                 \PY{l+s+sd}{\PYZsq{}\PYZsq{}\PYZsq{}}
         \PY{l+s+sd}{        load a syzygy tablebase}
         \PY{l+s+sd}{        raise an error if system cannot find the file}
         \PY{l+s+sd}{        \PYZsq{}\PYZsq{}\PYZsq{}}
                 \PY{k}{if} \PY{n}{os}\PY{o}{.}\PY{n}{path}\PY{o}{.}\PY{n}{isdir}\PY{p}{(}\PY{n}{syzygy\PYZus{}location}\PY{p}{)}\PY{p}{:}
                     \PY{k}{return} \PY{n}{chess}\PY{o}{.}\PY{n}{syzygy}\PY{o}{.}\PY{n}{open\PYZus{}tablebase}\PY{p}{(}\PY{n}{syzygy\PYZus{}location}\PY{p}{)}
                 \PY{k}{else}\PY{p}{:}
                     \PY{k}{raise} \PY{n+ne}{FileNotFoundError}\PY{p}{(}
                         \PY{n}{errno}\PY{o}{.}\PY{n}{ENOENT}\PY{p}{,} \PY{n}{os}\PY{o}{.}\PY{n}{strerror}\PY{p}{(}\PY{n}{errno}\PY{o}{.}\PY{n}{ENOENT}\PY{p}{)}\PY{p}{,} \PY{n}{syzygy\PYZus{}location}\PY{p}{)}
\end{Verbatim}

    Wie bereits aus dem Importieren der Opening-Books bekannt, wird die
benötigte Funktion im Konstruktor der Klasse KI aufgerufen.

    Zur Berechnung der Partei, die sich aktuell im Vorteil befindet, wird die
folgende Funktion \texttt{get\_dtz\_value(self,\ tablebase,\ board)}
benötigt, die den DTZ-Wert berechnet. Dieser Wert wird auf der Basis des
aktuellen Spielzustandes in Form der Variable \texttt{board} und der
Datenbank in Form der Variable \texttt{tablebase} übergeben.

Daraufhin wird mittels eines try-except und der Funktion
\texttt{probe\_dtz(board)} versucht den DTZ Wert zu bestimmen. Das
Abfangen eines möglichen \emph{Key Error} wird in diesem Fall benötigt,
da die Funktion \texttt{probe\_dtz} der ``chess`` Bibliothek diesen Fehler
wirft, wenn der aktuelle Spielzustand zu viele Steine, oder
Informationen zu en passant oder einer Rochade enthält.

Die in der Künstlichen Intelligenz verwendete Datenbank unterstützt
Endspiele mit bis zu fünf Steinen, jedoch wird dabei nicht auf die
Sonderregeln en passant und Rochade geachtet.

Wenn der DTZ-Wert bestimmt werden konnte wird diese an die Stelle
zurückgegeben an der die Funktion \texttt{get\_dtz\_value} aufgerufen
wurde. Falls es zu einem \emph{Key Error} kommen sollte wird der Wert
\texttt{None} zurückgegeben.

    \begin{Verbatim}[commandchars=\\\{\}]
{\color{incolor}In [{\color{incolor}26}]:} \PY{k}{def} \PY{n+nf}{get\PYZus{}dtz\PYZus{}value}\PY{p}{(}\PY{n}{tablebase}\PY{p}{,} \PY{n}{board}\PY{p}{)}\PY{p}{:}
                 \PY{k}{try}\PY{p}{:}
                     \PY{k}{return} \PY{n}{tablebase}\PY{o}{.}\PY{n}{probe\PYZus{}dtz}\PY{p}{(}\PY{n}{board}\PY{p}{)}
                 \PY{k}{except} \PY{n+ne}{KeyError}\PY{p}{:}
                     \PY{k}{return} \PY{k+kc}{None}
\end{Verbatim}

    \section{Implementierung des Iterative Deepening
Algorithmus}\label{implementierung-des-iterative-deepening-algorithmus}

Um den besten Zug zu finden muss eine Schach-KI im Optimalfall alle
möglichen Züge bis zum Ende des Spiels durchgehen, diese bewerten und
den aussichtsreichsten Zug wählen. Da dies aber technisch nicht
realistisch ist, werden die Züge nur
bis zu einer gewissen Tiefe angeschaut und die Zustände über mitgegebene
Evaluierungsfunktionen bewertet.

Die KI geht dabei dann wie folgt vor: Alle möglichen Züge werden
durchgegangen. Die daraus entstehenden Zustände werden analysiert und
evaluiert. Dabei jedoch zählt nicht nur der direkt erreichbare Zustand,
sondern auch die aus diesem Zustand erreichbaren Zustände und so weiter.
Aus diesem Grund wird immer bis zu einer bestimmten Tiefe in die Züge
hineingeschaut und die sich daraus ergebenden Zustände evaluiert.

Um dies jedoch nicht fest immer bis zu einer bestimmten Tiefe durchgehen
zu lassen, sondern variabel anzupassen, je nachdem wie viele Züge von
dem gegebenen Zustand aus möglich sind, kann ein Zeitlimit dienen. Dabei
wird anfangs die Tiefe auf 1 gesetzt und dann mittels des
Minimax-Algorithmus die Züge evaluiert. Danach wird die Tiefe um 1
erhöht und erneut der Minimax-Algorithmus angewandt. Dies wird solange
wiederholt, bis die angegebene Zeit abgelaufen ist. Dieser Algorithmus
nennt sich ``Iterative Deepening``. Nähere Informationen dazu sind in
Kapitel \ref{implementierung-des-iterative-deepening-algorithmus} aufgeführt.

Dem Algorithmus muss dazu der aktuelle Zustand sowie eine maximale Tiefe
mitgegeben werden. Ist diese erreicht bricht der Algorithmus ab,
unabhängig davon, ob das Zeitlimit überschritten ist oder nicht.
Zunächst müssen beim Ausführen einige Werte festgelegt werden. Der
folgende Algorithmus zeigt, wie der Prozess des Iterative Deepening in
der \texttt{player/ai.py} implementiert ist.

    \begin{Verbatim}[commandchars=\\\{\}]
{\color{incolor}In [{\color{incolor}27}]:} \PY{k}{def} \PY{n+nf}{iterative\PYZus{}deepening}\PY{p}{(}\PY{n}{board}\PY{p}{,} \PY{n}{max\PYZus{}depth}\PY{p}{,} \PY{n}{evaluation\PYZus{}func}\PY{p}{,} \PY{n}{evaluation\PYZus{}funcs\PYZus{}dict}\PY{p}{)}\PY{p}{:}
             \PY{n}{depth} \PY{o}{=} \PY{l+m+mi}{1}
         
             \PY{n}{time\PYZus{}limit} \PY{o}{=} \PY{n}{get\PYZus{}timeout\PYZus{}by\PYZus{}dif}\PY{p}{(}\PY{l+m+mi}{3}\PY{p}{)}
         
             \PY{n}{start\PYZus{}time} \PY{o}{=} \PY{n+nb}{int}\PY{p}{(}\PY{n}{time}\PY{o}{.}\PY{n}{time}\PY{p}{(}\PY{p}{)}\PY{p}{)}
             \PY{n}{end\PYZus{}time} \PY{o}{=} \PY{n}{start\PYZus{}time} \PY{o}{+} \PY{n}{time\PYZus{}limit}
             \PY{n}{current\PYZus{}time} \PY{o}{=} \PY{n}{start\PYZus{}time}
         
             \PY{n}{player} \PY{o}{=} \PY{n+nb}{bool}\PY{p}{(}\PY{n}{board}\PY{o}{.}\PY{n}{turn}\PY{p}{)}
             \PY{n}{best\PYZus{}possible\PYZus{}result} \PY{o}{=} \PY{n}{get\PYZus{}best\PYZus{}possible\PYZus{}result}\PY{p}{(}\PY{n}{board}\PY{p}{,} \PY{n}{player}\PY{p}{)}
         
             \PY{n}{legal\PYZus{}moves} \PY{o}{=} \PY{n+nb}{list}\PY{p}{(}\PY{n}{board}\PY{o}{.}\PY{n}{legal\PYZus{}moves}\PY{p}{)}
             \PY{k}{while} \PY{n}{current\PYZus{}time} \PY{o}{\PYZlt{}} \PY{n}{end\PYZus{}time} \PY{o+ow}{and} \PY{n}{depth} \PY{o}{\PYZlt{}}\PY{o}{=} \PY{n}{max\PYZus{}depth}\PY{p}{:}
                 \PY{n}{move\PYZus{}val\PYZus{}dict} \PY{o}{=} \PY{p}{\PYZob{}}\PY{p}{\PYZcb{}}
         
                 \PY{n}{best\PYZus{}value} \PY{o}{=} \PY{n+nb}{float}\PY{p}{(}\PY{l+s+s1}{\PYZsq{}}\PY{l+s+s1}{\PYZhy{}inf}\PY{l+s+s1}{\PYZsq{}}\PY{p}{)}
                 \PY{n}{best\PYZus{}move} \PY{o}{=} \PY{n}{legal\PYZus{}moves}\PY{p}{[}\PY{l+m+mi}{0}\PY{p}{]}
         
                 \PY{k}{for} \PY{n}{move} \PY{o+ow}{in} \PY{n}{legal\PYZus{}moves}\PY{p}{:}
                     \PY{n}{tmp\PYZus{}board} \PY{o}{=} \PY{n}{chess}\PY{o}{.}\PY{n}{Board}\PY{p}{(}\PY{n+nb}{str}\PY{p}{(}\PY{n}{board}\PY{o}{.}\PY{n}{fen}\PY{p}{(}\PY{p}{)}\PY{p}{)}\PY{p}{)}
                     \PY{n}{tmp\PYZus{}board}\PY{o}{.}\PY{n}{push}\PY{p}{(}\PY{n}{move}\PY{p}{)}
                     \PY{n}{value} \PY{o}{=} \PY{n}{min\PYZus{}value}\PY{p}{(}\PY{n+nb}{str}\PY{p}{(}\PY{n}{tmp\PYZus{}board}\PY{o}{.}\PY{n}{fen}\PY{p}{(}\PY{p}{)}\PY{p}{)}\PY{p}{,} \PY{n}{player}\PY{p}{,} \PY{n+nb}{float}\PY{p}{(}\PY{l+s+s1}{\PYZsq{}}\PY{l+s+s1}{\PYZhy{}inf}\PY{l+s+s1}{\PYZsq{}}\PY{p}{)}\PY{p}{,} \PY{n+nb}{float}\PY{p}{(}\PY{l+s+s1}{\PYZsq{}}\PY{l+s+s1}{inf}\PY{l+s+s1}{\PYZsq{}}\PY{p}{)}\PY{p}{,} \PY{n}{depth} \PY{o}{\PYZhy{}} \PY{l+m+mi}{1}\PY{p}{,} \PY{n}{end\PYZus{}time}\PY{p}{,} \PY{n}{evaluation\PYZus{}func}\PY{p}{,} \PY{n}{best\PYZus{}possible\PYZus{}result}\PY{p}{,} \PY{n}{evaluation\PYZus{}funcs\PYZus{}dict}\PY{p}{)}
                     \PY{k}{if} \PY{n}{value} \PY{o+ow}{is} \PY{k+kc}{False}\PY{p}{:}
                         \PY{n}{value} \PY{o}{=} \PY{n+nb}{float}\PY{p}{(}\PY{l+s+s1}{\PYZsq{}}\PY{l+s+s1}{\PYZhy{}inf}\PY{l+s+s1}{\PYZsq{}}\PY{p}{)}
                     \PY{n}{move\PYZus{}val\PYZus{}dict}\PY{p}{[}\PY{n}{move}\PY{p}{]} \PY{o}{=} \PY{n}{value}
                     \PY{k}{if} \PY{n}{value} \PY{o}{==} \PY{n}{MAX\PYZus{}BOARD\PYZus{}VALUE}\PY{p}{:}
                         \PY{k}{return} \PY{n}{move}
                     \PY{k}{if} \PY{n}{value} \PY{o}{\PYZgt{}} \PY{n}{best\PYZus{}value}\PY{p}{:}
                         \PY{n}{best\PYZus{}value} \PY{o}{=} \PY{n}{value}
                         \PY{n}{best\PYZus{}move} \PY{o}{=} \PY{n}{move}
         
                 \PY{n}{legal\PYZus{}moves}\PY{o}{.}\PY{n}{sort}\PY{p}{(}\PY{n}{key}\PY{o}{=}\PY{n}{move\PYZus{}val\PYZus{}dict}\PY{o}{.}\PY{n}{get}\PY{p}{,} \PY{n}{reverse}\PY{o}{=}\PY{k+kc}{True}\PY{p}{)}
                 \PY{n}{depth} \PY{o}{*}\PY{o}{=} \PY{l+m+mi}{2}
                 \PY{n}{current\PYZus{}time} \PY{o}{=} \PY{n+nb}{int}\PY{p}{(}\PY{n}{time}\PY{o}{.}\PY{n}{time}\PY{p}{(}\PY{p}{)}\PY{p}{)}
         
             \PY{k}{return} \PY{n}{best\PYZus{}move}
         
         \PY{k}{def} \PY{n+nf}{get\PYZus{}timeout\PYZus{}by\PYZus{}dif}\PY{p}{(}\PY{n}{difficulty}\PY{p}{)}\PY{p}{:}
             \PY{n}{time\PYZus{}limit} \PY{o}{=} \PY{p}{\PYZob{}}
                 \PY{l+m+mi}{1}\PY{p}{:} \PY{l+m+mi}{5}\PY{p}{,}
                 \PY{l+m+mi}{2}\PY{p}{:} \PY{l+m+mi}{10}\PY{p}{,}
                 \PY{l+m+mi}{3}\PY{p}{:} \PY{l+m+mi}{45}
             \PY{p}{\PYZcb{}}
             \PY{k}{return} \PY{n}{time\PYZus{}limit}\PY{o}{.}\PY{n}{get}\PY{p}{(}\PY{n}{difficulty}\PY{p}{)}
\end{Verbatim}

    Zunächst wird die Starttiefe auf 1 festgesetzt. Danach wird die
Startzeit auf die aktuelle Zeit gesetzt und die Endzeit berechnet, indem
auf die Startzeit das Zeitlimit addiert wird. Zudem wird der erste Wert
für die aktuelle Zeit auf die Startzeit festgelegt.

Anschließend wird für den Spieler, der aktuell am Zug ist, berechnet,
was das bestmöglich zu erreichende Resultat ist. Dies wird mit der
Funktion \texttt{get\_best\_possible\_result} durchgeführt. Dies ist
dazu gut, um finale Zustände dahingehend zu evaluieren, ob diese für den
Nutzer die bestmögliche Option ist (Sieg oder Unentschieden, wenn Sieg
nicht mehr möglich ist) und dementsprechend zu bewerten. Die Funktion
sieht wie folgt aus.

    \begin{Verbatim}[commandchars=\\\{\}]
{\color{incolor}In [{\color{incolor}28}]:} \PY{k}{def} \PY{n+nf}{get\PYZus{}best\PYZus{}possible\PYZus{}result}\PY{p}{(}\PY{n}{board}\PY{p}{,} \PY{n}{player}\PY{p}{)}\PY{p}{:}
             \PY{k}{if} \PY{n}{player} \PY{o+ow}{and} \PY{n}{board}\PY{o}{.}\PY{n}{has\PYZus{}insufficient\PYZus{}material}\PY{p}{(}\PY{n}{chess}\PY{o}{.}\PY{n}{WHITE}\PY{p}{)}\PY{p}{:}
                 \PY{k}{return} \PY{l+m+mi}{0}
             \PY{k}{if} \PY{o+ow}{not} \PY{n}{player} \PY{o+ow}{and} \PY{n}{board}\PY{o}{.}\PY{n}{has\PYZus{}insufficient\PYZus{}material}\PY{p}{(}\PY{n}{chess}\PY{o}{.}\PY{n}{BLACK}\PY{p}{)}\PY{p}{:}
                 \PY{k}{return} \PY{l+m+mi}{0}
             \PY{k}{if} \PY{n}{player} \PY{o+ow}{and} \PY{o+ow}{not} \PY{n}{board}\PY{o}{.}\PY{n}{has\PYZus{}insufficient\PYZus{}material}\PY{p}{(}\PY{n}{chess}\PY{o}{.}\PY{n}{WHITE}\PY{p}{)}\PY{p}{:}
                 \PY{k}{return} \PY{l+m+mi}{1}
             \PY{k}{if} \PY{o+ow}{not} \PY{n}{player} \PY{o+ow}{and} \PY{o+ow}{not} \PY{n}{board}\PY{o}{.}\PY{n}{has\PYZus{}insufficient\PYZus{}material}\PY{p}{(}\PY{n}{chess}\PY{o}{.}\PY{n}{BLACK}\PY{p}{)}\PY{p}{:}
                 \PY{k}{return} \PY{o}{\PYZhy{}}\PY{l+m+mi}{1}
\end{Verbatim}

    Dabei muss der Funktion der aktuelle Zustand sowie der Spieler, der an
der Reihe ist, mitgegeben werden. Ist der aktuelle Spieler der der
weißen Figuren (player == True) und hat weiß unzureichende Materialien
für einen Sieg, so ist der bestmögliche Zustand ein Remis. Das gleiche
Ergebnis wird zurückgegeben, wenn der Spieler der der schwarzen Figuren
ist (player == False) und schwarz unzureichende Materialien hat.

Ist der Spieler jedoch weiß und er hat noch ausreichend Materialien, so
wird der Wert 1 zurückgegeben, da ein Sieg noch erreichbar ist. Genauso
wird für den schwarzen Spieler der Wert -1 zurückgegeben, falls er noch
ausreichende Materialien besitzt, da dieser noch einen Sieg erreichen
kann und der Wert -1 für einen Sieg von Schwarz steht.

Nach dieser Abfrage wird der eigentliche Algorithmus des ``Iterative
Deepening` durchgeführt.

    Dabei wird zunächst eine Liste aller legalen Züge erstellt. Dann wird
eine Schleife so lange durchlaufen, bis entweder die Zeit abgelaufen ist
oder aber die maximale Tiefe erreicht ist.

In dieser Schleife wird ein Dictionary aller Züge mit ihren berechneten
Werte erstellt. Zudem werden Anfangswerte für die besten Züge und dessen
Wert festgelegt. Der Anfangswert des besten Zugs wird auf den ersten Zug
festgesetzt. Der Wert dieses wird auf den Wert ``- Unendlich`` gesetzt.

Nun wird über alle legalen Züge iteriert und für jeden Zug ein
temporäres "board'' angelegt, das den erreichbaren Zustand widerspiegelt. Nun
wird mittels des Minimax-Algorithmus der Wert dieses boards ermittelt.
Dabei wird als Alpha ``- Unendlich`` und als Beta ``Unendlich`` mitgegeben.
Was die Werte Alpha und Beta aussagen, wird im Kapitel \ref{implementierung-des-minimax-algorithmus-mit-alpha-beta-pruning}
beschrieben. Zudem wird die Tiefe auf einen Wert festgelegt, der um
einen Wert geringer ist als die maximale Tiefe, da durch Aufruf der
Funktion in die erste Tiefe hineingegangen wurde. Zudem wird die Zeit
mitgegeben, zu der der Algorithmus enden muss, damit der Minimax
Algorithmus dementsprechend endet und das Zeitlimit nicht überschreitet.

Nachdem der Minimax-Algorithmus fertig durchlaufen ist, wird der Wert
mit dem Zug zu dem Dictionary hinzugefügt. Dazu wird zuvor jedoch noch überprüft, ob der Wert 'False' zurückgegeben wurde. Ist dies der Fall, ist das Zeitlimit abgelaufen und der Zug somit ungültig. In dem Fall wird dem Zug der maximal schlechteste Wert zugewiesen. 

Gleicht der Wert ansonsten dem maximalen Wert für einen Zustand, ist also dementsprechend ein Sieg,
wird der Zug direkt zurückgegeben, da mit diesem dann auf jeden Fall ein
Sieg erreicht werden kann. Andernfalls wird der Wert verglichen, ob er
besser ist als der aktuelle Wert. Ist dies der Fall, so wird der neue
beste Wert auf den aktuell berechneten festgelegt, ebenso wie der beste
Zug auf den der aktuellen Iteration gesetzt wird.

Nachdem alle Züge durchlaufen wurden, wird die Liste aller legalen Züge
anhand der berechneten Werte sortiert, damit im nächsten Durchlauf die
Züge in dieser Reihenfolge durchlaufen werden. Dies verbessert den
Durchsatz beim Alpha Beta Pruning, wie in Kapitel \ref{alpha_beta} beschrieben und
garantiert zudem, dass der beste Wert der vergangenen Runde gewählt
wird, falls der Minimax-Algorithmus beim Durchlaufen der nächst tieferen
Tiefe das Zeitlimit erreicht, bevor die Runde komplett evaluiert werden
konnte.

Abschließend wird noch die Tiefe um 1 erhöht und die aktuelle Zeit auf
die Systemzeit gesetzt, damit anhand dieser entschieden werden kann, ob
der Algorithmus noch weiter durchlaufen darf.

Nachdem dann die Zeit abgelaufen ist und alle Züge in der für die
angegebenen Zeit maximalen Tiefe evaluiert wurden, wird der bestmögliche Zug zurückgegeben. Dieser wird dann von der KI ausgeführt.

    \section{Implementierung des Minimax-Algorithmus mit
Alpha-Beta-Pruning}\label{implementierung-des-minimax-algorithmus-mit-alpha-beta-pruning}

Um den bestmöglichen Zug zu erkennen, wird beim Minimax Algorithmus
jeder Zug bis zu einer gewissen Tiefe betrachtet. Dabei wird unter der
Prämisse gehandelt, dass auch der Gegner stets den besten Zug macht.
Dies führt dazu, dass der bestmöglichste Zug ausgewählt wird (max), der
erreichbar ist, wenn der Gegner mit dem für ihn jeweils besten Zug
antwortet, der für den Spieler somit der schlechteste ist (min).

Die Umsetzung dabei erfolgt in zwei Funktionen \texttt{min\_value} und
\texttt{max\_value}. Auch diese sind direkt in der Datei
\texttt{player/ai.py} vorzufinden. Erstere berechnet dabei den schlechtmöglichsten Ausgang für den Spieler aus einer bestimmten Position, also
den besten Ausgang für den Gegner. Letztere berechnet den bestmöglichen Ausgang. Beide Funktionen sind sehr ähnlich aufgebaut und in
folgendem Code-Snippet zu sehen.

    \begin{Verbatim}[commandchars=\\\{\}]
{\color{incolor}In [{\color{incolor}29}]:} \PY{k}{def} \PY{n+nf}{min\PYZus{}value}\PY{p}{(}\PY{n}{board\PYZus{}fen}\PY{p}{,} \PY{n}{player}\PY{p}{,} \PY{n}{alpha}\PY{p}{,} \PY{n}{beta}\PY{p}{,} \PY{n}{depth}\PY{p}{,} \PY{n}{time\PYZus{}limit}\PY{p}{,} \PY{n}{evaluation\PYZus{}func}\PY{p}{,} \PY{n}{best\PYZus{}possible\PYZus{}result}\PY{p}{,} \PY{n}{evaluation\PYZus{}funcs\PYZus{}dict}\PY{p}{)}\PY{p}{:}
             \PY{n}{board} \PY{o}{=} \PY{n}{chess}\PY{o}{.}\PY{n}{Board}\PY{p}{(}\PY{n}{board\PYZus{}fen}\PY{p}{)}
             \PY{n}{v} \PY{o}{=} \PY{n+nb}{float}\PY{p}{(}\PY{l+s+s1}{\PYZsq{}}\PY{l+s+s1}{inf}\PY{l+s+s1}{\PYZsq{}}\PY{p}{)}
         
             \PY{k}{if} \PY{n}{board}\PY{o}{.}\PY{n}{is\PYZus{}game\PYZus{}over}\PY{p}{(}\PY{p}{)} \PY{o+ow}{or} \PY{n}{depth} \PY{o}{==} \PY{l+m+mi}{0}\PY{p}{:}
                 \PY{k}{return} \PY{n}{evaluation\PYZus{}func}\PY{p}{(}\PY{n}{board}\PY{p}{,} \PY{n}{player}\PY{p}{,} \PY{n}{best\PYZus{}possible\PYZus{}result}\PY{p}{,} \PY{n}{evaluation\PYZus{}funcs\PYZus{}dict}\PY{p}{)}
             \PY{k}{if} \PY{n+nb}{int}\PY{p}{(}\PY{n}{time}\PY{o}{.}\PY{n}{time}\PY{p}{(}\PY{p}{)}\PY{p}{)} \PY{o}{\PYZgt{}}\PY{o}{=} \PY{n}{time\PYZus{}limit}\PY{p}{:}
                 \PY{k}{return} \PY{k+kc}{False}
         
             \PY{k}{for} \PY{n}{move} \PY{o+ow}{in} \PY{n}{board}\PY{o}{.}\PY{n}{legal\PYZus{}moves}\PY{p}{:}
                 \PY{n}{tmp\PYZus{}board} \PY{o}{=} \PY{n}{chess}\PY{o}{.}\PY{n}{Board}\PY{p}{(}\PY{n}{board\PYZus{}fen}\PY{p}{)}
                 \PY{n}{tmp\PYZus{}board}\PY{o}{.}\PY{n}{push}\PY{p}{(}\PY{n}{move}\PY{p}{)}
         
                 \PY{n}{deeper\PYZus{}val} \PY{o}{=} \PY{n}{max\PYZus{}value}\PY{p}{(}\PY{n+nb}{str}\PY{p}{(}\PY{n}{tmp\PYZus{}board}\PY{o}{.}\PY{n}{fen}\PY{p}{(}\PY{p}{)}\PY{p}{)}\PY{p}{,} \PY{n}{player}\PY{p}{,} \PY{n}{alpha}\PY{p}{,} \PY{n}{beta}\PY{p}{,} \PY{n}{depth} \PY{o}{\PYZhy{}}\PY{l+m+mi}{1}\PY{p}{,} \PY{n}{time\PYZus{}limit}\PY{p}{,} \PY{n}{evaluation\PYZus{}func}\PY{p}{,} \PY{n}{best\PYZus{}possible\PYZus{}result}\PY{p}{,} \PY{n}{evaluation\PYZus{}funcs\PYZus{}dict}\PY{p}{)}
                 \PY{k}{if} \PY{n}{deeper\PYZus{}val} \PY{o+ow}{is} \PY{k+kc}{False}\PY{p}{:}
                     \PY{k}{return} \PY{k+kc}{False}            
                 \PY{n}{v} \PY{o}{=} \PY{n+nb}{min}\PY{p}{(}\PY{n}{v}\PY{p}{,} \PY{n}{deeper\PYZus{}val}\PY{p}{)}  
         
                 \PY{k}{if} \PY{n}{v} \PY{o}{\PYZlt{}}\PY{o}{=} \PY{n}{alpha}\PY{p}{:}
                     \PY{k}{return} \PY{n}{v}
                 \PY{n}{beta} \PY{o}{=} \PY{n+nb}{min}\PY{p}{(}\PY{n}{beta}\PY{p}{,} \PY{n}{v}\PY{p}{)}
             \PY{k}{return} \PY{n}{v}
\end{Verbatim}

    Neben des aktuellen Zustands des Spiels in FEN-Notation und dem Spieler,
für den es den Zustand zu evaluieren gilt, wird außerdem eine Tiefe
sowie ein Zeitlimit mitgegeben sowie die Werte Alpha und Beta. Alpha und
Beta sind dabei, wie in Kapitel \ref{alpha_beta} erklärt, dazu da, um den
Minimax-Algorithmus zu beschleunigen, indem nicht jeder mögliche Zustand
betrachtet wird. Durch diese Werte fallen nämlich solche weg, die direkt
als irrelevant betrachtet werden können, da ohnehin bereits ein
besserrer (im Fall von \texttt{min\_value}) bzw. schlechterer (im Fall
von \texttt{max\_value}) Wert gefunden wurde.

Nachdem das Schachbrett über die FEN-Notation erstellt wurde, wird
überprüft, ob das Spiel bereits vorbei ist oder die mitgegebene Tiefe
auf 0 liegt. In beiden Fällen wird der aktuelle Zustand direkt evaluiert
und zurückgegeben. Ansonsten wird geprüft, ob das übergebene Zeitlimit
bereits erreicht wurde. In dem Fall wird der Wert 'False' 
zurückgegeben, damit dieser Zug später automatisch den maximal negativen Wert zugewiesen bekommen kann (siehe Iterative Deepening).

Ist auch dies nicht der Fall, werden alle von dem gegebenen Zustand aus
erreichbaren Zustände durchgegangen. Dazu wird zunächst ein temporäres
Schachbrett erstellt und der Zug auf diesem ausgeführt.
Dies ist dann der neue, erreichbare Zustand. Dieser wird dann in die
nächsttieferen Iteration gegeben, bei der nun der maximale Wert gesucht
wird. Dabei wird auch der Spieler mitgegeben sowie die Werte Alpha und
Beta und das Zeitlimit ebenso wie die um eins reduzierte Tiefe. Ist
dabei ein Wert dabei, der kleiner ist als das aktuelle v, wird dieser
Wert als das neue v genommen. Andernfalls bleibt v beim aktuellen Wert.

Anschließend wird überprüft, ob der Wert v kleiner ist als der aktuelle
Alpha Wert. Ist dies der Fall, muss der Pfad keine weitere Beachtung
finden und es kann direkt v zurückgegeben werden. ist dies nicht der
Fall, so wird Beta auf v gesetzt, falls dieser Wert kleiner als das
aktuelle Beta ist. Nach Durchgang aller legalen Züge wird dann der sich
aus allen diesen Iterationen ergebende Wert v zurückgegeben.

Die \texttt{max\_value} Funktion läuft gleich ab, mit dem einzigen
Unterschied, dass hier der maximale statt der minmale Wert gesucht wird
und dementsprechend die Vergleiche sowie Startwerte angepasst sind.
Außerdem gibt dieser bei der um eins tieferen Iteration in die
\texttt{min\_value} Funktion, an der Stelle, an der diese in die
\texttt{max\_value} Funktion gibt. Der restliche Aufbau bleibt
unverändert.

    \begin{Verbatim}[commandchars=\\\{\}]
{\color{incolor}In [{\color{incolor}30}]:} \PY{k}{def} \PY{n+nf}{max\PYZus{}value}\PY{p}{(}\PY{n}{board\PYZus{}fen}\PY{p}{,} \PY{n}{player}\PY{p}{,} \PY{n}{alpha}\PY{p}{,} \PY{n}{beta}\PY{p}{,} \PY{n}{depth}\PY{p}{,} \PY{n}{time\PYZus{}limit}\PY{p}{,} \PY{n}{evaluation\PYZus{}func}\PY{p}{,} \PY{n}{best\PYZus{}possible\PYZus{}result}\PY{p}{,} \PY{n}{evaluation\PYZus{}funcs\PYZus{}dict}\PY{p}{)}\PY{p}{:}
             \PY{n}{board} \PY{o}{=} \PY{n}{chess}\PY{o}{.}\PY{n}{Board}\PY{p}{(}\PY{n}{board\PYZus{}fen}\PY{p}{)}
             \PY{n}{v} \PY{o}{=} \PY{n+nb}{float}\PY{p}{(}\PY{l+s+s1}{\PYZsq{}}\PY{l+s+s1}{\PYZhy{}inf}\PY{l+s+s1}{\PYZsq{}}\PY{p}{)}
         
             \PY{k}{if} \PY{n}{board}\PY{o}{.}\PY{n}{is\PYZus{}game\PYZus{}over}\PY{p}{(}\PY{p}{)} \PY{o+ow}{or} \PY{n}{depth} \PY{o}{==} \PY{l+m+mi}{0}\PY{p}{:}
                 \PY{k}{return} \PY{n}{evaluation\PYZus{}func}\PY{p}{(}\PY{n}{board}\PY{p}{,} \PY{n}{player}\PY{p}{,} \PY{n}{best\PYZus{}possible\PYZus{}result}\PY{p}{,} \PY{n}{evaluation\PYZus{}funcs\PYZus{}dict}\PY{p}{)}
             \PY{k}{if} \PY{n+nb}{int}\PY{p}{(}\PY{n}{time}\PY{o}{.}\PY{n}{time}\PY{p}{(}\PY{p}{)}\PY{p}{)} \PY{o}{\PYZgt{}}\PY{o}{=} \PY{n}{time\PYZus{}limit}\PY{p}{:}
                 \PY{k}{return} \PY{k+kc}{False}
         
             \PY{k}{for} \PY{n}{move} \PY{o+ow}{in} \PY{n}{board}\PY{o}{.}\PY{n}{legal\PYZus{}moves}\PY{p}{:}
                 \PY{n}{tmp\PYZus{}board} \PY{o}{=} \PY{n}{chess}\PY{o}{.}\PY{n}{Board}\PY{p}{(}\PY{n}{board\PYZus{}fen}\PY{p}{)}
                 \PY{n}{tmp\PYZus{}board}\PY{o}{.}\PY{n}{push}\PY{p}{(}\PY{n}{move}\PY{p}{)}
         
                 \PY{n}{deeper\PYZus{}val} \PY{o}{=} \PY{n}{min\PYZus{}value}\PY{p}{(}\PY{n+nb}{str}\PY{p}{(}\PY{n}{tmp\PYZus{}board}\PY{o}{.}\PY{n}{fen}\PY{p}{(}\PY{p}{)}\PY{p}{)}\PY{p}{,} \PY{n}{player}\PY{p}{,} \PY{n}{alpha}\PY{p}{,} \PY{n}{beta}\PY{p}{,} \PY{n}{depth} \PY{o}{\PYZhy{}}\PY{l+m+mi}{1}\PY{p}{,} \PY{n}{time\PYZus{}limit}\PY{p}{,} \PY{n}{evaluation\PYZus{}func}\PY{p}{,} \PY{n}{best\PYZus{}possible\PYZus{}result}\PY{p}{,} \PY{n}{evaluation\PYZus{}funcs\PYZus{}dict}\PY{p}{)}
                 \PY{k}{if} \PY{n}{deeper\PYZus{}val} \PY{o+ow}{is} \PY{k+kc}{False}\PY{p}{:}
                     \PY{k}{return} \PY{k+kc}{False}
                 \PY{n}{v} \PY{o}{=} \PY{n+nb}{max}\PY{p}{(}\PY{n}{v}\PY{p}{,} \PY{n}{deeper\PYZus{}val}\PY{p}{)}
         
                 \PY{k}{if} \PY{n}{v} \PY{o}{\PYZgt{}}\PY{o}{=} \PY{n}{beta}\PY{p}{:}
                     \PY{k}{return} \PY{n}{v}
                 \PY{n}{alpha} \PY{o}{=} \PY{n+nb}{max}\PY{p}{(}\PY{n}{alpha}\PY{p}{,} \PY{n}{v}\PY{p}{)}
             \PY{k}{return} \PY{n}{v}
\end{Verbatim}

    Mit diesem Algorithmus wird ein Baum aus allen Pfaden erstellt, anhand
der der beste Zug ermittelt werden kann. Da es sich jedoch nur selten um Endzustände handelt, für
die eine Bewertung trivial erfolgen kann, ist eine Evaluierung der
Zustände nötig. Diese wird im nächsten Kapitel beschrieben.

    \section{Evaluierung eines gegebenen
Schachbretts}\label{evaluierung-eines-gegebenen-schachbretts}

    Um gegebene Zustände auch mitten im Spiel bewerten zu können, müssen
diese anhand bestimmter Kriterien bewertet werden können, die über Sieg
oder Niederlage hinausgehen. Dazu gibt es verschiedene Ansätze, wie in
Kapitel \ref{evaluation_func} erläutert. Einige davon wurden im Laufe des Projektes
umgesetzt und implementiert. Diese werden in diesem Kapitel erläutert.
Zunächst jedoch gilt es aufzuzeigen, wie die Evaluierung der Zustände im
generellen umgesetzt wird.

Zunächst wird zum Evaluieren eines Zustandes jener Zustand sowie der
Spieler, für den dieser zu evaluieren ist, in die Funktion
\texttt{evaluate\_board} gegeben, die die Evaluierung zentral verwaltet.
Diese befindet sich in der \texttt{player/ai.py}.

    \begin{Verbatim}[commandchars=\\\{\}]
{\color{incolor}In [{\color{incolor}31}]:} \PY{k}{def} \PY{n+nf}{evaluate\PYZus{}board}\PY{p}{(}\PY{n}{board}\PY{p}{,} \PY{n}{player}\PY{p}{,} \PY{n}{best\PYZus{}possible\PYZus{}result}\PY{p}{,} \PY{n}{evaluation\PYZus{}funcs\PYZus{}dict}\PY{p}{)}\PY{p}{:}
             \PY{n}{player\PYZus{}color} \PY{o}{=} \PY{n}{chess}\PY{o}{.}\PY{n}{WHITE} \PY{k}{if} \PY{n}{player} \PY{k}{else} \PY{n}{chess}\PY{o}{.}\PY{n}{BLACK}
         
             \PY{k}{if} \PY{n}{board}\PY{o}{.}\PY{n}{is\PYZus{}game\PYZus{}over}\PY{p}{(}\PY{p}{)}\PY{p}{:}
                 \PY{n}{result} \PY{o}{=} \PY{n}{Tools}\PY{o}{.}\PY{n}{get\PYZus{}board\PYZus{}result}\PY{p}{(}\PY{n}{board}\PY{p}{)}
                 \PY{k}{if} \PY{n}{result} \PY{o+ow}{is} \PY{n}{best\PYZus{}possible\PYZus{}result}\PY{p}{:}
                     \PY{k}{return} \PY{n}{MAX\PYZus{}BOARD\PYZus{}VALUE}
                 \PY{k}{if} \PY{n}{result} \PY{o+ow}{is} \PY{n}{best\PYZus{}possible\PYZus{}result} \PY{o}{*} \PY{o}{\PYZhy{}}\PY{l+m+mi}{1}\PY{p}{:}
                     \PY{k}{return} \PY{o}{\PYZhy{}}\PY{l+m+mi}{1} \PY{o}{*} \PY{n}{MAX\PYZus{}BOARD\PYZus{}VALUE}
         
             \PY{n}{evaluation\PYZus{}val} \PY{o}{=} \PY{l+m+mi}{0}
             \PY{k}{for} \PY{n}{func}\PY{p}{,} \PY{n}{value} \PY{o+ow}{in} \PY{n}{evaluation\PYZus{}funcs\PYZus{}dict}\PY{o}{.}\PY{n}{items}\PY{p}{(}\PY{p}{)}\PY{p}{:}
                 \PY{k}{if} \PY{n}{value} \PY{o}{\PYZgt{}} \PY{l+m+mi}{0}\PY{p}{:}
                     \PY{n}{evaluation\PYZus{}val} \PY{o}{=} \PY{n}{evaluation\PYZus{}val} \PY{o}{+} \PY{n}{value} \PY{o}{*} \PY{n}{func}\PY{p}{(}\PY{n}{board}\PY{p}{,} \PY{n}{player\PYZus{}color}\PY{p}{)}
             \PY{k}{return} \PY{n}{evaluation\PYZus{}val}
\end{Verbatim}

    Dabei wird zunächst anhand des Spielers die Farbe dieses ermittelt, die
später bei den einzelnen Evaluierungsfunktionen benötigt wird.

Dann wird für den Fall, dass der gegebene Zustand einem Endzustand
gleicht, das Ergebnis dieses ermittelt. Gleicht dieses dem bestmöglichen
Ergebnis, dass mittels der \texttt{get\_board\_result} Funktion zuvor
ermittelt wurde, so wird der maximale Wert (Unendlich)
für den Zustand zurückgegeben. Ist das Ergebnis des übermittelten
Zustands jedoch eine Niederlage, wird der maximale Wert umgekehrt und
zurückgegeben (minus Unendlich).

Gleicht der Zustand keinem Endzustand, so werden bestimmte
Evaluationsfunktionen durchlaufen und aufaddiert. Dazu startet der Wert
bei 0 und für jede Evaluierungsfunktion wird das Ergebnis dieser
multipliziert mit einem festgelegten Faktor zu dem Gesamtwert
aufaddiert. Die Faktoren sowie die durchzuführenden
Evaluierungsfunktionen sind dabei abhängig vom Schwierigkeitsgrad der KI
sowie vom Spielstatus (Eröffnung, Mittelspiel, Endspiel).

Dabei werden zur Performanz-Steigerung jedoch nur Evaluierungsfunktionen
durchgegangen, deren Faktor höher als 0 liegt. Der Grund dafür ist,
dass bestimmte Funktionen je nach Spielstatus leichter aus der
Evaluierung herausgenommen werden können, ohne, dass das gesamte
Dictionary angepasst werden muss und zudem keine unnötige Rechenzeit
durch Berechnung eines Werts benötigt wird, der im Endeffekt ohnehin
nicht zum Evaluierungswert aufaddiert wird.

Der daraus entstehende Evaluierungswert, der zurückgegeben wird, gibt
einen guten Aufschluss über den Wert des aktuellen Zustands. Dazu werden
verschiedene Evaluierungsfunktionen verwendet, wie in folgendem
Abschnitt zu sehen ist. Diese befinden sich in der Datei
\texttt{misc/ai\_evaluation\_lib.py}.

    \subsection{Materialbewertung}\label{materialbewertung}

Eine zentrale sowie einfache Bewertung ist dabei die Bewertung der
vorhandenen Materialien auf dem Spielfeld. Dabei werden alle Figuren der
jeweiligen Spieler zusammengezählt und je nach Figur mit einem Wert
multipliziert.

Hierbei ist zunächst jedem Figurentyp ein Wert zuzuweisen. Üblicherweise
werden Bauern dabei 1 Punkt, Türmen 5 Punkte, Springern sowie Läufern
jeweils 3 Punkte und der Dame 9 Punkte zugeordnet. Dies geschieht über
die Funktion \texttt{assign\_piece\_value()}. Dabei wird der Typ
angegeben und die Punkte zurückgegeben. Zusätzlich kann angegeben
werden, ob auch der König einen Wert zugewiesen bekommen soll. Diese
werden dann über die \texttt{map} Funktion nach Farbe in der
\texttt{get\_value\_by\_color} Funktion für alle auf den Feldern
befindlichen Figuren zusammengerechnet.

\begin{Verbatim}[commandchars=\\\{\}]
{\color{incolor}In [{\color{incolor}32}]:} \PY{n}{PAWN\PYZus{}VALUE} \PY{o}{=} \PY{l+m+mi}{1}
         \PY{n}{ROOK\PYZus{}VALUE} \PY{o}{=} \PY{l+m+mi}{5}
         \PY{n}{KNIGHT\PYZus{}VALUE} \PY{o}{=} \PY{l+m+mi}{3}
         \PY{n}{BISHOP\PYZus{}VALUE} \PY{o}{=} \PY{l+m+mi}{3}
         \PY{n}{QUEEN\PYZus{}VALUE} \PY{o}{=} \PY{l+m+mi}{9}
         \PY{n}{KING\PYZus{}VALUE} \PY{o}{=} \PY{l+m+mi}{15}
         
         \PY{k}{def} \PY{n+nf}{assign\PYZus{}piece\PYZus{}value}\PY{p}{(}\PY{n}{piece\PYZus{}type}\PY{p}{,} \PY{n}{count\PYZus{}king}\PY{o}{=}\PY{k+kc}{True}\PY{p}{)}\PY{p}{:}
             \PY{k}{return} \PY{p}{\PYZob{}}
                 \PY{l+m+mi}{1}\PY{p}{:} \PY{n}{PAWN\PYZus{}VALUE}\PY{p}{,}
                 \PY{l+m+mi}{2}\PY{p}{:} \PY{n}{KNIGHT\PYZus{}VALUE}\PY{p}{,}
                 \PY{l+m+mi}{3}\PY{p}{:} \PY{n}{BISHOP\PYZus{}VALUE}\PY{p}{,}
                 \PY{l+m+mi}{4}\PY{p}{:} \PY{n}{ROOK\PYZus{}VALUE}\PY{p}{,}
                 \PY{l+m+mi}{5}\PY{p}{:} \PY{n}{QUEEN\PYZus{}VALUE}\PY{p}{,}
                 \PY{l+m+mi}{6}\PY{p}{:} \PY{n}{KING\PYZus{}VALUE} \PY{k}{if} \PY{n}{count\PYZus{}king} \PY{k}{else} \PY{l+m+mi}{0}
             \PY{p}{\PYZcb{}}\PY{o}{.}\PY{n}{get}\PY{p}{(}\PY{n}{piece\PYZus{}type}\PY{p}{,} \PY{l+m+mi}{0}\PY{p}{)}
         
         
         \PY{k}{def} \PY{n+nf}{get\PYZus{}value\PYZus{}by\PYZus{}color}\PY{p}{(}\PY{n}{board}\PY{p}{,} \PY{n}{color}\PY{p}{,} \PY{n}{count\PYZus{}king}\PY{o}{=}\PY{k+kc}{True}\PY{p}{)}\PY{p}{:}
             \PY{n}{pieces\PYZus{}value} \PY{o}{=} \PY{n+nb}{map}\PY{p}{(}
                 \PY{k}{lambda} \PY{n}{piece\PYZus{}type}\PY{p}{:} \PY{n+nb}{len}\PY{p}{(}\PY{n}{board}\PY{o}{.}\PY{n}{pieces}\PY{p}{(}\PY{n}{piece\PYZus{}type}\PY{p}{,} \PY{n}{color}\PY{p}{)}\PY{p}{)} \PY{o}{*} \PY{n}{assign\PYZus{}piece\PYZus{}value}\PY{p}{(}\PY{n}{piece\PYZus{}type}\PY{p}{,} \PY{n}{count\PYZus{}king}\PY{p}{)}\PY{p}{,} \PY{n}{chess}\PY{o}{.}\PY{n}{PIECE\PYZus{}TYPES}\PY{p}{)}
             \PY{k}{return} \PY{n+nb}{sum}\PY{p}{(}\PY{n}{pieces\PYZus{}value}\PY{p}{)}
         
         
         \PY{k}{def} \PY{n+nf}{get\PYZus{}board\PYZus{}value}\PY{p}{(}\PY{n}{board}\PY{p}{,} \PY{n}{color}\PY{p}{,} \PY{n}{count\PYZus{}king}\PY{o}{=}\PY{k+kc}{True}\PY{p}{)}\PY{p}{:}
             \PY{n}{white\PYZus{}value} \PY{o}{=} \PY{n}{get\PYZus{}value\PYZus{}by\PYZus{}color}\PY{p}{(}\PY{n}{board}\PY{p}{,} \PY{n}{chess}\PY{o}{.}\PY{n}{WHITE}\PY{p}{,} \PY{n}{count\PYZus{}king}\PY{p}{)}
             \PY{n}{black\PYZus{}value} \PY{o}{=} \PY{n}{get\PYZus{}value\PYZus{}by\PYZus{}color}\PY{p}{(}\PY{n}{board}\PY{p}{,} \PY{n}{chess}\PY{o}{.}\PY{n}{BLACK}\PY{p}{,} \PY{n}{count\PYZus{}king}\PY{p}{)}
         
             \PY{k}{return} \PY{n}{white\PYZus{}value} \PY{o}{\PYZhy{}} \PY{n}{black\PYZus{}value} \PY{k}{if} \PY{n}{color} \PY{o+ow}{is} \PY{n}{chess}\PY{o}{.}\PY{n}{WHITE} \PY{k}{else} \PY{n}{black\PYZus{}value} \PY{o}{\PYZhy{}} \PY{n}{white\PYZus{}value}
\end{Verbatim}

    Um den Gesamtwert des Schachbretts zu berechnen muss zunächst der Wert
aller weißer Figuren berechnet werden und von diesem der Wert aller
schwarzen Figuren abgezogen werden. Je nach angegebenem Spieler wird für
diesen ein positiver Wert zurückgegeben, wenn das Spiel zu dessen
Gunsten verläuft und ein negativer, wenn dies nicht der Fall ist.

Damit die Werte der Spieler berechnet werden können, wird die Anzahl
aller Figurentypen der jeweiligen Farbe berechnet und diese mit dem Wert
der Figurentypen multipliziert. Am Ende werden die Ergebnisse für alle
Figurentypen zusammengezählt und zurückgegeben.

    \subsection{Materialbewertung attackierter
Figuren}\label{materialbewertung-attackierter-figuren}

Die Berechnung der attackierten Figuren ist ähnlich zu dem Vorgehen bei
der Berechnung des Brettwerts. Dabei werden erst die Werte, der vom
weißen Spieler attackierten Figuren berechnet und davon die Werte der
vom schwarzen Spieler attackierten Figuren abgezogen. Auch hierbei ist
ein positives Ergebnis zum Vorteil des angegebenen Spielers und ein
negativer Wert zum Vorteil des Gegenübers. Ebenso gilt umso höher der
Wert, desto deutlicher der Vorteil.

    \begin{Verbatim}[commandchars=\\\{\}]
{\color{incolor}In [{\color{incolor}33}]:} \PY{k}{def} \PY{n+nf}{get\PYZus{}attacked\PYZus{}pieces\PYZus{}value\PYZus{}by\PYZus{}color}\PY{p}{(}\PY{n}{board}\PY{p}{,} \PY{n}{attacker\PYZus{}color}\PY{p}{,} \PY{n}{defender\PYZus{}color}\PY{p}{)}\PY{p}{:}
             \PY{c+c1}{\PYZsh{} filters squares for attacked squares, on which a figure of defender is placed}
             \PY{n}{attacked\PYZus{}squares} \PY{o}{=} \PY{n+nb}{filter}\PY{p}{(}\PY{k}{lambda} \PY{n}{square}\PY{p}{:} \PY{n}{board}\PY{o}{.}\PY{n}{is\PYZus{}attacked\PYZus{}by}\PY{p}{(}\PY{n}{attacker\PYZus{}color}\PY{p}{,} \PY{n}{square}\PY{p}{)} \PY{o+ow}{and} \PY{o+ow}{not} \PY{n}{board}\PY{o}{.}\PY{n}{piece\PYZus{}at}\PY{p}{(}
                 \PY{n}{square}\PY{p}{)} \PY{o+ow}{is} \PY{k+kc}{None} \PY{o+ow}{and} \PY{n}{board}\PY{o}{.}\PY{n}{piece\PYZus{}at}\PY{p}{(}\PY{n}{square}\PY{p}{)}\PY{o}{.}\PY{n}{color} \PY{o+ow}{is} \PY{n}{defender\PYZus{}color}\PY{p}{,} \PY{n}{chess}\PY{o}{.}\PY{n}{SQUARES}\PY{p}{)}
             \PY{c+c1}{\PYZsh{} maps piece type to attacked figure}
             \PY{n}{attacked\PYZus{}pieces} \PY{o}{=} \PY{n+nb}{map}\PY{p}{(}\PY{k}{lambda} \PY{n}{square}\PY{p}{:} \PY{n}{board}\PY{o}{.}\PY{n}{piece\PYZus{}at}\PY{p}{(}\PY{n}{square}\PY{p}{)}\PY{o}{.}\PY{n}{piece\PYZus{}type}\PY{p}{,} \PY{n}{attacked\PYZus{}squares}\PY{p}{)}
             \PY{c+c1}{\PYZsh{} maps piece value to attacked pieces}
             \PY{n}{value} \PY{o}{=} \PY{n+nb}{map}\PY{p}{(}\PY{n}{assign\PYZus{}piece\PYZus{}value}\PY{p}{,} \PY{n}{attacked\PYZus{}pieces}\PY{p}{)}
             \PY{c+c1}{\PYZsh{} sums piece value of all attacked pieces}
             \PY{k}{return} \PY{n+nb}{sum}\PY{p}{(}\PY{n}{value}\PY{p}{)}
         
         
         \PY{k}{def} \PY{n+nf}{get\PYZus{}attacked\PYZus{}pieces\PYZus{}value}\PY{p}{(}\PY{n}{board}\PY{p}{,} \PY{n}{color}\PY{p}{)}\PY{p}{:}
             \PY{c+c1}{\PYZsh{} factor for weakening value if opponet turn comes first}
             \PY{n}{white\PYZus{}factor} \PY{o}{=} \PY{l+m+mi}{1} \PY{k}{if} \PY{n+nb}{bool}\PY{p}{(}\PY{n}{board}\PY{o}{.}\PY{n}{turn}\PY{p}{)} \PY{k}{else} \PY{l+m+mi}{1}\PY{o}{/}\PY{l+m+mi}{2}
             \PY{n}{black\PYZus{}factor} \PY{o}{=} \PY{l+m+mi}{1}\PY{o}{/}\PY{l+m+mi}{2} \PY{k}{if} \PY{n+nb}{bool}\PY{p}{(}\PY{n}{board}\PY{o}{.}\PY{n}{turn}\PY{p}{)} \PY{k}{else} \PY{l+m+mi}{1}
             
             \PY{n}{white\PYZus{}value} \PY{o}{=} \PY{n}{white\PYZus{}factor} \PY{o}{*} \PY{n}{get\PYZus{}attacked\PYZus{}pieces\PYZus{}value\PYZus{}by\PYZus{}color}\PY{p}{(}\PY{n}{board}\PY{p}{,} \PY{n}{chess}\PY{o}{.}\PY{n}{WHITE}\PY{p}{,} \PY{n}{chess}\PY{o}{.}\PY{n}{BLACK}\PY{p}{)}
             \PY{n}{black\PYZus{}value} \PY{o}{=} \PY{n}{black\PYZus{}factor} \PY{o}{*} \PY{n}{get\PYZus{}attacked\PYZus{}pieces\PYZus{}value\PYZus{}by\PYZus{}color}\PY{p}{(}\PY{n}{board}\PY{p}{,} \PY{n}{chess}\PY{o}{.}\PY{n}{BLACK}\PY{p}{,} \PY{n}{chess}\PY{o}{.}\PY{n}{WHITE}\PY{p}{)}
         
             \PY{k}{return} \PY{n}{white\PYZus{}value} \PY{o}{\PYZhy{}} \PY{n}{black\PYZus{}value} \PY{k}{if} \PY{n}{color} \PY{o+ow}{is} \PY{n}{chess}\PY{o}{.}\PY{n}{WHITE} \PY{k}{else} \PY{n}{black\PYZus{}value} \PY{o}{\PYZhy{}} \PY{n}{white\PYZus{}value}
\end{Verbatim}

    Um diese Werte der attackierten Figuren zu berechnen wird jedes Feld
durchgegangen. Daraus werden die Felder gefiltert, die von einer Figur
der Farbe des Verteidigers belegt sind und von einer Figur der
angreifenden Farbe attackiert werden können. Anschließend wird zu diesen
Feldern der Typ der Figur zugeordnet, die sich auf dem Feld befindet.
Daraufhin werden diesen ihre jeweiligen Werte zugeordnet und diese
abschließend aufsummiert.

    \subsection{Positionsbewertung}\label{positionsbewertung}

Um die Positionen der einzelnen Figuren zu bewerten, werden zunächst für
jeden Figurentypen Matrizen benötigt, die über jedes Feld eine Aussage
über den Wert der Position der Figur geben. Diese sind in Kapitel \ref{material_evaluation}
einsehbar. An Hand dieser Matrizen wird dann Aussage über den Wert
getroffen.

    Dabei wird zunächst jedes Feld auf dem Schachbrett durchgegangen und die
darauf befindliche Figur berechnet. Dies wird mittels einer
verschachtelten Schleife gelöst, die zunächst alle Reihen durchgeht und
dann die einzelnen Felder in dieser Reihe.

    \begin{Verbatim}[commandchars=\\\{\}]
{\color{incolor}In [{\color{incolor}35}]:} \PY{k+kn}{import} \PY{n+nn}{numpy} \PY{k}{as} \PY{n+nn}{np}
         
         \PY{n}{PAWN\PYZus{}POSITION\PYZus{}MATRIX} \PY{o}{=}      \PY{n}{np}\PY{o}{.}\PY{n}{array}\PY{p}{(}   \PY{p}{[}\PY{p}{[}\PY{l+m+mi}{0}\PY{p}{,}\PY{l+m+mi}{0}\PY{p}{,}\PY{l+m+mi}{0}\PY{p}{,}\PY{l+m+mi}{0}\PY{p}{,}\PY{l+m+mi}{0}\PY{p}{,}\PY{l+m+mi}{0}\PY{p}{,}\PY{l+m+mi}{0}\PY{p}{,}\PY{l+m+mi}{0}\PY{p}{]}\PY{p}{,}
                                                 \PY{p}{[}\PY{l+m+mi}{50}\PY{p}{,}\PY{l+m+mi}{50}\PY{p}{,}\PY{l+m+mi}{50}\PY{p}{,}\PY{l+m+mi}{50}\PY{p}{,}\PY{l+m+mi}{50}\PY{p}{,}\PY{l+m+mi}{50}\PY{p}{,}\PY{l+m+mi}{50}\PY{p}{,}\PY{l+m+mi}{50}\PY{p}{]}\PY{p}{,}
                                                 \PY{p}{[}\PY{l+m+mi}{10}\PY{p}{,}\PY{l+m+mi}{10}\PY{p}{,}\PY{l+m+mi}{20}\PY{p}{,}\PY{l+m+mi}{30}\PY{p}{,}\PY{l+m+mi}{30}\PY{p}{,}\PY{l+m+mi}{20}\PY{p}{,}\PY{l+m+mi}{10}\PY{p}{,}\PY{l+m+mi}{10}\PY{p}{]}\PY{p}{,}
                                                 \PY{p}{[}\PY{l+m+mi}{5}\PY{p}{,}\PY{l+m+mi}{5}\PY{p}{,}\PY{l+m+mi}{10}\PY{p}{,}\PY{l+m+mi}{25}\PY{p}{,}\PY{l+m+mi}{25}\PY{p}{,}\PY{l+m+mi}{10}\PY{p}{,}\PY{l+m+mi}{5}\PY{p}{,}\PY{l+m+mi}{5}\PY{p}{]}\PY{p}{,}
                                                 \PY{p}{[}\PY{l+m+mi}{0}\PY{p}{,}\PY{l+m+mi}{0}\PY{p}{,}\PY{l+m+mi}{0}\PY{p}{,}\PY{l+m+mi}{20}\PY{p}{,}\PY{l+m+mi}{20}\PY{p}{,}\PY{l+m+mi}{0}\PY{p}{,}\PY{l+m+mi}{0}\PY{p}{,}\PY{l+m+mi}{0}\PY{p}{]}\PY{p}{,}
                                                 \PY{p}{[}\PY{l+m+mi}{5}\PY{p}{,}\PY{o}{\PYZhy{}}\PY{l+m+mi}{5}\PY{p}{,}\PY{o}{\PYZhy{}}\PY{l+m+mi}{10}\PY{p}{,}\PY{l+m+mi}{0}\PY{p}{,}\PY{l+m+mi}{0}\PY{p}{,}\PY{o}{\PYZhy{}}\PY{l+m+mi}{10}\PY{p}{,}\PY{o}{\PYZhy{}}\PY{l+m+mi}{5}\PY{p}{,}\PY{l+m+mi}{5}\PY{p}{]}\PY{p}{,}
                                                 \PY{p}{[}\PY{l+m+mi}{5}\PY{p}{,}\PY{l+m+mi}{10}\PY{p}{,}\PY{l+m+mi}{10}\PY{p}{,}\PY{o}{\PYZhy{}}\PY{l+m+mi}{20}\PY{p}{,}\PY{o}{\PYZhy{}}\PY{l+m+mi}{20}\PY{p}{,}\PY{l+m+mi}{10}\PY{p}{,}\PY{l+m+mi}{10}\PY{p}{,}\PY{l+m+mi}{5}\PY{p}{]}\PY{p}{,}
                                                 \PY{p}{[}\PY{l+m+mi}{0}\PY{p}{,}\PY{l+m+mi}{0}\PY{p}{,}\PY{l+m+mi}{0}\PY{p}{,}\PY{l+m+mi}{0}\PY{p}{,}\PY{l+m+mi}{0}\PY{p}{,}\PY{l+m+mi}{0}\PY{p}{,}\PY{l+m+mi}{0}\PY{p}{,}\PY{l+m+mi}{0}\PY{p}{]}\PY{p}{]}\PY{p}{)}
         
         \PY{n}{KNIGHT\PYZus{}POSITION\PYZus{}MATRIX} \PY{o}{=}    \PY{n}{np}\PY{o}{.}\PY{n}{array}\PY{p}{(}   \PY{p}{[}\PY{p}{[}\PY{o}{\PYZhy{}}\PY{l+m+mi}{50}\PY{p}{,}\PY{o}{\PYZhy{}}\PY{l+m+mi}{40}\PY{p}{,}\PY{o}{\PYZhy{}}\PY{l+m+mi}{30}\PY{p}{,}\PY{o}{\PYZhy{}}\PY{l+m+mi}{30}\PY{p}{,}\PY{o}{\PYZhy{}}\PY{l+m+mi}{30}\PY{p}{,}\PY{o}{\PYZhy{}}\PY{l+m+mi}{30}\PY{p}{,}\PY{o}{\PYZhy{}}\PY{l+m+mi}{40}\PY{p}{,}\PY{o}{\PYZhy{}}\PY{l+m+mi}{50}\PY{p}{]}\PY{p}{,}
                                                 \PY{p}{[}\PY{o}{\PYZhy{}}\PY{l+m+mi}{40}\PY{p}{,}\PY{o}{\PYZhy{}}\PY{l+m+mi}{20}\PY{p}{,}\PY{l+m+mi}{0}\PY{p}{,}\PY{l+m+mi}{0}\PY{p}{,}\PY{l+m+mi}{0}\PY{p}{,}\PY{l+m+mi}{0}\PY{p}{,}\PY{o}{\PYZhy{}}\PY{l+m+mi}{20}\PY{p}{,}\PY{o}{\PYZhy{}}\PY{l+m+mi}{40}\PY{p}{]}\PY{p}{,}
                                                 \PY{p}{[}\PY{o}{\PYZhy{}}\PY{l+m+mi}{30}\PY{p}{,}\PY{l+m+mi}{0}\PY{p}{,}\PY{l+m+mi}{10}\PY{p}{,}\PY{l+m+mi}{15}\PY{p}{,}\PY{l+m+mi}{15}\PY{p}{,}\PY{l+m+mi}{10}\PY{p}{,}\PY{l+m+mi}{0}\PY{p}{,}\PY{o}{\PYZhy{}}\PY{l+m+mi}{30}\PY{p}{]}\PY{p}{,}
                                                 \PY{p}{[}\PY{o}{\PYZhy{}}\PY{l+m+mi}{30}\PY{p}{,}\PY{l+m+mi}{5}\PY{p}{,}\PY{l+m+mi}{15}\PY{p}{,}\PY{l+m+mi}{20}\PY{p}{,}\PY{l+m+mi}{20}\PY{p}{,}\PY{l+m+mi}{15}\PY{p}{,}\PY{l+m+mi}{5}\PY{p}{,}\PY{o}{\PYZhy{}}\PY{l+m+mi}{30}\PY{p}{]}\PY{p}{,}
                                                 \PY{p}{[}\PY{o}{\PYZhy{}}\PY{l+m+mi}{30}\PY{p}{,}\PY{l+m+mi}{0}\PY{p}{,}\PY{l+m+mi}{15}\PY{p}{,}\PY{l+m+mi}{20}\PY{p}{,}\PY{l+m+mi}{20}\PY{p}{,}\PY{l+m+mi}{15}\PY{p}{,}\PY{l+m+mi}{0}\PY{p}{,}\PY{o}{\PYZhy{}}\PY{l+m+mi}{30}\PY{p}{]}\PY{p}{,}
                                                 \PY{p}{[}\PY{o}{\PYZhy{}}\PY{l+m+mi}{30}\PY{p}{,}\PY{l+m+mi}{5}\PY{p}{,}\PY{l+m+mi}{10}\PY{p}{,}\PY{l+m+mi}{15}\PY{p}{,}\PY{l+m+mi}{15}\PY{p}{,}\PY{l+m+mi}{10}\PY{p}{,}\PY{l+m+mi}{5}\PY{p}{,}\PY{o}{\PYZhy{}}\PY{l+m+mi}{30}\PY{p}{]}\PY{p}{,}
                                                 \PY{p}{[}\PY{o}{\PYZhy{}}\PY{l+m+mi}{40}\PY{p}{,}\PY{o}{\PYZhy{}}\PY{l+m+mi}{20}\PY{p}{,}\PY{l+m+mi}{0}\PY{p}{,}\PY{l+m+mi}{5}\PY{p}{,}\PY{l+m+mi}{5}\PY{p}{,}\PY{l+m+mi}{0}\PY{p}{,}\PY{o}{\PYZhy{}}\PY{l+m+mi}{20}\PY{p}{,}\PY{o}{\PYZhy{}}\PY{l+m+mi}{40}\PY{p}{]}\PY{p}{,}
                                                 \PY{p}{[}\PY{o}{\PYZhy{}}\PY{l+m+mi}{50}\PY{p}{,}\PY{o}{\PYZhy{}}\PY{l+m+mi}{40}\PY{p}{,}\PY{o}{\PYZhy{}}\PY{l+m+mi}{30}\PY{p}{,}\PY{o}{\PYZhy{}}\PY{l+m+mi}{30}\PY{p}{,}\PY{o}{\PYZhy{}}\PY{l+m+mi}{30}\PY{p}{,}\PY{o}{\PYZhy{}}\PY{l+m+mi}{30}\PY{p}{,}\PY{o}{\PYZhy{}}\PY{l+m+mi}{40}\PY{p}{,}\PY{o}{\PYZhy{}}\PY{l+m+mi}{50}\PY{p}{]}\PY{p}{]}\PY{p}{)}
         
         \PY{n}{BISHOP\PYZus{}POSITION\PYZus{}MATRIX} \PY{o}{=}    \PY{n}{np}\PY{o}{.}\PY{n}{array}\PY{p}{(}   \PY{p}{[}\PY{p}{[}\PY{o}{\PYZhy{}}\PY{l+m+mi}{20}\PY{p}{,}\PY{o}{\PYZhy{}}\PY{l+m+mi}{10}\PY{p}{,}\PY{o}{\PYZhy{}}\PY{l+m+mi}{10}\PY{p}{,}\PY{o}{\PYZhy{}}\PY{l+m+mi}{10}\PY{p}{,}\PY{o}{\PYZhy{}}\PY{l+m+mi}{10}\PY{p}{,}\PY{o}{\PYZhy{}}\PY{l+m+mi}{10}\PY{p}{,}\PY{o}{\PYZhy{}}\PY{l+m+mi}{10}\PY{p}{,}\PY{o}{\PYZhy{}}\PY{l+m+mi}{20}\PY{p}{]}\PY{p}{,}
                                                 \PY{p}{[}\PY{o}{\PYZhy{}}\PY{l+m+mi}{10}\PY{p}{,}\PY{l+m+mi}{0}\PY{p}{,}\PY{l+m+mi}{0}\PY{p}{,}\PY{l+m+mi}{0}\PY{p}{,}\PY{l+m+mi}{0}\PY{p}{,}\PY{l+m+mi}{0}\PY{p}{,}\PY{l+m+mi}{0}\PY{p}{,}\PY{o}{\PYZhy{}}\PY{l+m+mi}{10}\PY{p}{]}\PY{p}{,}
                                                 \PY{p}{[}\PY{o}{\PYZhy{}}\PY{l+m+mi}{10}\PY{p}{,}\PY{l+m+mi}{0}\PY{p}{,}\PY{l+m+mi}{5}\PY{p}{,}\PY{l+m+mi}{10}\PY{p}{,}\PY{l+m+mi}{10}\PY{p}{,}\PY{l+m+mi}{5}\PY{p}{,}\PY{l+m+mi}{0}\PY{p}{,}\PY{o}{\PYZhy{}}\PY{l+m+mi}{10}\PY{p}{]}\PY{p}{,}
                                                 \PY{p}{[}\PY{o}{\PYZhy{}}\PY{l+m+mi}{10}\PY{p}{,}\PY{l+m+mi}{5}\PY{p}{,}\PY{l+m+mi}{5}\PY{p}{,}\PY{l+m+mi}{10}\PY{p}{,}\PY{l+m+mi}{10}\PY{p}{,}\PY{l+m+mi}{5}\PY{p}{,}\PY{l+m+mi}{5}\PY{p}{,}\PY{o}{\PYZhy{}}\PY{l+m+mi}{10}\PY{p}{]}\PY{p}{,}
                                                 \PY{p}{[}\PY{o}{\PYZhy{}}\PY{l+m+mi}{10}\PY{p}{,}\PY{l+m+mi}{0}\PY{p}{,}\PY{l+m+mi}{10}\PY{p}{,}\PY{l+m+mi}{10}\PY{p}{,}\PY{l+m+mi}{10}\PY{p}{,}\PY{l+m+mi}{10}\PY{p}{,}\PY{l+m+mi}{0}\PY{p}{,}\PY{o}{\PYZhy{}}\PY{l+m+mi}{10}\PY{p}{]}\PY{p}{,}
                                                 \PY{p}{[}\PY{o}{\PYZhy{}}\PY{l+m+mi}{10}\PY{p}{,}\PY{l+m+mi}{10}\PY{p}{,}\PY{l+m+mi}{10}\PY{p}{,}\PY{l+m+mi}{10}\PY{p}{,}\PY{l+m+mi}{10}\PY{p}{,}\PY{l+m+mi}{10}\PY{p}{,}\PY{l+m+mi}{10}\PY{p}{,}\PY{o}{\PYZhy{}}\PY{l+m+mi}{10}\PY{p}{]}\PY{p}{,}
                                                 \PY{p}{[}\PY{o}{\PYZhy{}}\PY{l+m+mi}{10}\PY{p}{,}\PY{l+m+mi}{5}\PY{p}{,}\PY{l+m+mi}{0}\PY{p}{,}\PY{l+m+mi}{0}\PY{p}{,}\PY{l+m+mi}{0}\PY{p}{,}\PY{l+m+mi}{0}\PY{p}{,}\PY{l+m+mi}{5}\PY{p}{,}\PY{o}{\PYZhy{}}\PY{l+m+mi}{10}\PY{p}{]}\PY{p}{,}
                                                 \PY{p}{[}\PY{o}{\PYZhy{}}\PY{l+m+mi}{20}\PY{p}{,}\PY{o}{\PYZhy{}}\PY{l+m+mi}{10}\PY{p}{,}\PY{o}{\PYZhy{}}\PY{l+m+mi}{10}\PY{p}{,}\PY{o}{\PYZhy{}}\PY{l+m+mi}{10}\PY{p}{,}\PY{o}{\PYZhy{}}\PY{l+m+mi}{10}\PY{p}{,}\PY{o}{\PYZhy{}}\PY{l+m+mi}{10}\PY{p}{,}\PY{o}{\PYZhy{}}\PY{l+m+mi}{10}\PY{p}{,}\PY{o}{\PYZhy{}}\PY{l+m+mi}{20}\PY{p}{]}\PY{p}{]}\PY{p}{)}
         
         \PY{n}{ROOK\PYZus{}POSITION\PYZus{}MATRIX} \PY{o}{=}      \PY{n}{np}\PY{o}{.}\PY{n}{array}\PY{p}{(}   \PY{p}{[}\PY{p}{[}\PY{l+m+mi}{0}\PY{p}{,}\PY{l+m+mi}{0}\PY{p}{,}\PY{l+m+mi}{0}\PY{p}{,}\PY{l+m+mi}{0}\PY{p}{,}\PY{l+m+mi}{0}\PY{p}{,}\PY{l+m+mi}{0}\PY{p}{,}\PY{l+m+mi}{0}\PY{p}{,}\PY{l+m+mi}{0}\PY{p}{]}\PY{p}{,}
                                                 \PY{p}{[}\PY{l+m+mi}{5}\PY{p}{,}\PY{l+m+mi}{10}\PY{p}{,}\PY{l+m+mi}{10}\PY{p}{,}\PY{l+m+mi}{10}\PY{p}{,}\PY{l+m+mi}{10}\PY{p}{,}\PY{l+m+mi}{10}\PY{p}{,}\PY{l+m+mi}{10}\PY{p}{,}\PY{l+m+mi}{5}\PY{p}{]}\PY{p}{,}
                                                 \PY{p}{[}\PY{o}{\PYZhy{}}\PY{l+m+mi}{5}\PY{p}{,}\PY{l+m+mi}{0}\PY{p}{,}\PY{l+m+mi}{0}\PY{p}{,}\PY{l+m+mi}{0}\PY{p}{,}\PY{l+m+mi}{0}\PY{p}{,}\PY{l+m+mi}{0}\PY{p}{,}\PY{l+m+mi}{0}\PY{p}{,}\PY{o}{\PYZhy{}}\PY{l+m+mi}{5}\PY{p}{]}\PY{p}{,}
                                                 \PY{p}{[}\PY{o}{\PYZhy{}}\PY{l+m+mi}{5}\PY{p}{,}\PY{l+m+mi}{0}\PY{p}{,}\PY{l+m+mi}{0}\PY{p}{,}\PY{l+m+mi}{0}\PY{p}{,}\PY{l+m+mi}{0}\PY{p}{,}\PY{l+m+mi}{0}\PY{p}{,}\PY{l+m+mi}{0}\PY{p}{,}\PY{o}{\PYZhy{}}\PY{l+m+mi}{5}\PY{p}{]}\PY{p}{,}
                                                 \PY{p}{[}\PY{o}{\PYZhy{}}\PY{l+m+mi}{5}\PY{p}{,}\PY{l+m+mi}{0}\PY{p}{,}\PY{l+m+mi}{0}\PY{p}{,}\PY{l+m+mi}{0}\PY{p}{,}\PY{l+m+mi}{0}\PY{p}{,}\PY{l+m+mi}{0}\PY{p}{,}\PY{l+m+mi}{0}\PY{p}{,}\PY{o}{\PYZhy{}}\PY{l+m+mi}{5}\PY{p}{]}\PY{p}{,}
                                                 \PY{p}{[}\PY{o}{\PYZhy{}}\PY{l+m+mi}{5}\PY{p}{,}\PY{l+m+mi}{0}\PY{p}{,}\PY{l+m+mi}{0}\PY{p}{,}\PY{l+m+mi}{0}\PY{p}{,}\PY{l+m+mi}{0}\PY{p}{,}\PY{l+m+mi}{0}\PY{p}{,}\PY{l+m+mi}{0}\PY{p}{,}\PY{o}{\PYZhy{}}\PY{l+m+mi}{5}\PY{p}{]}\PY{p}{,}
                                                 \PY{p}{[}\PY{o}{\PYZhy{}}\PY{l+m+mi}{5}\PY{p}{,}\PY{l+m+mi}{0}\PY{p}{,}\PY{l+m+mi}{0}\PY{p}{,}\PY{l+m+mi}{0}\PY{p}{,}\PY{l+m+mi}{0}\PY{p}{,}\PY{l+m+mi}{0}\PY{p}{,}\PY{l+m+mi}{0}\PY{p}{,}\PY{o}{\PYZhy{}}\PY{l+m+mi}{5}\PY{p}{]}\PY{p}{,}
                                                 \PY{p}{[}\PY{l+m+mi}{0}\PY{p}{,}\PY{l+m+mi}{0}\PY{p}{,}\PY{l+m+mi}{0}\PY{p}{,}\PY{l+m+mi}{5}\PY{p}{,}\PY{l+m+mi}{5}\PY{p}{,}\PY{l+m+mi}{0}\PY{p}{,}\PY{l+m+mi}{0}\PY{p}{,}\PY{l+m+mi}{0}\PY{p}{]}\PY{p}{]}\PY{p}{)}
         
         \PY{n}{QUEEN\PYZus{}POSITION\PYZus{}MATRIX} \PY{o}{=}     \PY{n}{np}\PY{o}{.}\PY{n}{array}\PY{p}{(}   \PY{p}{[}\PY{p}{[}\PY{o}{\PYZhy{}}\PY{l+m+mi}{20}\PY{p}{,}\PY{o}{\PYZhy{}}\PY{l+m+mi}{10}\PY{p}{,}\PY{o}{\PYZhy{}}\PY{l+m+mi}{10}\PY{p}{,}\PY{o}{\PYZhy{}}\PY{l+m+mi}{5}\PY{p}{,}\PY{o}{\PYZhy{}}\PY{l+m+mi}{5}\PY{p}{,}\PY{o}{\PYZhy{}}\PY{l+m+mi}{10}\PY{p}{,}\PY{o}{\PYZhy{}}\PY{l+m+mi}{10}\PY{p}{,}\PY{o}{\PYZhy{}}\PY{l+m+mi}{20}\PY{p}{]}\PY{p}{,}
                                                 \PY{p}{[}\PY{o}{\PYZhy{}}\PY{l+m+mi}{10}\PY{p}{,}\PY{l+m+mi}{0}\PY{p}{,}\PY{l+m+mi}{0}\PY{p}{,}\PY{l+m+mi}{0}\PY{p}{,}\PY{l+m+mi}{0}\PY{p}{,}\PY{l+m+mi}{0}\PY{p}{,}\PY{l+m+mi}{0}\PY{p}{,}\PY{o}{\PYZhy{}}\PY{l+m+mi}{10}\PY{p}{]}\PY{p}{,}
                                                 \PY{p}{[}\PY{o}{\PYZhy{}}\PY{l+m+mi}{10}\PY{p}{,}\PY{l+m+mi}{0}\PY{p}{,}\PY{l+m+mi}{5}\PY{p}{,}\PY{l+m+mi}{5}\PY{p}{,}\PY{l+m+mi}{5}\PY{p}{,}\PY{l+m+mi}{5}\PY{p}{,}\PY{l+m+mi}{0}\PY{p}{,}\PY{o}{\PYZhy{}}\PY{l+m+mi}{10}\PY{p}{]}\PY{p}{,}
                                                 \PY{p}{[}\PY{o}{\PYZhy{}}\PY{l+m+mi}{5}\PY{p}{,}\PY{l+m+mi}{0}\PY{p}{,}\PY{l+m+mi}{5}\PY{p}{,}\PY{l+m+mi}{5}\PY{p}{,}\PY{l+m+mi}{5}\PY{p}{,}\PY{l+m+mi}{5}\PY{p}{,}\PY{l+m+mi}{0}\PY{p}{,}\PY{o}{\PYZhy{}}\PY{l+m+mi}{5}\PY{p}{]}\PY{p}{,}
                                                 \PY{p}{[}\PY{l+m+mi}{0}\PY{p}{,}\PY{l+m+mi}{0}\PY{p}{,}\PY{l+m+mi}{5}\PY{p}{,}\PY{l+m+mi}{5}\PY{p}{,}\PY{l+m+mi}{5}\PY{p}{,}\PY{l+m+mi}{5}\PY{p}{,}\PY{l+m+mi}{0}\PY{p}{,}\PY{o}{\PYZhy{}}\PY{l+m+mi}{5}\PY{p}{]}\PY{p}{,}
                                                 \PY{p}{[}\PY{o}{\PYZhy{}}\PY{l+m+mi}{10}\PY{p}{,}\PY{l+m+mi}{5}\PY{p}{,}\PY{l+m+mi}{5}\PY{p}{,}\PY{l+m+mi}{5}\PY{p}{,}\PY{l+m+mi}{5}\PY{p}{,}\PY{l+m+mi}{5}\PY{p}{,}\PY{l+m+mi}{0}\PY{p}{,}\PY{o}{\PYZhy{}}\PY{l+m+mi}{10}\PY{p}{]}\PY{p}{,}
                                                 \PY{p}{[}\PY{o}{\PYZhy{}}\PY{l+m+mi}{10}\PY{p}{,}\PY{l+m+mi}{0}\PY{p}{,}\PY{l+m+mi}{5}\PY{p}{,}\PY{l+m+mi}{0}\PY{p}{,}\PY{l+m+mi}{0}\PY{p}{,}\PY{l+m+mi}{0}\PY{p}{,}\PY{l+m+mi}{0}\PY{p}{,}\PY{o}{\PYZhy{}}\PY{l+m+mi}{10}\PY{p}{]}\PY{p}{,}
                                                 \PY{p}{[}\PY{o}{\PYZhy{}}\PY{l+m+mi}{20}\PY{p}{,}\PY{o}{\PYZhy{}}\PY{l+m+mi}{10}\PY{p}{,}\PY{o}{\PYZhy{}}\PY{l+m+mi}{10}\PY{p}{,}\PY{o}{\PYZhy{}}\PY{l+m+mi}{5}\PY{p}{,}\PY{o}{\PYZhy{}}\PY{l+m+mi}{5}\PY{p}{,}\PY{o}{\PYZhy{}}\PY{l+m+mi}{10}\PY{p}{,}\PY{o}{\PYZhy{}}\PY{l+m+mi}{10}\PY{p}{,}\PY{o}{\PYZhy{}}\PY{l+m+mi}{20}\PY{p}{]}\PY{p}{]}\PY{p}{)}
         
         \PY{n}{KING\PYZus{}POSITION\PYZus{}MATRIX} \PY{o}{=}      \PY{n}{np}\PY{o}{.}\PY{n}{array}\PY{p}{(}   \PY{p}{[}\PY{p}{[}\PY{o}{\PYZhy{}}\PY{l+m+mi}{30}\PY{p}{,}\PY{o}{\PYZhy{}}\PY{l+m+mi}{40}\PY{p}{,}\PY{o}{\PYZhy{}}\PY{l+m+mi}{40}\PY{p}{,}\PY{o}{\PYZhy{}}\PY{l+m+mi}{50}\PY{p}{,}\PY{o}{\PYZhy{}}\PY{l+m+mi}{50}\PY{p}{,}\PY{o}{\PYZhy{}}\PY{l+m+mi}{40}\PY{p}{,}\PY{o}{\PYZhy{}}\PY{l+m+mi}{40}\PY{p}{,}\PY{o}{\PYZhy{}}\PY{l+m+mi}{30}\PY{p}{]}\PY{p}{,}
                                                 \PY{p}{[}\PY{o}{\PYZhy{}}\PY{l+m+mi}{30}\PY{p}{,}\PY{o}{\PYZhy{}}\PY{l+m+mi}{40}\PY{p}{,}\PY{o}{\PYZhy{}}\PY{l+m+mi}{40}\PY{p}{,}\PY{o}{\PYZhy{}}\PY{l+m+mi}{50}\PY{p}{,}\PY{o}{\PYZhy{}}\PY{l+m+mi}{50}\PY{p}{,}\PY{o}{\PYZhy{}}\PY{l+m+mi}{40}\PY{p}{,}\PY{o}{\PYZhy{}}\PY{l+m+mi}{40}\PY{p}{,}\PY{o}{\PYZhy{}}\PY{l+m+mi}{30}\PY{p}{]}\PY{p}{,}
                                                 \PY{p}{[}\PY{o}{\PYZhy{}}\PY{l+m+mi}{30}\PY{p}{,}\PY{o}{\PYZhy{}}\PY{l+m+mi}{40}\PY{p}{,}\PY{o}{\PYZhy{}}\PY{l+m+mi}{40}\PY{p}{,}\PY{o}{\PYZhy{}}\PY{l+m+mi}{50}\PY{p}{,}\PY{o}{\PYZhy{}}\PY{l+m+mi}{50}\PY{p}{,}\PY{o}{\PYZhy{}}\PY{l+m+mi}{40}\PY{p}{,}\PY{o}{\PYZhy{}}\PY{l+m+mi}{40}\PY{p}{,}\PY{o}{\PYZhy{}}\PY{l+m+mi}{30}\PY{p}{]}\PY{p}{,}
                                                 \PY{p}{[}\PY{o}{\PYZhy{}}\PY{l+m+mi}{30}\PY{p}{,}\PY{o}{\PYZhy{}}\PY{l+m+mi}{40}\PY{p}{,}\PY{o}{\PYZhy{}}\PY{l+m+mi}{40}\PY{p}{,}\PY{o}{\PYZhy{}}\PY{l+m+mi}{50}\PY{p}{,}\PY{o}{\PYZhy{}}\PY{l+m+mi}{50}\PY{p}{,}\PY{o}{\PYZhy{}}\PY{l+m+mi}{40}\PY{p}{,}\PY{o}{\PYZhy{}}\PY{l+m+mi}{40}\PY{p}{,}\PY{o}{\PYZhy{}}\PY{l+m+mi}{30}\PY{p}{]}\PY{p}{,}
                                                 \PY{p}{[}\PY{o}{\PYZhy{}}\PY{l+m+mi}{20}\PY{p}{,}\PY{o}{\PYZhy{}}\PY{l+m+mi}{30}\PY{p}{,}\PY{o}{\PYZhy{}}\PY{l+m+mi}{30}\PY{p}{,}\PY{o}{\PYZhy{}}\PY{l+m+mi}{40}\PY{p}{,}\PY{o}{\PYZhy{}}\PY{l+m+mi}{40}\PY{p}{,}\PY{o}{\PYZhy{}}\PY{l+m+mi}{30}\PY{p}{,}\PY{o}{\PYZhy{}}\PY{l+m+mi}{30}\PY{p}{,}\PY{o}{\PYZhy{}}\PY{l+m+mi}{20}\PY{p}{]}\PY{p}{,}
                                                 \PY{p}{[}\PY{o}{\PYZhy{}}\PY{l+m+mi}{10}\PY{p}{,}\PY{o}{\PYZhy{}}\PY{l+m+mi}{20}\PY{p}{,}\PY{o}{\PYZhy{}}\PY{l+m+mi}{20}\PY{p}{,}\PY{o}{\PYZhy{}}\PY{l+m+mi}{20}\PY{p}{,}\PY{o}{\PYZhy{}}\PY{l+m+mi}{20}\PY{p}{,}\PY{o}{\PYZhy{}}\PY{l+m+mi}{20}\PY{p}{,}\PY{o}{\PYZhy{}}\PY{l+m+mi}{20}\PY{p}{,}\PY{o}{\PYZhy{}}\PY{l+m+mi}{10}\PY{p}{]}\PY{p}{,}
                                                 \PY{p}{[}\PY{l+m+mi}{20}\PY{p}{,}\PY{l+m+mi}{20}\PY{p}{,}\PY{l+m+mi}{0}\PY{p}{,}\PY{l+m+mi}{0}\PY{p}{,}\PY{l+m+mi}{0}\PY{p}{,}\PY{l+m+mi}{0}\PY{p}{,}\PY{l+m+mi}{20}\PY{p}{,}\PY{l+m+mi}{20}\PY{p}{]}\PY{p}{,}
                                                 \PY{p}{[}\PY{l+m+mi}{20}\PY{p}{,}\PY{l+m+mi}{30}\PY{p}{,}\PY{l+m+mi}{10}\PY{p}{,}\PY{l+m+mi}{0}\PY{p}{,}\PY{l+m+mi}{0}\PY{p}{,}\PY{l+m+mi}{10}\PY{p}{,}\PY{l+m+mi}{30}\PY{p}{,}\PY{l+m+mi}{20}\PY{p}{]}\PY{p}{]}\PY{p}{)}
         
         \PY{k}{def} \PY{n+nf}{assign\PYZus{}piece\PYZus{}matrix}\PY{p}{(}\PY{n}{piece\PYZus{}type}\PY{p}{)}\PY{p}{:}
             \PY{k}{return} \PY{p}{\PYZob{}}
                 \PY{l+m+mi}{1}\PY{p}{:} \PY{n}{PAWN\PYZus{}POSITION\PYZus{}MATRIX}\PY{p}{,}
                 \PY{l+m+mi}{2}\PY{p}{:} \PY{n}{KNIGHT\PYZus{}POSITION\PYZus{}MATRIX}\PY{p}{,}
                 \PY{l+m+mi}{3}\PY{p}{:} \PY{n}{BISHOP\PYZus{}POSITION\PYZus{}MATRIX}\PY{p}{,}
                 \PY{l+m+mi}{4}\PY{p}{:} \PY{n}{ROOK\PYZus{}POSITION\PYZus{}MATRIX}\PY{p}{,}
                 \PY{l+m+mi}{5}\PY{p}{:} \PY{n}{QUEEN\PYZus{}POSITION\PYZus{}MATRIX}\PY{p}{,}
                 \PY{l+m+mi}{6}\PY{p}{:} \PY{n}{KING\PYZus{}POSITION\PYZus{}MATRIX}
             \PY{p}{\PYZcb{}}\PY{o}{.}\PY{n}{get}\PY{p}{(}\PY{n}{piece\PYZus{}type}\PY{p}{,} \PY{n}{np}\PY{o}{.}\PY{n}{zeros}\PY{p}{(}\PY{p}{(}\PY{l+m+mi}{8}\PY{p}{,}\PY{l+m+mi}{8}\PY{p}{)}\PY{p}{)}\PY{p}{)}
         
         \PY{k}{def} \PY{n+nf}{get\PYZus{}board\PYZus{}positions\PYZus{}value}\PY{p}{(}\PY{n}{board}\PY{p}{,} \PY{n}{color}\PY{p}{)}\PY{p}{:}
             \PY{n+nb}{sum} \PY{o}{=} \PY{l+m+mi}{0}
             \PY{k}{for} \PY{n}{rank} \PY{o+ow}{in} \PY{n+nb}{range}\PY{p}{(}\PY{l+m+mi}{0}\PY{p}{,}\PY{l+m+mi}{8}\PY{p}{)}\PY{p}{:}
                 \PY{k}{for} \PY{n}{file} \PY{o+ow}{in} \PY{n+nb}{range}\PY{p}{(}\PY{l+m+mi}{0}\PY{p}{,}\PY{l+m+mi}{8}\PY{p}{)}\PY{p}{:}
                     \PY{n}{piece} \PY{o}{=} \PY{n}{board}\PY{o}{.}\PY{n}{piece\PYZus{}at}\PY{p}{(}\PY{n}{chess}\PY{o}{.}\PY{n}{square}\PY{p}{(}\PY{n}{file}\PY{p}{,} \PY{n}{rank}\PY{p}{)}\PY{p}{)}
                     \PY{k}{if} \PY{p}{(}\PY{n}{piece} \PY{o+ow}{and} \PY{n}{piece}\PY{o}{.}\PY{n}{color} \PY{o}{==} \PY{n}{color}\PY{p}{)}\PY{p}{:}
                         \PY{n}{piece\PYZus{}pos\PYZus{}value} \PY{o}{=} \PY{n}{get\PYZus{}position\PYZus{}value\PYZus{}by\PYZus{}square}\PY{p}{(}\PY{n}{board}\PY{p}{,} \PY{n}{rank}\PY{p}{,} \PY{n}{file}\PY{p}{,} \PY{n}{color}\PY{p}{)}
                         \PY{n+nb}{sum} \PY{o}{+}\PY{o}{=} \PY{n}{piece\PYZus{}pos\PYZus{}value}
             \PY{k}{return} \PY{n+nb}{sum} \PY{o}{/} \PY{l+m+mi}{100}
\end{Verbatim}

    Nachdem die Figur ermittelt wurde wird diese, falls diese der
angegebenen Farbe angehört, gemeinsam mit den Werten für Reihe und
Spalte an die Funktion \texttt{get\_position\_value\_by\_square}
übergeben. Nachdem diese den Wert zurückgegeben hat wird dies zu der
bisherigen Summe aufaddiert und am Ende die Summe aller Figuren geteilt
durch 10 zurückgegeben. Der Divisor 10 rührt daher, dass die Positionen
der einzelnen Figuren in dessen Matrizen recht hoch gewertet sind und
somit insgesamt noch etwas abgeschwächt werden müssen, um die Funktion
mit den restlichen Evaluierungsfunktionen ungefähr auf eine
Bedeutungshöhe zu bringen.

In der Funktion \texttt{get\_position\_value\_by\_square} wird mittels
der Matrix der Wert der Figur an der gegebenen Position ermittelt.

    \begin{Verbatim}[commandchars=\\\{\}]
{\color{incolor}In [{\color{incolor}36}]:} \PY{k}{def} \PY{n+nf}{get\PYZus{}position\PYZus{}value\PYZus{}by\PYZus{}square}\PY{p}{(}\PY{n}{board}\PY{p}{,} \PY{n}{rank}\PY{p}{,} \PY{n}{file}\PY{p}{,} \PY{n}{color}\PY{p}{)}\PY{p}{:}
             \PY{n}{piece\PYZus{}type} \PY{o}{=} \PY{n}{board}\PY{o}{.}\PY{n}{piece\PYZus{}type\PYZus{}at}\PY{p}{(}\PY{n}{chess}\PY{o}{.}\PY{n}{square}\PY{p}{(}\PY{n}{file}\PY{p}{,} \PY{n}{rank}\PY{p}{)}\PY{p}{)}
             \PY{n}{piece\PYZus{}matrix} \PY{o}{=} \PY{n}{assign\PYZus{}piece\PYZus{}matrix}\PY{p}{(}\PY{n}{piece\PYZus{}type}\PY{p}{)} \PY{k}{if} \PY{n}{color} \PY{o}{==} \PY{n}{chess}\PY{o}{.}\PY{n}{BLACK} \PY{k}{else} \PY{n}{np}\PY{o}{.}\PY{n}{flip}\PY{p}{(}\PY{n}{assign\PYZus{}piece\PYZus{}matrix}\PY{p}{(}\PY{n}{piece\PYZus{}type}\PY{p}{)}\PY{p}{)}
             \PY{n}{piece\PYZus{}pos\PYZus{}value} \PY{o}{=} \PY{n}{piece\PYZus{}matrix}\PY{p}{[}\PY{n}{rank}\PY{p}{,}\PY{n}{file}\PY{p}{]}
             \PY{k}{return} \PY{n}{piece\PYZus{}pos\PYZus{}value}
\end{Verbatim}

    Dazu wird zunächst der Typ der Figur auf dem gegebenen Feld ermittelt.
Dann wird die dazu passende Matrix bestimmt und entsprechend gespiegelt,
falls die Farbe des angegebenen Spielers weiß sein sollte, damit die
Matrix mit den Positionen aus der Sicht des Spielers übereinstimmt.

Schlussendlich wird der Wert in der Matrix über die Reihe und Spalte
ermittelt und zurückgegeben.

Um die Positionen des Gegners zu berechnen, kann folgende Funktion
dienen, die die angegebene Farbe invertiert und so den Wert der
gegnerischen Positionen berechnen und zurückgeben kann.

    \begin{Verbatim}[commandchars=\\\{\}]
{\color{incolor}In [{\color{incolor}37}]:} \PY{k}{def} \PY{n+nf}{get\PYZus{}opp\PYZus{}board\PYZus{}positions\PYZus{}value}\PY{p}{(}\PY{n}{board}\PY{p}{,} \PY{n}{color}\PY{p}{)}\PY{p}{:}
             \PY{n}{opp\PYZus{}color} \PY{o}{=} \PY{n}{chess}\PY{o}{.}\PY{n}{WHITE} \PY{k}{if} \PY{n}{color} \PY{o+ow}{is} \PY{n}{chess}\PY{o}{.}\PY{n}{BLACK} \PY{k}{else} \PY{n}{chess}\PY{o}{.}\PY{n}{BLACK}
             \PY{k}{return} \PY{o}{\PYZhy{}}\PY{l+m+mi}{1} \PY{o}{*} \PY{n}{get\PYZus{}board\PYZus{}positions\PYZus{}value}\PY{p}{(}\PY{n}{board}\PY{p}{,} \PY{n}{opp\PYZus{}color}\PY{p}{)}
\end{Verbatim}

    \subsection{Königszonen-Sicherheit}\label{kuxf6nigszonen-sicherheit}

Ein weiterer, wichtiger Wert ist die Sicherheit des Königs bemessen anhand der Figuren, die dessen Zone angreifen.

Dazu wird diese Zone berechnet und dann alle Figuren, die diese Zone
angreifen ermittelt. Mittels der in Kapitel \ref{defense_evaluation} vorgestellten
Berechnung wird dann der Wert des Angriffs auf die Königszone berechnet.

        \begin{Verbatim}[commandchars=\\\{\}]
{\color{incolor}In [{\color{incolor}38}]:} \PY{k}{def} \PY{n+nf}{calculate\PYZus{}king\PYZus{}zone\PYZus{}safety}\PY{p}{(}\PY{n}{board}\PY{p}{,} \PY{n}{color}\PY{p}{)}\PY{p}{:}
             \PY{n}{attacker\PYZus{}color} \PY{o}{=} \PY{n}{chess}\PY{o}{.}\PY{n}{WHITE} \PY{k}{if} \PY{n}{color} \PY{o}{==} \PY{n}{chess}\PY{o}{.}\PY{n}{BLACK} \PY{k}{else} \PY{n}{chess}\PY{o}{.}\PY{n}{BLACK}
             \PY{n}{king\PYZus{}zone} \PY{o}{=} \PY{n}{calculate\PYZus{}king\PYZus{}zone}\PY{p}{(}\PY{n}{board}\PY{p}{,} \PY{n}{color}\PY{p}{)}
             \PY{n}{attackers} \PY{o}{=} \PY{n}{get\PYZus{}attackers\PYZus{}by\PYZus{}squares}\PY{p}{(}\PY{n}{board}\PY{p}{,} \PY{n}{king\PYZus{}zone}\PY{p}{,} \PY{n}{attacker\PYZus{}color}\PY{p}{)}
             \PY{n}{attack\PYZus{}weight} \PY{o}{=} \PY{n}{get\PYZus{}king\PYZus{}attack\PYZus{}weight}\PY{p}{(}\PY{n+nb}{len}\PY{p}{(}\PY{n}{attackers}\PY{p}{)}\PY{p}{)}
             \PY{n}{value\PYZus{}of\PYZus{}attack} \PY{o}{=} \PY{l+m+mi}{0}
             \PY{k}{for} \PY{n}{attacker} \PY{o+ow}{in} \PY{n}{attackers}\PY{p}{:}
                 \PY{n}{value\PYZus{}of\PYZus{}attack} \PY{o}{+}\PY{o}{=} \PY{n}{get\PYZus{}king\PYZus{}attack\PYZus{}constants}\PY{p}{(}\PY{n}{attacker}\PY{o}{.}\PY{n}{piece\PYZus{}type}\PY{p}{)}
             
             \PY{k}{return} \PY{o}{\PYZhy{}}\PY{l+m+mi}{1} \PY{o}{*} \PY{p}{(}\PY{n}{value\PYZus{}of\PYZus{}attack} \PY{o}{*} \PY{n}{attack\PYZus{}weight}\PY{p}{)} \PY{o}{/} \PY{l+m+mi}{1000}
\end{Verbatim}

    Dabei wird zuerst die Farbe des Angreifers berechnet, indem die gegebene
Farbe umgekehrt wird. Danach wird mittels der Funktion
\texttt{calculate\_king\_zone} die Königszone berechnet. Diese Funktion
gibt eine Menge von Feldern zurück, die der Königszone angehören.

    \begin{Verbatim}[commandchars=\\\{\}]
{\color{incolor}In [{\color{incolor}39}]:} \PY{k}{def} \PY{n+nf}{calculate\PYZus{}king\PYZus{}zone}\PY{p}{(}\PY{n}{board}\PY{p}{,} \PY{n}{color}\PY{p}{)}\PY{p}{:}
             \PY{n}{king\PYZus{}zone} \PY{o}{=} \PY{n}{chess}\PY{o}{.}\PY{n}{SquareSet}\PY{p}{(}\PY{p}{)}
             \PY{n}{king\PYZus{}rank}\PY{p}{,} \PY{n}{king\PYZus{}file} \PY{o}{=} \PY{n}{get\PYZus{}piece\PYZus{}position}\PY{p}{(}\PY{n}{board}\PY{p}{,} \PY{n}{chess}\PY{o}{.}\PY{n}{Piece}\PY{p}{(}\PY{n}{chess}\PY{o}{.}\PY{n}{KING}\PY{p}{,} \PY{n}{color}\PY{p}{)}\PY{p}{)}
         
             \PY{n}{rank\PYZus{}range} \PY{o}{=} \PY{n+nb}{range}\PY{p}{(}\PY{l+m+mi}{0}\PY{p}{,}\PY{l+m+mi}{4}\PY{p}{)} \PY{k}{if} \PY{n}{color} \PY{o}{==} \PY{n}{chess}\PY{o}{.}\PY{n}{WHITE} \PY{k}{else} \PY{n+nb}{range}\PY{p}{(}\PY{o}{\PYZhy{}}\PY{l+m+mi}{3}\PY{p}{,} \PY{l+m+mi}{1}\PY{p}{)}
             \PY{k}{for} \PY{n}{rank\PYZus{}summand} \PY{o+ow}{in} \PY{n}{rank\PYZus{}range}\PY{p}{:}
                 \PY{k}{if} \PY{p}{(}\PY{n}{king\PYZus{}rank} \PY{o}{+} \PY{n}{rank\PYZus{}summand}\PY{p}{)} \PY{o+ow}{in} \PY{n+nb}{range}\PY{p}{(}\PY{l+m+mi}{0}\PY{p}{,} \PY{l+m+mi}{8}\PY{p}{)}\PY{p}{:}
                     \PY{k}{for} \PY{n}{file\PYZus{}summand} \PY{o+ow}{in} \PY{n+nb}{range}\PY{p}{(}\PY{o}{\PYZhy{}}\PY{l+m+mi}{1}\PY{p}{,} \PY{l+m+mi}{2}\PY{p}{)}\PY{p}{:}
                         \PY{k}{if} \PY{p}{(}\PY{n}{king\PYZus{}file} \PY{o}{+} \PY{n}{file\PYZus{}summand}\PY{p}{)} \PY{o+ow}{in} \PY{n+nb}{range} \PY{p}{(}\PY{l+m+mi}{0}\PY{p}{,}\PY{l+m+mi}{8}\PY{p}{)}\PY{p}{:}
                             \PY{n}{king\PYZus{}zone}\PY{o}{.}\PY{n}{add}\PY{p}{(}\PY{n}{chess}\PY{o}{.}\PY{n}{square}\PY{p}{(}\PY{n}{king\PYZus{}file} \PY{o}{+} \PY{n}{file\PYZus{}summand}\PY{p}{,} \PY{n}{king\PYZus{}rank} \PY{o}{+} \PY{n}{rank\PYZus{}summand}\PY{p}{)}\PY{p}{)}
             
             \PY{k}{return} \PY{n}{king\PYZus{}zone}
\end{Verbatim}

    Um dies zu ermöglichen, wird zunächst die Position des Königs ermittelt.
Dabei werden alle Felder durchgegangen bis der König der entsprechenden
Farbe gefunden wurde. Dann wird Reihe sowie Spalte zurückgegeben:

        \begin{Verbatim}[commandchars=\\\{\}]
{\color{incolor}In [{\color{incolor}40}]:} \PY{k}{def} \PY{n+nf}{get\PYZus{}piece\PYZus{}position}\PY{p}{(}\PY{n}{board}\PY{p}{,} \PY{n}{piece}\PY{p}{)}\PY{p}{:}
             \PY{k}{for} \PY{n}{rank} \PY{o+ow}{in} \PY{n+nb}{range}\PY{p}{(}\PY{l+m+mi}{0}\PY{p}{,}\PY{l+m+mi}{8}\PY{p}{)}\PY{p}{:}
                 \PY{k}{for} \PY{n}{file} \PY{o+ow}{in} \PY{n+nb}{range}\PY{p}{(}\PY{l+m+mi}{0}\PY{p}{,}\PY{l+m+mi}{8}\PY{p}{)}\PY{p}{:}
                     \PY{k}{if} \PY{n}{board}\PY{o}{.}\PY{n}{piece\PYZus{}at}\PY{p}{(}\PY{n}{chess}\PY{o}{.}\PY{n}{square}\PY{p}{(}\PY{n}{file}\PY{p}{,} \PY{n}{rank}\PY{p}{)}\PY{p}{)} \PY{o}{==} \PY{n}{piece}\PY{p}{:}
                         \PY{k}{return} \PY{n}{rank}\PY{p}{,} \PY{n}{file}
\end{Verbatim}

    Nun werden alle Felder der Königszone einzeln durchgegangen und zu der
Menge an Feldern hinzugefügt. Dabei werden bei den Spalten alle Felder
bis 3 Felder in Richtung des Gegners durchgegangen und für jede Spalte
ein Feld links bis zu einem Feld rechts von dem König mitgezählt. Hierfür
wird zuvor jeweils überprüft, ob sich die Spalte beziehungsweise die
Reihe noch auf dem Spielfeld befindet. Ist dies der Fall, wird das Feld
der Menge hinzugefügt und diese wird am Ende zurückgegeben.

Als nächstes wird dann für diese Menge an Feldern ermittelt, welche
Figuren diese Felder angreifen. Dies wird mittels der
\texttt{get\_attackers\_by\_squares} Funktion durchgeführt.

        \begin{Verbatim}[commandchars=\\\{\}]
{\color{incolor}In [{\color{incolor}41}]:} \PY{k}{def} \PY{n+nf}{get\PYZus{}attackers\PYZus{}by\PYZus{}squares}\PY{p}{(}\PY{n}{board}\PY{p}{,} \PY{n}{square\PYZus{}set}\PY{p}{,} \PY{n}{attacker\PYZus{}color}\PY{p}{)}\PY{p}{:}
             \PY{n}{attacker\PYZus{}dict} \PY{o}{=} \PY{p}{\PYZob{}}\PY{p}{\PYZcb{}}
             \PY{k}{for} \PY{n}{square} \PY{o+ow}{in} \PY{n}{square\PYZus{}set}\PY{p}{:}
                 \PY{n}{attacker\PYZus{}square\PYZus{}set} \PY{o}{=} \PY{n}{board}\PY{o}{.}\PY{n}{attackers}\PY{p}{(}\PY{n}{attacker\PYZus{}color}\PY{p}{,} \PY{n}{square}\PY{p}{)}
                 \PY{k}{for} \PY{n}{attacker\PYZus{}square} \PY{o+ow}{in} \PY{n}{attacker\PYZus{}square\PYZus{}set}\PY{p}{:}
                     \PY{n}{attacker\PYZus{}piece} \PY{o}{=} \PY{n}{board}\PY{o}{.}\PY{n}{piece\PYZus{}at}\PY{p}{(}\PY{n}{attacker\PYZus{}square}\PY{p}{)}
                     \PY{k}{if} \PY{o+ow}{not} \PY{p}{(}\PY{n}{attacker\PYZus{}piece}\PY{o}{.}\PY{n}{piece\PYZus{}type} \PY{o+ow}{is} \PY{n}{chess}\PY{o}{.}\PY{n}{PAWN} \PY{o+ow}{or} \PY{n}{attacker\PYZus{}piece}\PY{o}{.}\PY{n}{piece\PYZus{}type} \PY{o+ow}{is} \PY{n}{chess}\PY{o}{.}\PY{n}{KING}\PY{p}{)}\PY{p}{:}
                         \PY{n}{attacker\PYZus{}dict}\PY{p}{[}\PY{n}{attacker\PYZus{}piece}\PY{p}{]} \PY{o}{=} \PY{n}{attacker\PYZus{}dict}\PY{o}{.}\PY{n}{get}\PY{p}{(}\PY{n}{attacker\PYZus{}piece}\PY{p}{,} \PY{l+m+mi}{0}\PY{p}{)} \PY{o}{+} \PY{l+m+mi}{1}
             \PY{k}{return} \PY{n}{attacker\PYZus{}dict}
\end{Verbatim}

    Dabei wird jedes Feld einzeln durchgegangen und für alle jeweils eine
Menge an Figuren erstellt, die dieses Feld angreifen. Alle Angreifer
werden dann einzeln durchgegangen und dessen Figurentyp ermittelt. Wenn
dies weder ein Bauer noch ein König ist, wird die Figur einem Dictionary
von allen Angreifern hinzugefügt. Der Wert dieses Eintrags setzt sich
aus der Anzahl der Felder zusammen, die von dieser Figur angegriffen
werden. Dieses Dictionary wird nach Durchgang jedes Feldes
zurückgegeben.

Nachdem dieses Dictionary zurückgegeben wurde, wird die Gewichtung der
Attacke ermittelt. Dabei wird die Anzahl der verschiedenen Figuren, die
die Zone angreifen, zur Hilfe genommen und ein entsprechender Wert
zurückgegeben, wie in Kapitel \ref{defense_evaluation} beschrieben.

        \begin{Verbatim}[commandchars=\\\{\}]
{\color{incolor}In [{\color{incolor}42}]:} \PY{k}{def} \PY{n+nf}{get\PYZus{}king\PYZus{}attack\PYZus{}weight}\PY{p}{(}\PY{n}{piece\PYZus{}counter}\PY{p}{)}\PY{p}{:}
             \PY{k}{return} \PY{p}{\PYZob{}}
                 \PY{l+m+mi}{0}\PY{p}{:} \PY{l+m+mi}{0}\PY{p}{,}
                 \PY{l+m+mi}{1}\PY{p}{:} \PY{l+m+mi}{0}\PY{p}{,}
                 \PY{l+m+mi}{2}\PY{p}{:} \PY{l+m+mi}{50}\PY{p}{,}
                 \PY{l+m+mi}{3}\PY{p}{:} \PY{l+m+mi}{75}\PY{p}{,}
                 \PY{l+m+mi}{4}\PY{p}{:} \PY{l+m+mi}{88}\PY{p}{,}
                 \PY{l+m+mi}{5}\PY{p}{:} \PY{l+m+mi}{94}\PY{p}{,}
                 \PY{l+m+mi}{6}\PY{p}{:} \PY{l+m+mi}{97}\PY{p}{,}
                 \PY{l+m+mi}{7}\PY{p}{:} \PY{l+m+mi}{99}
             \PY{p}{\PYZcb{}}\PY{o}{.}\PY{n}{get}\PY{p}{(}\PY{n}{piece\PYZus{}counter}\PY{p}{)}
         
         \PY{k}{def} \PY{n+nf}{get\PYZus{}king\PYZus{}attack\PYZus{}constants}\PY{p}{(}\PY{n}{piece}\PY{p}{)}\PY{p}{:}
             \PY{k}{return} \PY{p}{\PYZob{}}
                 \PY{n}{chess}\PY{o}{.}\PY{n}{KNIGHT}\PY{p}{:} \PY{l+m+mi}{20}\PY{p}{,}
                 \PY{n}{chess}\PY{o}{.}\PY{n}{BISHOP}\PY{p}{:} \PY{l+m+mi}{20}\PY{p}{,}
                 \PY{n}{chess}\PY{o}{.}\PY{n}{ROOK}\PY{p}{:} \PY{l+m+mi}{40}\PY{p}{,}
                 \PY{n}{chess}\PY{o}{.}\PY{n}{QUEEN}\PY{p}{:} \PY{l+m+mi}{80}
             \PY{p}{\PYZcb{}}\PY{o}{.}\PY{n}{get}\PY{p}{(}\PY{n}{piece}\PY{p}{,} \PY{l+m+mi}{0}\PY{p}{)}
\end{Verbatim}

    Abschließend wird für jeden Angreifer noch der Wert dieses ermittelt,
wie ebenfalls in Kapitel \ref{defense_evaluation} beschrieben, und diese alle aufaddiert.
Dieser wird dann mit dem berechneten Gewicht multipliziert und
durch 1000 dividiert. Das Ergebnis aus dieser Berechnung wird dann
zurückgegeben und gibt einen Aufschluss über die Sicherheit des Königs.
Dies kann auch für die Sicherheit des gegnerischen Königs angewandt
werden, indem schlicht die Farbe des Angreifers auf die eigene Farbe
gesetzt wird. Auch hier kann die Sicherheit des gegnerischen Königs
berechnet werden, indem die Funktion ausgeführt wird, nachdem zuvor die
angegebene Farbe invertiert wird.

    \begin{Verbatim}[commandchars=\\\{\}]
{\color{incolor}In [{\color{incolor}43}]:} \PY{k}{def} \PY{n+nf}{calculate\PYZus{}opp\PYZus{}king\PYZus{}zone\PYZus{}safety}\PY{p}{(}\PY{n}{board}\PY{p}{,} \PY{n}{color}\PY{p}{)}\PY{p}{:}
             \PY{n}{opp\PYZus{}color} \PY{o}{=} \PY{n}{chess}\PY{o}{.}\PY{n}{WHITE} \PY{k}{if} \PY{n}{color} \PY{o+ow}{is} \PY{n}{chess}\PY{o}{.}\PY{n}{BLACK} \PY{k}{else} \PY{n}{chess}\PY{o}{.}\PY{n}{BLACK}
             \PY{k}{return} \PY{o}{\PYZhy{}}\PY{l+m+mi}{1} \PY{o}{*} \PY{n}{calculate\PYZus{}king\PYZus{}zone\PYZus{}safety}\PY{p}{(}\PY{n}{board}\PY{p}{,} \PY{n}{opp\PYZus{}color}\PY{p}{)}
\end{Verbatim}

    \subsection{Mobilität}\label{mobilituxe4t}

Bei der Berechnung eines Wertes zur Mobilität wird dieser anhand der
Differenz der legalen Züge von Spieler schwarz und Spieler weiß
festgemacht. Dazu wird zunächst überprüft, ob es überhaupt möglich ist
einen Zug auszuführen. Ist dies nicht der Fall ist sowohl die Anzahl der
Züge des Spielers, der am Zug ist, als auch die des nachfolgenden
Spielers 0. Andernfalls wird ein zufälliger (hier: der erste) legaler
Zug ausgeführt und der Wert der legalen Züge von diesem berechnet. Dabei
wird aus Performanz Gründen nicht der Durchschnitt aller möglichen Züge
berechnet, sondern ein zufälliger Wert. Andernfalls hätte diese einzelne
Evaluierungsfunktion so viel Zeit in Anspruch genommen, dass die
Gesamtzahl an evaluierbaren Zügen um knapp ein Zehntel sinkt.

    \begin{Verbatim}[commandchars=\\\{\}]
{\color{incolor}In [{\color{incolor}44}]:} \PY{k}{def} \PY{n+nf}{get\PYZus{}num\PYZus{}of\PYZus{}legal\PYZus{}moves\PYZus{}by\PYZus{}color}\PY{p}{(}\PY{n}{board}\PY{p}{,} \PY{n}{color}\PY{p}{)}\PY{p}{:}
             \PY{n}{player} \PY{o}{=} \PY{n}{chess}\PY{o}{.}\PY{n}{WHITE} \PY{k}{if} \PY{n+nb}{bool}\PY{p}{(}\PY{n}{board}\PY{o}{.}\PY{n}{turn}\PY{p}{)} \PY{k}{else} \PY{n}{chess}\PY{o}{.}\PY{n}{BLACK}
             \PY{k}{if} \PY{n}{player} \PY{o}{==} \PY{n}{color} \PY{o+ow}{or} \PY{n+nb}{len}\PY{p}{(}\PY{n+nb}{list}\PY{p}{(}\PY{n}{board}\PY{o}{.}\PY{n}{legal\PYZus{}moves}\PY{p}{)}\PY{p}{)} \PY{o+ow}{is} \PY{l+m+mi}{0}\PY{p}{:}
                 \PY{k}{return} \PY{n+nb}{len}\PY{p}{(}\PY{n+nb}{list}\PY{p}{(}\PY{n}{board}\PY{o}{.}\PY{n}{legal\PYZus{}moves}\PY{p}{)}\PY{p}{)}
             \PY{k}{else}\PY{p}{:}
                 \PY{n}{overall\PYZus{}legal\PYZus{}moves} \PY{o}{=} \PY{l+m+mi}{0}
                 \PY{k}{for} \PY{n}{action} \PY{o+ow}{in} \PY{n}{board}\PY{o}{.}\PY{n}{legal\PYZus{}moves}\PY{p}{:}
                     \PY{n}{tmp\PYZus{}board} \PY{o}{=} \PY{n}{chess}\PY{o}{.}\PY{n}{Board}\PY{p}{(}\PY{n+nb}{str}\PY{p}{(}\PY{n}{board}\PY{o}{.}\PY{n}{fen}\PY{p}{(}\PY{p}{)}\PY{p}{)}\PY{p}{)}
                     \PY{n}{tmp\PYZus{}board}\PY{o}{.}\PY{n}{push}\PY{p}{(}\PY{n}{action}\PY{p}{)}
                     \PY{n}{overall\PYZus{}legal\PYZus{}moves} \PY{o}{+}\PY{o}{=} \PY{n+nb}{len}\PY{p}{(}\PY{n+nb}{list}\PY{p}{(}\PY{n}{tmp\PYZus{}board}\PY{o}{.}\PY{n}{legal\PYZus{}moves}\PY{p}{)}\PY{p}{)}
                 \PY{k}{return} \PY{n}{overall\PYZus{}legal\PYZus{}moves} \PY{o}{/} \PY{n+nb}{len}\PY{p}{(}\PY{n+nb}{list}\PY{p}{(}\PY{n}{board}\PY{o}{.}\PY{n}{legal\PYZus{}moves}\PY{p}{)}\PY{p}{)}
             
         \PY{k}{def} \PY{n+nf}{calculate\PYZus{}mobility\PYZus{}value}\PY{p}{(}\PY{n}{board}\PY{p}{,} \PY{n}{color}\PY{p}{)}\PY{p}{:}
             \PY{n}{player} \PY{o}{=} \PY{n}{chess}\PY{o}{.}\PY{n}{WHITE} \PY{k}{if} \PY{n+nb}{bool}\PY{p}{(}\PY{n}{board}\PY{o}{.}\PY{n}{turn}\PY{p}{)} \PY{k}{else} \PY{n}{chess}\PY{o}{.}\PY{n}{BLACK}
             \PY{n}{current\PYZus{}turn\PYZus{}len} \PY{o}{=} \PY{n+nb}{len}\PY{p}{(}\PY{n+nb}{list}\PY{p}{(}\PY{n}{board}\PY{o}{.}\PY{n}{legal\PYZus{}moves}\PY{p}{)}\PY{p}{)}
         
             \PY{k}{if} \PY{n}{current\PYZus{}turn\PYZus{}len} \PY{o}{\PYZgt{}} \PY{l+m+mi}{0}\PY{p}{:}
                 \PY{n}{tmp\PYZus{}board} \PY{o}{=} \PY{n}{chess}\PY{o}{.}\PY{n}{Board}\PY{p}{(}\PY{n+nb}{str}\PY{p}{(}\PY{n}{board}\PY{o}{.}\PY{n}{fen}\PY{p}{(}\PY{p}{)}\PY{p}{)}\PY{p}{)}
                 \PY{n}{tmp\PYZus{}board}\PY{o}{.}\PY{n}{push}\PY{p}{(}\PY{n+nb}{list}\PY{p}{(}\PY{n}{board}\PY{o}{.}\PY{n}{legal\PYZus{}moves}\PY{p}{)}\PY{p}{[}\PY{l+m+mi}{0}\PY{p}{]}\PY{p}{)}
                 \PY{n}{next\PYZus{}turn\PYZus{}len} \PY{o}{=} \PY{n+nb}{len}\PY{p}{(}\PY{n+nb}{list}\PY{p}{(}\PY{n}{tmp\PYZus{}board}\PY{o}{.}\PY{n}{legal\PYZus{}moves}\PY{p}{)}\PY{p}{)}
             \PY{k}{else}\PY{p}{:}
                 \PY{n}{next\PYZus{}turn\PYZus{}len} \PY{o}{=} \PY{l+m+mi}{0}
             
             \PY{k}{if} \PY{n}{player} \PY{o}{==} \PY{n}{color}\PY{p}{:}
                 \PY{k}{return} \PY{p}{(}\PY{n}{current\PYZus{}turn\PYZus{}len} \PY{o}{\PYZhy{}} \PY{n}{next\PYZus{}turn\PYZus{}len}\PY{p}{)} \PY{o}{/} \PY{l+m+mi}{10}
             \PY{k}{else}\PY{p}{:}
                 \PY{k}{return} \PY{p}{(}\PY{n}{next\PYZus{}turn\PYZus{}len} \PY{o}{\PYZhy{}} \PY{n}{current\PYZus{}turn\PYZus{}len}\PY{p}{)} \PY{o}{/} \PY{l+m+mi}{10}
\end{Verbatim}

    Abschließend wird die Differenz der Anzahl der legalen Züge vom
aktuellen Spielzustand und des berechneten möglichen nächsten Zustandes
berechnet und zurückgegeben. Dabei wird entweder ein positiver Wert
zurückgegeben, wenn der aktuelle Spieler mehr Züge hat. Dies ist der
Fall, wenn der Spieler, der aktuell an der Reihe ist, dem entspricht,
aus dessen Sicht der Zustand evaluiert wird. Andernfalls wird ein Wert
aus der Sicht des Spielers zurückgegeben, der als nächstes am Zug ist.
So wird gewährleistet, dass das Ergebnis stets aus Sicht des Spielers
berechnet wird, aus dessen Sicht der Zustand evaluiert werden soll.

    \subsection{Bewertung an Hand angelegter
Historie}\label{bewertung-an-hand-angelegter-historie}

Besonders am Anfang des Spiels ist es oft unschlüssig, wie man ein
Schachbrett bewerten kann, da noch sehr viele Optionen des Spielverlaufs
offen sind. Dabei kann eine angelegte Historie helfen, die Aufschluss
über Siegchancen geben kann. Wie eine solche Historie angelegt werden
kann, wurde in Kapitel \ref{verwalten-des-schachspiels-und-pflege-des-spielverlaufs} besprochen.

    \begin{Verbatim}[commandchars=\\\{\}]
{\color{incolor}In [{\color{incolor}45}]:} \PY{k}{def} \PY{n+nf}{get\PYZus{}board\PYZus{}value\PYZus{}by\PYZus{}history}\PY{p}{(}\PY{n}{board}\PY{p}{,} \PY{n}{color}\PY{p}{)}\PY{p}{:}
             \PY{n}{dataset} \PY{o}{=} \PY{n}{pd}\PY{o}{.}\PY{n}{read\PYZus{}csv}\PY{p}{(}\PY{n}{HISTORY\PYZus{}FILE\PYZus{}LOC}\PY{p}{)}
             \PY{n}{row} \PY{o}{=} \PY{n}{dataset}\PY{o}{.}\PY{n}{loc}\PY{p}{[}\PY{n}{dataset}\PY{p}{[}\PY{l+s+s1}{\PYZsq{}}\PY{l+s+s1}{board}\PY{l+s+s1}{\PYZsq{}}\PY{p}{]} \PY{o}{==} \PY{n}{board}\PY{o}{.}\PY{n}{fen}\PY{p}{(}\PY{p}{)}\PY{o}{.}\PY{n}{split}\PY{p}{(}\PY{l+s+s2}{\PYZdq{}}\PY{l+s+s2}{ }\PY{l+s+s2}{\PYZdq{}}\PY{p}{)}\PY{p}{[}\PY{l+m+mi}{0}\PY{p}{]}\PY{p}{]}
             \PY{n}{value} \PY{o}{=} \PY{n}{row}\PY{p}{[}\PY{l+s+s1}{\PYZsq{}}\PY{l+s+s1}{value}\PY{l+s+s1}{\PYZsq{}}\PY{p}{]}\PY{o}{.}\PY{n}{item}\PY{p}{(}\PY{p}{)} \PY{k}{if} \PY{n+nb}{len}\PY{p}{(}\PY{n}{row}\PY{p}{[}\PY{l+s+s1}{\PYZsq{}}\PY{l+s+s1}{value}\PY{l+s+s1}{\PYZsq{}}\PY{p}{]}\PY{p}{)} \PY{o}{==} \PY{l+m+mi}{1} \PY{k}{else} \PY{l+m+mi}{0}
             \PY{k}{return} \PY{n}{value} \PY{k}{if} \PY{n}{color} \PY{o+ow}{is} \PY{n}{chess}\PY{o}{.}\PY{n}{WHITE} \PY{k}{else} \PY{o}{\PYZhy{}}\PY{l+m+mi}{1}\PY{o}{*}\PY{n}{value}
\end{Verbatim}

    Abgefragt werden kann diese, indem zunächst die Daten aus der Datei
geladen werden. Dann wird die Reihe abstrahiert, dessen Wert in der
Spalte ``board`` dem mitgegebenen Zustand (in FEN-Notation konvertiert)
gleicht. Dazu wird der entsprechende Wert ausgelesen und zurückgegeben.
Beim Zurückgeben wird der Wert noch negiert, falls der angegebene
Spieler der der schwarzen Steine ist, da in der Historie die Werte aus
Sicht des Spielers der weißen Figuren gespeichert wird.

    \section{Ausgabe des Schachbretts und Eingabe von
Schachzügen}\label{ausgabe-des-schachbretts-und-eingabe-von-schachzuxfcgen}

Nachdem die Schach-KI bereits Berechnungen durchführen und somit auf
Schachzüge reagieren kann ist die Visualisierung des Spielverlaufs ein
essenzieller Bestandteil, wenn ein Spieler mit der KI spielen soll.
Dabei hat die Visualisierung einerseits die Aufgabe die aktuelle
Spielsituation zu illustrieren, als auch andererseits die vom Spieler
gewählten Züge entgegenzunehmen.

In Kapitel \ref{kriterienformulierung} sind die Kriterien für diese wissenschaftliche Arbeit
aufgelistet und ein wichtiger Aspekt dabei ist die Visualisierung des
Spiels mittels ASCII-Zeichen in der Konsole. Damit das Schachbrett
angezeigt werden kann werden zwei verschiedene Bibliotheken benötigt.

\texttt{sty} ist eine Python Bibliothek, die die Hintergrund- und
Schriftfarbe in der Konsole für einzelne Zeichen angeben kann. Hierbei
unterstützt die Bibliothek verschiedene 24bit Farben, die eine optimale
Auswahl an Farben für das Schachbrett garantieren. Eine Hürde bei der
Umsetzung ist, dass Microsoft Windows Betriebssysteme aktuell Probleme
mit der Visualisierung und der Verwendung dieses Moduls hat. Um jedoch
trotzdem ein Schachbrett mit unterschiedlich farbigen Feldern und
Figuren, die sich von dem Hintergrund der Konsole abheben, garantieren
zu können wird die Bibliothek \texttt{colorama} benötigt.

\texttt{colorama} ist ebenfalls eine Python Bibliothek, mittels der
Hintergrund- und Schriftfarbe angegeben werden kann. Jedoch ist der
Farbraum auf acht verschiedene Farben begrenzt und bietet somit im
Gegensatz zu \texttt{sty}, bei der Bibliothek 16.777.216 verschiedene
Farben verfügbar sind, eine weniger gut unterscheidbare Farbpalette für
die Visualisierung des Schachbretts.

Aus der Bibliothek \texttt{sty} muss \texttt{fg} für die Darstellung der
Schriftfarbe, \texttt{bg} für die Darstellung der Hintergrundfarbe und
\texttt{rs} importiert werden. \texttt{rs} wird benötigt um die Farben
nach Verwendung wieder auf die Standardfarben der Konsole zu setzen.

Hingegen muss aus der Bibliothek \texttt{colorama} die Funktion
\texttt{init} importiert werden, die speziell für die Visualisierung der
Farben auf Microsoft Windows Betriebssystemen benötigt wird. Dabei
filtert die Funktion \texttt{init} die ANSI Zeichenketten, da diese von
Microsoft Windows Produkten nicht untertstützt wird und ersetzt diese
durch die äquivalenten Windows Befehle.

Ebenfalls wird aus \texttt{colorama} die Unterstützung für die
Veränderung von Schrift- und Hintergrundfarben
\texttt{Fore}/\texttt{Back} importiert.

    \begin{Verbatim}[commandchars=\\\{\}]
{\color{incolor}In [{\color{incolor}42}]:} \PY{k+kn}{from} \PY{n+nn}{sty} \PY{k}{import} \PY{n}{fg}\PY{p}{,} \PY{n}{bg}\PY{p}{,} \PY{n}{rs}
         \PY{k+kn}{from} \PY{n+nn}{colorama} \PY{k}{import} \PY{n}{init}
         \PY{k+kn}{from} \PY{n+nn}{colorama} \PY{k}{import} \PY{n}{Fore}\PY{p}{,} \PY{n}{Back}
\end{Verbatim}

    Neben dem Import der benötigten Module werden Konstanten definiert, die
im weiteren Verlauf der Visualisierung mehrfach benötigt werden.
Hierunter fällt die Definition der Hintergrund- und Schriftfarben für
\texttt{sty} und \texttt{colorama} und eine Liste, die alle benötigten
Buchstaben eines Schachbretts enthält, die zur Benennung der X-Achse
benötigt werden. Ebenfalls werden drei Konstanten definiert, die eine
Anweisung für den Nutzer enthalten und zu einem passenden Zeitpunkt
ausgegeben werden, wie bspw. die Eingabe des zu verwendenden Zuges des
Nutzers.

    \begin{Verbatim}[commandchars=\\\{\}]
{\color{incolor}In [{\color{incolor}43}]:} \PY{n}{FG\PYZus{}BLACK} \PY{o}{=} \PY{n}{fg}\PY{p}{(}\PY{l+m+mi}{0}\PY{p}{,} \PY{l+m+mi}{0}\PY{p}{,} \PY{l+m+mi}{0}\PY{p}{)}
         \PY{n}{FG\PYZus{}WHITE} \PY{o}{=} \PY{n}{fg}\PY{p}{(}\PY{l+m+mi}{255}\PY{p}{,} \PY{l+m+mi}{255}\PY{p}{,}\PY{l+m+mi}{255}\PY{p}{)}
         \PY{n}{BG\PYZus{}BLACK} \PY{o}{=} \PY{n}{bg}\PY{p}{(}\PY{l+m+mi}{222}\PY{p}{,} \PY{l+m+mi}{184}\PY{p}{,} \PY{l+m+mi}{135}\PY{p}{)}
         \PY{n}{BG\PYZus{}WHITE} \PY{o}{=} \PY{n}{bg}\PY{p}{(}\PY{l+m+mi}{211}\PY{p}{,} \PY{l+m+mi}{211}\PY{p}{,} \PY{l+m+mi}{211}\PY{p}{)}
         
         \PY{n}{FG\PYZus{}BLACK\PYZus{}WIN} \PY{o}{=} \PY{n}{Fore}\PY{o}{.}\PY{n}{BLACK}
         \PY{n}{FG\PYZus{}WHITE\PYZus{}WIN} \PY{o}{=} \PY{n}{Fore}\PY{o}{.}\PY{n}{RED}
         \PY{n}{BG\PYZus{}BLACK\PYZus{}WIN} \PY{o}{=} \PY{n}{Back}\PY{o}{.}\PY{n}{YELLOW}
         \PY{n}{BG\PYZus{}WHITE\PYZus{}WIN} \PY{o}{=} \PY{n}{Back}\PY{o}{.}\PY{n}{GREEN}
         
         \PY{n}{NUM\PYZus{}TO\PYZus{}ALPHABET} \PY{o}{=} \PY{p}{[}\PY{l+s+s2}{\PYZdq{}}\PY{l+s+s2}{ }\PY{l+s+s2}{\PYZdq{}}\PY{p}{,}\PY{l+s+s2}{\PYZdq{}}\PY{l+s+s2}{A}\PY{l+s+s2}{\PYZdq{}}\PY{p}{,} \PY{l+s+s2}{\PYZdq{}}\PY{l+s+s2}{B}\PY{l+s+s2}{\PYZdq{}}\PY{p}{,} \PY{l+s+s2}{\PYZdq{}}\PY{l+s+s2}{C}\PY{l+s+s2}{\PYZdq{}}\PY{p}{,} \PY{l+s+s2}{\PYZdq{}}\PY{l+s+s2}{D}\PY{l+s+s2}{\PYZdq{}}\PY{p}{,} \PY{l+s+s2}{\PYZdq{}}\PY{l+s+s2}{E}\PY{l+s+s2}{\PYZdq{}}\PY{p}{,} \PY{l+s+s2}{\PYZdq{}}\PY{l+s+s2}{F}\PY{l+s+s2}{\PYZdq{}}\PY{p}{,} \PY{l+s+s2}{\PYZdq{}}\PY{l+s+s2}{G}\PY{l+s+s2}{\PYZdq{}}\PY{p}{,} \PY{l+s+s2}{\PYZdq{}}\PY{l+s+s2}{H}\PY{l+s+s2}{\PYZdq{}}\PY{p}{]}
         
         \PY{n}{ASK\PYZus{}FOR\PYZus{}MOVE\PYZus{}MESSAGE} \PY{o}{=} \PY{l+s+s2}{\PYZdq{}}\PY{l+s+s2}{Possible Moves: }\PY{l+s+si}{\PYZob{}\PYZcb{}}\PY{l+s+se}{\PYZbs{}n}\PY{l+s+s2}{Enter your move: }\PY{l+s+s2}{\PYZdq{}}
         \PY{n}{WRONG\PYZus{}INPUT\PYZus{}MESSAGE} \PY{o}{=} \PY{l+s+s2}{\PYZdq{}}\PY{l+s+s2}{Given move not in legal moves. Please repeat}\PY{l+s+s2}{\PYZdq{}}
         \PY{n}{PLAYER\PYZus{}TURN\PYZus{}MESSAGE} \PY{o}{=} \PY{l+s+s2}{\PYZdq{}}\PY{l+s+se}{\PYZbs{}n}\PY{l+s+se}{\PYZbs{}n}\PY{l+s+s2}{It}\PY{l+s+s2}{\PYZsq{}}\PY{l+s+s2}{s }\PY{l+s+si}{\PYZob{}\PYZcb{}}\PY{l+s+s2}{\PYZsq{}}\PY{l+s+s2}{s turn: }\PY{l+s+s2}{\PYZdq{}}
\end{Verbatim}

    Damit die Schach-KI die Ausgabe je nach Betriebssystem ändern kann muss
erst ermittelt werden, ob der Nutzer ein Microsoft Produkt nutzt, oder
nicht. Dafür wird eine Funktion \texttt{os\_is\_windows} definiert, die
als Rückgabewert ein Boolean-Wert gibt, ob es sich um ein Windows
handelt, oder nicht.

    \begin{Verbatim}[commandchars=\\\{\}]
{\color{incolor}In [{\color{incolor}44}]:} \PY{k}{def} \PY{n+nf}{os\PYZus{}is\PYZus{}windows}\PY{p}{(}\PY{n+nb+bp}{self}\PY{p}{)}\PY{p}{:}
                 \PY{k}{return} \PY{p}{\PYZob{}}
                     \PY{l+s+s1}{\PYZsq{}}\PY{l+s+s1}{linux1}\PY{l+s+s1}{\PYZsq{}} \PY{p}{:} \PY{k+kc}{False}\PY{p}{,}
                     \PY{l+s+s1}{\PYZsq{}}\PY{l+s+s1}{linux2}\PY{l+s+s1}{\PYZsq{}} \PY{p}{:} \PY{k+kc}{False}\PY{p}{,}
                     \PY{l+s+s1}{\PYZsq{}}\PY{l+s+s1}{darwin}\PY{l+s+s1}{\PYZsq{}} \PY{p}{:} \PY{k+kc}{False}\PY{p}{,}
                     \PY{l+s+s1}{\PYZsq{}}\PY{l+s+s1}{win32}\PY{l+s+s1}{\PYZsq{}} \PY{p}{:} \PY{k+kc}{True}
                 \PY{p}{\PYZcb{}}\PY{p}{[}\PY{n}{sys}\PY{o}{.}\PY{n}{platform}\PY{p}{]}
\end{Verbatim}

    Verwendet wird die Funktion \texttt{os\_is\_windows} in
\texttt{ensure\_windows\_compability}, die die Aufgabe hat die
\texttt{colorama} Funktion \texttt{init()} zu verwenden, wenn es sich um
ein Microsoft Betriebssystem handelt. Die Funktion \texttt{init} bekommt
in diesem Fall den speziellen Übergabewert \texttt{autoreset=True}, der
dafür zuständig ist die Schrift- und Hintergrundfarbe nach Verwendung
von \texttt{colorama} auf die ursprüngliche zu stellen.

Die Funktion \texttt{ensure\_windows\_compability} gibt den
Boolean-Wert, ob es sich um ein Microsoft Betriebssystem handelt zurück.

    \begin{Verbatim}[commandchars=\\\{\}]
{\color{incolor}In [{\color{incolor}45}]:} \PY{k}{def} \PY{n+nf}{ensure\PYZus{}windows\PYZus{}compability}\PY{p}{(}\PY{n+nb+bp}{self}\PY{p}{)}\PY{p}{:}
                 \PY{n}{os\PYZus{}windows} \PY{o}{=} \PY{n+nb+bp}{self}\PY{o}{.}\PY{n}{os\PYZus{}is\PYZus{}windows}\PY{p}{(}\PY{p}{)}
                 \PY{k}{if} \PY{n}{os\PYZus{}windows}\PY{p}{:}
                     \PY{n}{init}\PY{p}{(}\PY{n}{autoreset}\PY{o}{=}\PY{k+kc}{True}\PY{p}{)}
                 \PY{k}{return} \PY{n}{os\PYZus{}windows}
\end{Verbatim}

    Die zuletzt eingeführten Funktionen werden von dem Programm im
Konstruktor aufgerufen und der Boolean-Wert der Variable
\texttt{need\_win\_support} zugewiesen. Diese Variable wird im späteren
Verlauf der Visualisierung benötigt, um entscheiden zu können, ob das
Modul \texttt{sty} oder \texttt{colorama} genutzt werden soll.

    Damit der Nutzer der Schach-KI schnell erkennen kann welche Figur sich
auf welchem Feld befindet wird in der Visualisierung auf der X-Achse die
Buchstaben A bis H angegeben. Zur Ausgabe dieser Zeichen wird die
Funktion \texttt{print\_alphabetical\_description} verwendet, die
mittels einer for-Schleife von 0 bis einschließlich 8 iteriert und
nacheinander die Zeichen aus der \texttt{NUM\_TO\_ALPHABET} Konstante
ausgibt.

Das erste Zeichen, das ausgegeben wird, ist ein Leerzeichen, da das
Schachbrett zur besseren Navigation neben den Buchstaben auch Zahlen
enthalten soll und somit diese erste Spalte nicht das Schachbrett,
sondern die Zahlen sind.

Die Python \texttt{print} Funktion wird während der gesamten Ausgabe des
Schachbretts mit zusätzlichen Parametern ausgestattet, die den Inhalt
zentrieren und diesem eine bestimmte Breite zuweisen. Ebenfalls ist es
in Python 3 Standard, dass nach der \texttt{print} Funktion eine neue
Zeile gestartet wird, dies wird mittels
\texttt{end=\textquotesingle{}\textquotesingle{}} umgangen, da das
Schachbrett nicht jedes Feld in einer neuen Zeile besitzen soll.

    \begin{Verbatim}[commandchars=\\\{\}]
{\color{incolor}In [{\color{incolor}46}]:} \PY{k}{def} \PY{n+nf}{print\PYZus{}alphabetical\PYZus{}description}\PY{p}{(}\PY{n+nb+bp}{self}\PY{p}{)}\PY{p}{:}
                 \PY{k}{for} \PY{n}{i} \PY{o+ow}{in} \PY{n+nb}{range}\PY{p}{(}\PY{l+m+mi}{0}\PY{p}{,}\PY{l+m+mi}{9}\PY{p}{)}\PY{p}{:}
                     \PY{n+nb}{print}\PY{p}{(}\PY{n}{NUM\PYZus{}TO\PYZus{}ALPHABET}\PY{p}{[}\PY{n}{i}\PY{p}{]}\PY{o}{.}\PY{n}{center}\PY{p}{(}\PY{l+m+mi}{3}\PY{p}{)}\PY{p}{,} \PY{n}{end}\PY{o}{=}\PY{l+s+s1}{\PYZsq{}}\PY{l+s+s1}{\PYZsq{}}\PY{p}{)}
                 \PY{n+nb}{print}\PY{p}{(}\PY{l+s+s2}{\PYZdq{}}\PY{l+s+se}{\PYZbs{}n}\PY{l+s+s2}{\PYZdq{}}\PY{p}{,} \PY{n}{end}\PY{o}{=}\PY{l+s+s1}{\PYZsq{}}\PY{l+s+s1}{\PYZsq{}}\PY{p}{)}
\end{Verbatim}

    Um das Schachbrett besser ausgeben und Farben zuordnen zu können wird es
in eine Matrix umgewandelt. Die Formatierung in eine Matrix wird in der
Funktion \texttt{create\_board\_matrix(self,\ board)} umgesetzt, die als
Übergabewert die aktuelle Spielsituation erhält.

Aus dem Schachbrett wird mittels der Funktion \texttt{fen()} die
aktuelle Schachsituation als Zeichenkette exportiert. Hierbei muss
darauf geachtet werden, dass nur der erste Teil der Zeichenkette
beachtet wird und nicht zusätzliche Informationen, wie bspw. die Partei,
die am Zug ist, exportiert wird.

Da das Board als Zeichenkette bspw. \texttt{8/2K5/4B3/3N4/8/8/4k3/8} so
aussehen kann und die Informationen einer Zeile durch ein \texttt{/}
voneinander getrennt sind, wird in einer for-Schleife pro Iteration die
Informationen bis zum nächsten \texttt{/} ausgelesen. Daraufhin wird für
jedes Element dieser Informationen überprüft, ob es sich um eine Zahl
oder einen Buchstaben handelt. Sollte es sich um eine Zahl handeln, dann
ist dies die Anzahl der freien Felder bis zum nächsten Zeichen oder der
nächsten Zeile. Für jede Zahl wird ein Leerzeichen pro Höhe der Zahl zu
einer Liste \texttt{line} hinzugefügt. Handelt es sich um keine Zahl,
sondern um einen Buchstaben, dann ist dies eine Figur und diese wird
ebenfalls zur Liste \texttt{line} hinzugefügt.

Nach jeder Zeile wird die Liste \texttt{line} zu einer weiteren Liste
\texttt{board\_matrix} hinzugefügt, die nach Abschluss der Funktion
\texttt{create\_board\_matrix} zurückgegeben wird.

    \begin{Verbatim}[commandchars=\\\{\}]
{\color{incolor}In [{\color{incolor}47}]:} \PY{k}{def} \PY{n+nf}{create\PYZus{}board\PYZus{}matrix}\PY{p}{(}\PY{n+nb+bp}{self}\PY{p}{,} \PY{n}{board}\PY{p}{)}\PY{p}{:}
                 \PY{n}{board\PYZus{}fen} \PY{o}{=} \PY{n}{board}\PY{o}{.}\PY{n}{fen}\PY{p}{(}\PY{p}{)}\PY{o}{.}\PY{n}{split}\PY{p}{(}\PY{l+s+s2}{\PYZdq{}}\PY{l+s+s2}{ }\PY{l+s+s2}{\PYZdq{}}\PY{p}{)}\PY{p}{[}\PY{l+m+mi}{0}\PY{p}{]}
                 \PY{n}{board\PYZus{}matrix} \PY{o}{=} \PY{p}{[}\PY{p}{]}
                 \PY{k}{for} \PY{n}{row} \PY{o+ow}{in} \PY{n}{board\PYZus{}fen}\PY{o}{.}\PY{n}{split}\PY{p}{(}\PY{l+s+s2}{\PYZdq{}}\PY{l+s+s2}{/}\PY{l+s+s2}{\PYZdq{}}\PY{p}{)}\PY{p}{:}
                     \PY{n}{line} \PY{o}{=} \PY{p}{[}\PY{p}{]}
                     \PY{k}{for} \PY{n}{character} \PY{o+ow}{in} \PY{n}{row}\PY{p}{:}
                         \PY{k}{if} \PY{n}{character}\PY{o}{.}\PY{n}{isdigit}\PY{p}{(}\PY{p}{)}\PY{p}{:}
                             \PY{k}{for} \PY{n}{empty} \PY{o+ow}{in} \PY{n+nb}{range}\PY{p}{(}\PY{n+nb}{int}\PY{p}{(}\PY{n}{character}\PY{p}{)}\PY{p}{)}\PY{p}{:}
                                 \PY{n}{line}\PY{o}{.}\PY{n}{append}\PY{p}{(}\PY{l+s+s2}{\PYZdq{}}\PY{l+s+s2}{ }\PY{l+s+s2}{\PYZdq{}}\PY{p}{)}
                         \PY{k}{else}\PY{p}{:} 
                             \PY{n}{line}\PY{o}{.}\PY{n}{append}\PY{p}{(}\PY{n}{character}\PY{p}{)}
                     \PY{n}{board\PYZus{}matrix}\PY{o}{.}\PY{n}{append}\PY{p}{(}\PY{n}{line}\PY{p}{)}
                 \PY{k}{return} \PY{n}{board\PYZus{}matrix}
\end{Verbatim}

    Zur optimalen Visualisierung des Schachbretts können die Buchstaben, die
eine Figur beschreiben gewandelt werden. Für
diese Umsetzung wird die Funktion \texttt{piece\_switcher(self,\ piece)}
benötigt, die anhand der übergebenen Buchstaben die Figur als
Unicode-Zeichen zurückgibt.

    \begin{Verbatim}[commandchars=\\\{\}]
{\color{incolor}In [{\color{incolor}48}]:} \PY{k}{def} \PY{n+nf}{piece\PYZus{}switcher}\PY{p}{(}\PY{n+nb+bp}{self}\PY{p}{,} \PY{n}{piece}\PY{p}{)}\PY{p}{:}
                 \PY{k}{return} \PY{p}{\PYZob{}}
                     \PY{l+s+s2}{\PYZdq{}}\PY{l+s+s2}{K}\PY{l+s+s2}{\PYZdq{}}\PY{p}{:} \PY{l+s+sa}{u}\PY{l+s+s1}{\PYZsq{}}\PY{l+s+se}{\PYZbs{}u2654}\PY{l+s+s1}{\PYZsq{}}\PY{p}{,}
                     \PY{l+s+s2}{\PYZdq{}}\PY{l+s+s2}{Q}\PY{l+s+s2}{\PYZdq{}}\PY{p}{:} \PY{l+s+sa}{u}\PY{l+s+s1}{\PYZsq{}}\PY{l+s+se}{\PYZbs{}u2655}\PY{l+s+s1}{\PYZsq{}}\PY{p}{,}
                     \PY{l+s+s2}{\PYZdq{}}\PY{l+s+s2}{R}\PY{l+s+s2}{\PYZdq{}}\PY{p}{:} \PY{l+s+sa}{u}\PY{l+s+s1}{\PYZsq{}}\PY{l+s+se}{\PYZbs{}u2656}\PY{l+s+s1}{\PYZsq{}}\PY{p}{,}
                     \PY{l+s+s2}{\PYZdq{}}\PY{l+s+s2}{B}\PY{l+s+s2}{\PYZdq{}}\PY{p}{:} \PY{l+s+sa}{u}\PY{l+s+s1}{\PYZsq{}}\PY{l+s+se}{\PYZbs{}u2657}\PY{l+s+s1}{\PYZsq{}}\PY{p}{,}
                     \PY{l+s+s2}{\PYZdq{}}\PY{l+s+s2}{N}\PY{l+s+s2}{\PYZdq{}}\PY{p}{:} \PY{l+s+sa}{u}\PY{l+s+s1}{\PYZsq{}}\PY{l+s+se}{\PYZbs{}u2658}\PY{l+s+s1}{\PYZsq{}}\PY{p}{,}
                     \PY{l+s+s2}{\PYZdq{}}\PY{l+s+s2}{P}\PY{l+s+s2}{\PYZdq{}}\PY{p}{:} \PY{l+s+sa}{u}\PY{l+s+s1}{\PYZsq{}}\PY{l+s+se}{\PYZbs{}u2659}\PY{l+s+s1}{\PYZsq{}}\PY{p}{,}
                     \PY{l+s+s2}{\PYZdq{}}\PY{l+s+s2}{k}\PY{l+s+s2}{\PYZdq{}}\PY{p}{:} \PY{l+s+sa}{u}\PY{l+s+s1}{\PYZsq{}}\PY{l+s+se}{\PYZbs{}u265A}\PY{l+s+s1}{\PYZsq{}}\PY{p}{,}
                     \PY{l+s+s2}{\PYZdq{}}\PY{l+s+s2}{q}\PY{l+s+s2}{\PYZdq{}}\PY{p}{:} \PY{l+s+sa}{u}\PY{l+s+s1}{\PYZsq{}}\PY{l+s+se}{\PYZbs{}u265B}\PY{l+s+s1}{\PYZsq{}}\PY{p}{,}
                     \PY{l+s+s2}{\PYZdq{}}\PY{l+s+s2}{r}\PY{l+s+s2}{\PYZdq{}}\PY{p}{:} \PY{l+s+sa}{u}\PY{l+s+s1}{\PYZsq{}}\PY{l+s+se}{\PYZbs{}u265C}\PY{l+s+s1}{\PYZsq{}}\PY{p}{,}
                     \PY{l+s+s2}{\PYZdq{}}\PY{l+s+s2}{b}\PY{l+s+s2}{\PYZdq{}}\PY{p}{:} \PY{l+s+sa}{u}\PY{l+s+s1}{\PYZsq{}}\PY{l+s+se}{\PYZbs{}u265D}\PY{l+s+s1}{\PYZsq{}}\PY{p}{,}
                     \PY{l+s+s2}{\PYZdq{}}\PY{l+s+s2}{n}\PY{l+s+s2}{\PYZdq{}}\PY{p}{:} \PY{l+s+sa}{u}\PY{l+s+s1}{\PYZsq{}}\PY{l+s+se}{\PYZbs{}u265E}\PY{l+s+s1}{\PYZsq{}}\PY{p}{,}
                     \PY{l+s+s2}{\PYZdq{}}\PY{l+s+s2}{p}\PY{l+s+s2}{\PYZdq{}}\PY{p}{:} \PY{l+s+sa}{u}\PY{l+s+s1}{\PYZsq{}}\PY{l+s+se}{\PYZbs{}u265F}\PY{l+s+s1}{\PYZsq{}}
                 \PY{p}{\PYZcb{}}\PY{o}{.}\PY{n}{get}\PY{p}{(}\PY{n}{piece}\PY{p}{,} \PY{n}{piece}\PY{p}{)}
\end{Verbatim}

    Für die Ausgabe eines Feldes und dessen Farben ist die Funktion
\texttt{create\_piece(self,\ character,\ field\_is\_dark)} zuständig,
die die Schachfigur als Buchstaben und einen booleschen Wert, ob das zu
erzeugenden Feld dunkel oder hell sein muss, übergeben bekommt.

Zu Beginn wird mittels des Buchstabens das passende Zeichen ausgewählt
und je nachdem, ob der Buchstaben groß oder klein geschrieben ist eine
andere Schriftfarbe zugeordnet. Die weißen Figuren werden als Groß- und
die schwarzen als Kleinbuchstaben geschrieben.

Daraufhin wird der Figur je nach dem Wert der Variable
\texttt{field\_is\_dark} eine Hintergrundfarbe zugewiesen und dies
zurückgegeben.

Da jedoch Microsoft Windows kein Unicode in Konsolen unterstützt, wird
bei Nutzern dieses Betriebssystems nur die Buchstaben ausgegeben. Die
Implementierung dieser Windows-Spezifikation wird innerhalb der Funktion
\texttt{create\_piece\_win(self,\ character,\ field\_is\_dark)}
umgesetzt, die analog zu der vorgestellten ist, jedoch sich in dem Punkt
unterscheidet, dass das Modul \texttt{colorama} verwendet wird und kein
Unicode-Zeichen ausgegeben wird, sondern der übergebene Buchstabe.

    \begin{Verbatim}[commandchars=\\\{\}]
{\color{incolor}In [{\color{incolor}49}]:} \PY{k}{def} \PY{n+nf}{create\PYZus{}piece}\PY{p}{(}\PY{n+nb+bp}{self}\PY{p}{,} \PY{n}{character}\PY{p}{,} \PY{n}{field\PYZus{}is\PYZus{}dark}\PY{p}{)}\PY{p}{:}
                 \PY{n}{chess\PYZus{}piece} \PY{o}{=} \PY{n+nb}{str}\PY{p}{(}\PY{n+nb+bp}{self}\PY{o}{.}\PY{n}{piece\PYZus{}switcher}\PY{p}{(}\PY{n}{character}\PY{p}{)}\PY{p}{)}
                 \PY{n}{chess\PYZus{}piece\PYZus{}color} \PY{o}{=} \PY{n}{FG\PYZus{}BLACK} \PY{k}{if} \PY{n}{character}\PY{o}{.}\PY{n}{isupper}\PY{p}{(}\PY{p}{)} 
			\PY{k}{else} \PY{n}{FG\PYZus{}WHITE}
                 \PY{n}{colored\PYZus{}chess\PYZus{}piece} \PY{o}{=} \PY{n}{chess\PYZus{}piece\PYZus{}color} \PY{o}{+} \PY{n}{chess\PYZus{}piece}\PY{o}{.}\PY{n}{center}\PY{p}{(}\PY{l+m+mi}{3}\PY{p}{)} 
			\PY{o}{+} \PY{n}{fg}\PY{o}{.}\PY{n}{rs}
                 \PY{n}{background\PYZus{}color} \PY{o}{=} \PY{n}{BG\PYZus{}BLACK} \PY{k}{if} \PY{n}{field\PYZus{}is\PYZus{}dark} \PY{o+ow}{is} \PY{k+kc}{False} 
			\PY{k}{else} \PY{n}{BG\PYZus{}WHITE}
                 \PY{n}{field} \PY{o}{=} \PY{n}{background\PYZus{}color} \PY{o}{+} \PY{n}{colored\PYZus{}chess\PYZus{}piece} \PY{o}{+} \PY{n}{bg}\PY{o}{.}\PY{n}{rs}
                 \PY{k}{return} \PY{n}{field}
\end{Verbatim}

    Zur Koordination dieser vorgestellten Funktionen und somit der Ausgabe
des Schachbretts wird die Funktion
\texttt{print\_board(self,\ player\_name,\ board)} verwendet. Dabei wird
zuerst die Konstante \texttt{PLAYER\_TURN\_MESSAGE} mit dem passenden
Spielername ausgegeben und daraufhin die Matrix des Schachbretts
berechnet.

Daraufhin werden die Buchstaben der x-Achse des Schachbretts ausgegeben
und darauf eine for-Schleife verwendet, die aus der Anzahl der
vorhandenen Spalten iteriert. Innerhalb dieser Schleife wird jeweils zu
Beginn einer Zeile die aktuelle y-Achsen Zahl ausgegeben und daraufhin
eine neue for-Schleife gestartet, die berechnet, ob das Feld dunkel oder
hell sein muss. Ebenfalls wird in dieser zweiten for-Schleife Gebrauch
von der Funktion \texttt{create\_piece} gemacht und die Schachfigur
ausgegeben. Am Ende jeder Zeile wird, analog zum Beginn einer Zeile, die
y-Achsen Zahl ausgegeben.

Nachdem alle Zeilen ausgegeben wurden, wird erneut die x-Achsen
Beschreibung der Konsole übermittelt.

    \begin{Verbatim}[commandchars=\\\{\}]
{\color{incolor}In [{\color{incolor}51}]:} \PY{k}{def} \PY{n+nf}{print\PYZus{}board}\PY{p}{(}\PY{n+nb+bp}{self}\PY{p}{,} \PY{n}{player\PYZus{}name}\PY{p}{,} \PY{n}{board}\PY{p}{)}\PY{p}{:}
                 \PY{n+nb}{super}\PY{p}{(}\PY{p}{)}\PY{o}{.}\PY{n}{print\PYZus{}board}\PY{p}{(}\PY{n}{player\PYZus{}name}\PY{p}{,} \PY{n}{board}\PY{p}{)}
                 \PY{n+nb}{print}\PY{p}{(}\PY{n}{PLAYER\PYZus{}TURN\PYZus{}MESSAGE}\PY{o}{.}\PY{n}{format}\PY{p}{(}\PY{n}{player\PYZus{}name}\PY{p}{)}\PY{p}{)}
                 \PY{n}{board\PYZus{}matrix} \PY{o}{=} \PY{n+nb+bp}{self}\PY{o}{.}\PY{n}{create\PYZus{}board\PYZus{}matrix}\PY{p}{(}\PY{n}{board}\PY{p}{)}
                 \PY{n+nb+bp}{self}\PY{o}{.}\PY{n}{print\PYZus{}alphabetical\PYZus{}description}\PY{p}{(}\PY{p}{)}
                 \PY{k}{for} \PY{n}{row\PYZus{}index}\PY{p}{,} \PY{n}{row} \PY{o+ow}{in} \PY{n+nb}{enumerate}\PY{p}{(}\PY{n}{board\PYZus{}matrix}\PY{p}{)}\PY{p}{:}
                     \PY{n+nb}{print}\PY{p}{(}\PY{p}{(}\PY{n+nb}{str}\PY{p}{(}\PY{l+m+mi}{8} \PY{o}{\PYZhy{}} \PY{n}{row\PYZus{}index}\PY{p}{)}\PY{p}{)}\PY{o}{.}\PY{n}{center}\PY{p}{(}\PY{l+m+mi}{3}\PY{p}{)}\PY{p}{,} \PY{n}{end}\PY{o}{=}\PY{l+s+s2}{\PYZdq{}}\PY{l+s+s2}{\PYZdq{}}\PY{p}{)}
                     \PY{k}{for} \PY{n}{field\PYZus{}index}\PY{p}{,} \PY{n}{field} \PY{o+ow}{in} \PY{n+nb}{enumerate}\PY{p}{(}\PY{n}{row}\PY{p}{)}\PY{p}{:}
                         \PY{n}{field\PYZus{}is\PYZus{}dark} \PY{o}{=} \PY{n+nb}{bool}\PY{p}{(}\PY{p}{(}\PY{n}{field\PYZus{}index} \PY{o}{+} \PY{n}{row\PYZus{}index}\PY{p}{)}\PY{o}{\PYZpc{}} \PY{l+m+mi}{2}\PY{p}{)}
                         \PY{n}{colored\PYZus{}field} \PY{o}{=} \PY{n+nb+bp}{self}\PY{o}{.}\PY{n}{create\PYZus{}piece}\PY{p}{(}\PY{n}{field}\PY{p}{,} \PY{n}{field\PYZus{}is\PYZus{}dark}\PY{p}{)} 
				\PY{k}{if} \PY{o+ow}{not} \PY{n+nb+bp}{self}\PY{o}{.}\PY{n}{need\PYZus{}win\PYZus{}support} 
				\PY{k}{else} \PY{n+nb+bp}{self}\PY{o}{.}\PY{n}{create\PYZus{}piece\PYZus{}win}\PY{p}{(}\PY{n}{field}\PY{p}{,} \PY{n}{field\PYZus{}is\PYZus{}dark}\PY{p}{)}
                         \PY{n+nb}{print}\PY{p}{(}\PY{n}{colored\PYZus{}field}\PY{p}{,} \PY{n}{end}\PY{o}{=}\PY{l+s+s1}{\PYZsq{}}\PY{l+s+s1}{\PYZsq{}}\PY{p}{)}
                     \PY{n+nb}{print}\PY{p}{(}\PY{p}{(}\PY{n+nb}{str}\PY{p}{(}\PY{l+m+mi}{8} \PY{o}{\PYZhy{}} \PY{n}{row\PYZus{}index}\PY{p}{)}\PY{p}{)}\PY{o}{.}\PY{n}{center}\PY{p}{(}\PY{l+m+mi}{3}\PY{p}{)}\PY{p}{)}
                 \PY{n+nb+bp}{self}\PY{o}{.}\PY{n}{print\PYZus{}alphabetical\PYZus{}description}\PY{p}{(}\PY{p}{)}
\end{Verbatim}


\hypertarget{starten-des-spiels-im-jupyter-notebook}{%
\section{Starten des Spiels im Jupyter
Notebook}\label{starten-des-spiels-im-jupyter-notebook}}

Um das Spiel auch im Jupyter Notebook starten zu können ist hier eine
vereinfachte Variante des \texttt{chess\_master} vorzufinden. Dabei
werden keine Spieler dynamisch erstellt, sondern abwechselnd der Nutzer und die KI nach einem Zug gefragt.

Zum Ausgeben des aktuellen Zustands ist hier außerdem noch eine Funktion zu sehen, mittels der die Schachbretter als SVG im Jupyter Notebook ausgegeben werden können.

    \begin{Verbatim}[commandchars=\\\{\}]
{\color{incolor}In [{\color{incolor}54}]:} \PY{k+kn}{from} \PY{n+nn}{IPython}\PY{n+nn}{.}\PY{n+nn}{display} \PY{k}{import} \PY{n}{SVG}\PY{p}{,} \PY{n}{display}
         
         \PY{k}{def} \PY{n+nf}{print\PYZus{}board\PYZus{}svg}\PY{p}{(}\PY{n}{board}\PY{p}{)}\PY{p}{:}
             \PY{n}{display}\PY{p}{(}\PY{n}{SVG}\PY{p}{(}\PY{n}{chess}\PY{o}{.}\PY{n}{svg}\PY{o}{.}\PY{n}{board}\PY{p}{(}\PY{n}{board}\PY{o}{=}\PY{n}{board}\PY{p}{)}\PY{p}{)}\PY{p}{)}
\end{Verbatim}

    Im vereinfachten Verwalter des Spiels wird auf ein Anlegen und Verwalten der Spielhistorie verzichtet. Stattdessen wird nach Erstellen des Spiels zunächst der Spieler nach seinem nächsten Zug gefragt und im nächsten Zug die KI aufgefordert einen zu berechnen. Danach wird der Zug zu dem board hinzugefügt und im nächsten Durchlauf wieder ausgegeben. Dies wird solange wiederholt bis das Spiel vorüber ist. Dazu sind zunächst noch alle nötigen Importe für das Spiel aufzulisten.

    \begin{Verbatim}[commandchars=\\\{\}]
{\color{incolor}In [{\color{incolor} }]:} \PY{k+kn}{import} \PY{n+nn}{pandas} \PY{k}{as} \PY{n+nn}{pd}
        \PY{k+kn}{import} \PY{n+nn}{misc}\PY{n+nn}{.}\PY{n+nn}{tools} \PY{k}{as} \PY{n+nn}{Tools}
        \PY{k+kn}{import} \PY{n+nn}{chess}
        \PY{k+kn}{import} \PY{n+nn}{chess}\PY{n+nn}{.}\PY{n+nn}{svg}
        \PY{k+kn}{import} \PY{n+nn}{chess}\PY{n+nn}{.}\PY{n+nn}{polyglot}
        \PY{k+kn}{import} \PY{n+nn}{os}
        \PY{k+kn}{import} \PY{n+nn}{numpy} \PY{k}{as} \PY{n+nn}{np}
        \PY{k+kn}{import} \PY{n+nn}{time}
        
        \PY{n}{board} \PY{o}{=} \PY{n}{chess}\PY{o}{.}\PY{n}{Board}\PY{p}{(}\PY{p}{)}
        \PY{k}{while} \PY{o+ow}{not} \PY{n}{board}\PY{o}{.}\PY{n}{is\PYZus{}game\PYZus{}over}\PY{p}{(}\PY{p}{)}\PY{p}{:}
            \PY{n}{print\PYZus{}board\PYZus{}svg}\PY{p}{(}\PY{n}{board}\PY{p}{)}
            \PY{k}{if} \PY{n}{board}\PY{o}{.}\PY{n}{turn}\PY{p}{:}
                \PY{n}{legal\PYZus{}moves} \PY{o}{=} \PY{n+nb}{list}\PY{p}{(}\PY{n+nb}{map}\PY{p}{(}\PY{n}{board}\PY{o}{.}\PY{n}{uci}\PY{p}{,} \PY{n}{board}\PY{o}{.}\PY{n}{legal\PYZus{}moves}\PY{p}{)}\PY{p}{)}
                \PY{n}{move} \PY{o}{=} \PY{k+kc}{None}
                \PY{k}{while} \PY{n}{move} \PY{o+ow}{not} \PY{o+ow}{in} \PY{n}{legal\PYZus{}moves}\PY{p}{:}
                    \PY{n+nb}{print}\PY{p}{(}\PY{l+s+s2}{\PYZdq{}}\PY{l+s+s2}{Legal moves:}\PY{l+s+s2}{\PYZdq{}}\PY{p}{)}
                    \PY{n+nb}{print}\PY{p}{(}\PY{n}{legal\PYZus{}moves}\PY{p}{)}
                    \PY{n}{move} \PY{o}{=} \PY{n+nb}{input}\PY{p}{(}\PY{l+s+s2}{\PYZdq{}}\PY{l+s+s2}{Please enter your move: }\PY{l+s+s2}{\PYZdq{}}\PY{p}{)}
                \PY{n}{move} \PY{o}{=} \PY{n}{chess}\PY{o}{.}\PY{n}{Move}\PY{o}{.}\PY{n}{from\PYZus{}uci}\PY{p}{(}\PY{n}{move}\PY{p}{)}
            \PY{k}{else}\PY{p}{:}
                \PY{n}{move} \PY{o}{=} \PY{n}{get\PYZus{}move}\PY{p}{(}\PY{n}{board}\PY{p}{)}
        
            \PY{n}{board}\PY{o}{.}\PY{n}{push}\PY{p}{(}\PY{n}{move}\PY{p}{)}
        
        \PY{n}{result} \PY{o}{=} \PY{n}{Tools}\PY{o}{.}\PY{n}{get\PYZus{}board\PYZus{}result}\PY{p}{(}\PY{n}{board}\PY{p}{)}
        \PY{k}{if} \PY{n}{result} \PY{o+ow}{is} \PY{l+m+mi}{1}\PY{p}{:}
            \PY{n+nb}{print}\PY{p}{(}\PY{l+s+s2}{\PYZdq{}}\PY{l+s+si}{\PYZob{}\PYZcb{}}\PY{l+s+s2}{ (White) has won}\PY{l+s+s2}{\PYZdq{}}\PY{o}{.}\PY{n}{format}\PY{p}{(}\PY{n}{players}\PY{p}{[}\PY{l+m+mi}{0}\PY{p}{]}\PY{o}{.}\PY{n}{name}\PY{p}{)}\PY{p}{)}
        \PY{k}{elif} \PY{n}{result} \PY{o+ow}{is} \PY{o}{\PYZhy{}}\PY{l+m+mi}{1}\PY{p}{:}
            \PY{n+nb}{print}\PY{p}{(}\PY{l+s+s2}{\PYZdq{}}\PY{l+s+si}{\PYZob{}\PYZcb{}}\PY{l+s+s2}{ (Black) has won}\PY{l+s+s2}{\PYZdq{}}\PY{o}{.}\PY{n}{format}\PY{p}{(}\PY{n}{players}\PY{p}{[}\PY{l+m+mi}{1}\PY{p}{]}\PY{o}{.}\PY{n}{name}\PY{p}{)}\PY{p}{)}
        \PY{k}{else}\PY{p}{:}
            \PY{n+nb}{print}\PY{p}{(}\PY{l+s+s2}{\PYZdq{}}\PY{l+s+s2}{Draw}\PY{l+s+s2}{\PYZdq{}}\PY{p}{)}
            
\end{Verbatim}

Dies ermöglicht ein Spielen des Spiels auch direkt im Jupyter Notebook, wenn auch mit eingeschränkten Funktionalitäten. Die KI jedoch ist voll funktionsfähig.
