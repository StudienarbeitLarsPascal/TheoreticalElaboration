%!TEX root = ../dokumentation.tex

\chapter{Einleitung}
Diese Einleitung dient dazu, in den folgenden Unterkapiteln den Einstieg in das Thema Künstliche Intelligenz und deren Nutzen zu finden. Dabei wird auf die Definition und Eigenschaften eingegangen, sowie die Historie und das Thema der Spieletheorie aufgegriffen. Passend zu der vorgestellten Problematik wird der Bogen zur prüfenden Forschungsfrage gespannt.

\section{Hinführung zum Thema}
Künstliche Intelligenz und Machine Learning sind zwei Fachbegriffe der Informatik, die im täglichen Leben immer häufiger auftreten. Besonders in den letzten Jahren tritt Künstliche Intelligenz (\acs{KI}), oder Artificial Intelligence, in allen Lebenslagen häufiger auf und selbst ein Informatik-Laie kommt nicht darum herum, mit diesem Begriff konfrontiert zu werden. 

Allgemein wird mit Artificial Intelligence versucht, die Intelligenz eines Menschens nachzubilden und diese auf einen Computer projizieren zu können.

Auch wenn es auf Grund der hohen Medienpräsenz dieses Themas in den letzten Jahren so erscheinen mag, ist es keine technische Errungenschaft der letzten Jahre, sondern kann bereits eine lange Historie aufweisen. Bereits seit der Entwicklung der ersten Computer besteht der Wunsch, einem Computer eine eigene Intelligenz zu ermöglichen.

Dabei durchlief KI durch technische Erneuerungen bereits häufiger Hochphasen, während denen das Thema in der Öffentlichkeit eine größere Aufmerksamkeit genoss. Ein Beispiel dafür ist das Programm ELIZA das gegen Ende der 1960er Jahre von Joseph Weizenbaum entwickelt wurde und in dem der Dialog zwischen einem Patienten und dessen Psychotherapeuten simuliert wird \cite{Weizenbaum1966}.

% Zitat ELIZA einfügen

Die neuste Hochphase wurde vor allem durch das Maschinelle Lernen und neuronale Netze ausgelöst. Jedoch ist ein nicht zu verachtender Aspekt, der der Künstlichen Intelligenz zu einer erneuten Hochphase verhilft, die Leistungsfähigkeit und der geringere Preis von Rechenleistung. 

Hierbei unterteilt sich die Künstliche Intelligenz in viele unterschiedliche Teilgebiete, wie beispielsweise Robotik oder Machine Learning. Ein weiteres Anwendungsgebiet der Künstlichen Intelligenz ist die sogenannte Spieltheorie, bei der ein Computer die Regeln eines Spiels erlernt um daraufhin als Gegner dienen zu können. Die Spieltheorie ist ein wesentlicher Bestandteil dieser wissenschaftlichen Arbeit und wird somit im kommenden Kapitel über den Zweck und das Ziel dieser Arbeit aufgegriffen.

Bereits in den 1950er Jahren befassten sich einige Informatiker mit der Spieltheorie und im speziellen mit dem Spiel Dame. Bereits 1950 wurde von Claude Shannon eine wissenschaftliche Arbeit verfasst, die sich mit dem Thema zum Brettspiel Schach befasst und die den Titel ``Programming a Computer for Playing Chess`` trägt \cite{Shannon1950}. Schach bringt dabei die Herausforderung mit sich, dass es ein Brettspiel ist, das über einige Regeln verfügt und durch die verschiedenen Figuren und deren möglichen Anordnungen eine gewisse Komplexität mit sich bringt. Durch diese für den Menschen große Komplexität und Schwierigkeit, viele Züge in die Tiefe vorhersehen zu können, entstand das Verlangen dies durch eine Künstliche Intelligenz zu realisieren.

% Quelle Shannon: https://vision.unipv.it/IA1/ProgrammingaComputerforPlayingChess.pdf

\section{Zweck und Ziel dieser Arbeit}\label{zweck_und_ziel}
Diese wissenschaftliche Arbeit handelt von der Umsetzung eines Programms, das auf der Basis von KI es ermöglicht als Gegner im Spiel Schach zu dienen. Neben der Implementierung der Schach-KI wird diese Arbeit das nötige Grundwissen vermitteln, das benötigt wird um die in der Umsetzung verwendeten Techniken und Algorithmen zu verstehen. 

Hierbei wird ein besonderer Fokus auf den Minimax Algorithmus und die Alpha-Beta-Suche gelegt. Ebenfalls wird eine Einführung in die Geschichte der Spieltheorie gegeben und die Problematik komplexer Spiele aufgezeigt. Im weiteren Verlauf dieser Arbeit wird die zur Implementierung zu verwendende Bibliothek evaluiert und deren Vor- und Nachteile aufgeschlüsselt, sowie ein Einblick in die geplante Architektur der Schach-KI gegeben.

Neben den theoretischen Hintergründen und der technischen Grundlage wird ein großer Teil der Arbeit die Implementierung und die Erläuterung der Umsetzung sein. Dabei wird auf die wichtigsten Aspekte und Algorithmen des Programms eingegangen und diese passend zu dem zuvor vermittelten Wissen behandelt. 

Zum Abschluss der Arbeit wird eine Evaluation folgen, die einen Aufschluss darüber gibt, ob die gesetzten Ziele erfüllt wurden und sich ebenfalls kritisch mit der Leistung der entwickelten Künstlichen Intelligenz auseinandersetzt. Teil der Evaluierung ist dabei das Spezifizieren der benötigten Faktoren der Funktionen, die zum Finden des optimalen Zuges verwendet werden. Dies führt zu der mit den vorliegenden Mitteln bestmöglichen Künstlichen Intelligenz.


% Was ist KI
% Was ist der Sinn dieser Arbeit und was soll entwickelt werden?

% Und zusätzlich die Entwicklung der Spiele-Spielende-Computer eingegangen
% Spieletheorie
% Algorithmen zur Zugermittlung

% Kapitel 3: Python-Chess Library Evaluierung
% Welche Library ist dafür einsetzbar und welche Anwendungsfälle werden davon abgedeckt? Bzw. welche nicht

% Kapitel 4: Praktische Umsetzung/Implementierung

% Kapitel 5: Kriterienerfüllung
% Faktkoren für Evaluierungsfunktionen ausloten
