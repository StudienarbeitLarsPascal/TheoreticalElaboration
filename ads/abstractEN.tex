%!TEX root = ../dokumentation.tex

\pagestyle{plain}

\iflang{de}{%
\addchap{\langabstractEN} % Text für Überschrift
Künstliche Intelligenz ist zurzeit ein aktuelles Thema, das sowohl in den Medien als auch im täglichen Leben anzutreffen ist. Die Entwicklung der Künstlichen Intelligenz hat bereits durch einige ausschlaggebende Faktoren, wie beispielsweise die (Weiter-) Entwicklung von neuer Hard- und Software, mehrere Hochphasen erlebt. Doch die ersten Schritte in diesem Gebiet wurden bereits kurz nach der Entwicklung der ersten Computer durchgeführt, da bereits seit geraumer Zeit der Wunsch besteht, einem Computer eine eigene Intelligenz zu ermöglichen.

Wichtige Bestandteile von Künstlicher Intelligenz wurde durch die Forschung an der Spieltheorie entdeckt. Bei diesem Fachgebiet geht es darum, ein Computer als Gegenspieler in diversen Brett- und Gesellschaftsspielen zu verwenden. Ein traditionelles Beispiel nimmt dabei das Spiel Schach ein, das sich durch eine große Menge an verschiedenen Zugmöglichkeiten, jedoch durch eine überschaubare Anzahl von Regeln auszeichnet.

Diese wissenschaftliche Arbeit behandelt neben der Entwicklung einer Künstlichen Schach Intelligenz eine Hinführung zum Thema der Künstlichen Spieleintelligenzen. Ebenfalls wird die Theorie dieser erläutert, bei der ein wichtiges Augenmerk auf der Geschichte als auch auf den benötigten Algorithmen zur Realisierung der Künstliche Intelligenz liegen wird. Die wichtigen Algorithmen, die detailliert erläutert werden, sind der Iterative Deepening und der Minimax Algorithmus. Diese nehmen im weiteren Verlauf der wissenschaftlichen Arbeit bei der Implementierung eine zentrale Rolle ein.

Neben der Erläuterung der Theorie und der wichtigsten Implementierungsschritte wird ebenfalls eine Evaluierung der entwickelten Künstlichen Intelligenz stattfinden und betrachtet werden, welche Schritte in Zukunft realisiert werden könnten und sollten.
}