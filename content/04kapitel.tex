%!TEX root = ../dokumentation.tex

\chapter{Implementation}

\section{Erstellen des Spiels und der
Spieler}\label{erstellen-des-spiels-und-der-spieler}

Vor dem Start eines Schachspiels müssen zunächst Spiel generell sowie
die partizpierenden Spieler der gewünschten Spielertypen erstellt
werden. Dazu muss der Spieler dem Programm entweder Parameter mitgeben,
die das Spiel direkt mit den gewünschten Spielern startet, oder aber der
Spieler wird zum Programmstart nach gewünschten Einstellungen gefragt.

In beiden Fällen wird anschließend eine Liste an Spielern an Hand der
gegebenen Funktionen erstellt, mittels welcher Das Spiel dann
stattfinden kann. Dies findet in der Funktion
\texttt{interrogate\_settings} statt.

Wird das Spiel mit Parametern gestartet werden diese mit in die
\texttt{interrogate\_settings} Funktion übermittelt. Andernfalls werden
alle fehlenden Informationen in der Funktion über die
Nutzerschnittstelle abgefragt. Dies sieht wie folgt aus:

    \begin{Verbatim}[commandchars=\\\{\}]
{\color{incolor}In [{\color{incolor} }]:} \PY{k}{def} \PY{n+nf}{interrogate\PYZus{}settings}\PY{p}{(}\PY{n+nb+bp}{self}\PY{p}{,} \PY{n}{player\PYZus{}names}\PY{o}{=}\PY{k+kc}{None}\PY{p}{,} \PY{n}{player\PYZus{}types}\PY{o}{=}\PY{k+kc}{None}\PY{p}{,} \PY{n}{player\PYZus{}difficulty} \PY{o}{=} \PY{k+kc}{None}\PY{p}{)}\PY{p}{:}
            \PY{n}{players} \PY{o}{=} \PY{p}{[}\PY{p}{]}
            \PY{k}{for} \PY{n}{i} \PY{o+ow}{in} \PY{n+nb}{range}\PY{p}{(}\PY{l+m+mi}{2}\PY{p}{)}\PY{p}{:}
                \PY{n}{num} \PY{o}{=} \PY{n}{i}\PY{o}{+}\PY{l+m+mi}{1}
                \PY{k}{if} \PY{n}{player\PYZus{}names} \PY{o+ow}{is} \PY{k+kc}{None}\PY{p}{:}
                    \PY{n}{name} \PY{o}{=} \PY{n+nb+bp}{self}\PY{o}{.}\PY{n}{get\PYZus{}player\PYZus{}name}\PY{p}{(}\PY{n}{num}\PY{p}{)}
                \PY{k}{else}\PY{p}{:}
                    \PY{n}{name} \PY{o}{=} \PY{n}{player\PYZus{}names}\PY{p}{[}\PY{n}{i}\PY{p}{]}
        
                \PY{k}{if} \PY{n}{player\PYZus{}types} \PY{o+ow}{is} \PY{k+kc}{None}\PY{p}{:}
                    \PY{n}{player\PYZus{}type} \PY{o}{=} \PY{n+nb+bp}{self}\PY{o}{.}\PY{n}{get\PYZus{}player\PYZus{}type}\PY{p}{(}\PY{n}{num}\PY{p}{)}
                \PY{k}{else}\PY{p}{:}
                    \PY{n}{player\PYZus{}type} \PY{o}{=} \PY{n}{player\PYZus{}types}\PY{p}{[}\PY{n}{i}\PY{p}{]}
        
                \PY{n}{difficulty} \PY{o}{=} \PY{k+kc}{None}
                \PY{k}{if} \PY{n}{player\PYZus{}type} \PY{o}{==} \PY{l+m+mi}{2} \PY{o+ow}{and} \PY{n}{player\PYZus{}difficulty} \PY{o+ow}{is} \PY{k+kc}{None}\PY{p}{:}
                    \PY{n}{difficulty} \PY{o}{=} \PY{n+nb+bp}{self}\PY{o}{.}\PY{n}{get\PYZus{}difficulty}\PY{p}{(}\PY{n}{num}\PY{p}{)}
                \PY{k}{elif} \PY{n}{player\PYZus{}type} \PY{o}{==} \PY{l+m+mi}{2} \PY{o+ow}{and} \PY{n}{player\PYZus{}difficulty} \PY{o+ow}{is} \PY{o+ow}{not} \PY{k+kc}{None}\PY{p}{:}
                    \PY{n}{difficulty} \PY{o}{=} \PY{n}{player\PYZus{}difficulty}\PY{p}{[}\PY{n}{i}\PY{p}{]}
        
                \PY{n}{new\PYZus{}player} \PY{o}{=} \PY{n}{PlayerSettings}\PY{p}{(}\PY{n}{num}\PY{p}{,} \PY{n}{name}\PY{p}{,} \PY{n}{player\PYZus{}type}\PY{p}{,} \PY{n}{difficulty}\PY{p}{)}
                \PY{n}{players}\PY{o}{.}\PY{n}{append}\PY{p}{(}\PY{n}{new\PYZus{}player}\PY{p}{)}
        
            \PY{k}{return} \PY{n}{players}
\end{Verbatim}

    Dabei werden für jeden Spieler einmal ein Block an Abfragen
durchgegangen, der alle Einstellungen abfragt und diese entsprechend
speichet.

Nach der Abfrage der Informationen "Name" und "Spielertyp" wird im
Falle, dass dieser Spielertyp "2", der der KI entspricht, gleicht, auch
noch der Schwierigkeitsgrad abgefragt, sollte dieser nicht bereits als
Startparameter übergeben wordne sein.

Anschließend wird ein neues Einstellungsparameterschema für Spieler,
genannt \texttt{PlayerSettings}, erstellt und der Liste hinzugefügt.
Diese Liste wird nach Durchgang der Schleife für beide Spieler zurück
gegeben.

Nachdem die Einstellungen für alle Spieler bekannt sind, werden die
Spieler erstellt. Dies funktioniert wie in nachfolgendem Code-Snippet zu
sehen.

    \begin{Verbatim}[commandchars=\\\{\}]
{\color{incolor}In [{\color{incolor} }]:} \PY{n}{player\PYZus{}settings} \PY{o}{=} \PY{n}{settings\PYZus{}ui}\PY{o}{.}\PY{n}{interrogate\PYZus{}settings}\PY{p}{(}\PY{p}{)}
        \PY{n}{players} \PY{o}{=} \PY{p}{[}\PY{p}{]}
        \PY{k}{for} \PY{n}{player\PYZus{}setting} \PY{o+ow}{in} \PY{n}{player\PYZus{}settings}\PY{p}{:}
            \PY{n+nb}{type} \PY{o}{=} \PY{n}{type\PYZus{}switcher}\PY{p}{(}\PY{n}{player\PYZus{}setting}\PY{o}{.}\PY{n}{type}\PY{p}{)}
            \PY{n}{players}\PY{o}{.}\PY{n}{append}\PY{p}{(}\PY{n+nb}{type}\PY{o}{.}\PY{n}{Player}\PY{p}{(}\PY{n}{player\PYZus{}setting}\PY{o}{.}\PY{n}{num}\PY{p}{,} \PY{n}{player\PYZus{}setting}\PY{o}{.}\PY{n}{name}\PY{p}{,} \PY{n}{ui\PYZus{}status}\PY{p}{,} \PY{n}{player\PYZus{}setting}\PY{o}{.}\PY{n}{difficulty}\PY{p}{)}\PY{p}{)}
\end{Verbatim}

    Dabei werden zunächst wie gehabt die Einstellungen abgefragt, ehe eine
Liste aller Spieler erstellt wird. Dann wird über jeden Eintrag in der
Einstellungsliste iteriert und für diesen zunächst der Spielertyp
berechnet. Dabei wird der \texttt{type\_switcher} zur Hilfe genommen,
welcher je nach angegebenen Typ die entsprechende Klasse des Spielertyps
zurückgibt. Mittels dieser wird dann ein neuer Spieler erstellt, wobei
beim Initialisierren alle weiteren Daten mitgegeben werden. Dieser neue
Spieler wird dann der Spielerliste hinzugefügt. Mit dieser Liste an
Spielern wird dann der \texttt{ChessMaster} initialisiert, der im
nächsten Kapitel beschrieben wird.

    \section{Verwalten des Schachspiels und Pflege des
Spielverlaufs}\label{verwalten-des-schachspiels-und-pflege-des-spielverlaufs}

Ein essentieller Part im Erstellen eines Schachprogrammes ist erstmal
das Verwalten des Schachspiels. Dabei muss garantiert werden, dass -
Solange das Spiel nicht vorbei ist, die Spieler abwechselnd einen Zug
auswählen können - Der Zug auf dem vorhandenen Schachbrett durchgeführt
wird - Das daraus entstehende Schachbrett dem Spieler sichtbar gemacht
wird

Dies ist Aufgabe des \texttt{ChessMaster} und ist mit folgendem Code
umgesetzt

    \begin{Verbatim}[commandchars=\\\{\}]
{\color{incolor}In [{\color{incolor} }]:} \PY{n}{board} \PY{o}{=} \PY{n}{chess}\PY{o}{.}\PY{n}{Board}\PY{p}{(}\PY{p}{)}
        \PY{k}{while} \PY{o+ow}{not} \PY{n}{board}\PY{o}{.}\PY{n}{is\PYZus{}game\PYZus{}over}\PY{p}{(}\PY{p}{)}\PY{p}{:}
            \PY{n}{current\PYZus{}player} \PY{o}{=} \PY{n}{players}\PY{p}{[}\PY{n+nb}{int}\PY{p}{(}\PY{o+ow}{not} \PY{n}{board}\PY{o}{.}\PY{n}{turn}\PY{p}{)}\PY{p}{]}
            \PY{n}{current\PYZus{}player}\PY{o}{.}\PY{n}{print\PYZus{}board}\PY{p}{(}\PY{n}{current\PYZus{}player}\PY{o}{.}\PY{n}{name}\PY{p}{,} \PY{n}{board}\PY{p}{)}
        
            \PY{n}{move} \PY{o}{=} \PY{n}{current\PYZus{}player}\PY{o}{.}\PY{n}{get\PYZus{}move}\PY{p}{(}\PY{n}{board}\PY{p}{)}
            \PY{n}{board}\PY{o}{.}\PY{n}{push}\PY{p}{(}\PY{n}{move}\PY{p}{)}
            \PY{n}{current\PYZus{}player}\PY{o}{.}\PY{n}{submit\PYZus{}move}\PY{p}{(}\PY{n}{move}\PY{p}{)}
\end{Verbatim}

    Dabei wird zunächst ein neues Schachbrett - hier board genannt -
erstellt. Solange das Spiel auf diesem nicht vorbei ist, was mittels der
\texttt{board.is\_game\_over()} Funktiong geprüft werden kann, wird dann
stets die gleiche Schleife durchlaufen.

In dieser wird zuallererst der aktuelle Spieler ermittelt und
referenziert. Dazu wird aus einer vorhandenen Liste aller Spieler
derjenige gewählt, dessen Position in der Liste der gespeicherten
Zugnummer des boards entspricht. Diese ist entweder 0, wenn der weiße
Spieler an der Reihe ist, oder 1, wenn der schwarze Spieler den nächsten
Zug auswählen kann.

Der daraus berechnete Spieler wurde zuvor beim Spielstart ein Spielertyp
zugewiesen, der auf einer Schnittstelle basiert und somit alle nötigen
Funktionen implementiert. Diese wurden im Kapitel INSERT beschrieben. Im
Verwalter des Schachspiels werden diese nun nach für nach aufgerufen.

Zunächst wird das board für den Spieler ausgegeben mittels der
\texttt{print\_board} Funktion. Ist der Spieler ein Nutzer, so wird
diese für gewöhnlich am Nutzerinterface ausgegeben. Andernfalls ist dies
nicht nötig und die Funktion kann leer bleiben, ohne etwas auszugeben.

Nun kommt es zum wichtigsten Teil - dem Berechnen des nächsten Zuges.
Dazu wird der Spieler aufgefordert an Hand eines gegebenen boards den
nächstne Zug zu nennen. In diesem Teil übernimmt beispielsweise die KI
ihre Berechnungen für den nächsten Schachzug. Der Nutzer dagegen gibt
diesen mittels eines Eingabefeldes ein.

Nachdem der Zug vom entsprechenden Spieler berechnet und zurückgegeben
wurde, wird dieser dem aktuellen board hinzugefügt. Dadurch wechselt
auch automatisch der Spieler, der an der Reihe ist, wodurch dieser im
nächsten Schleifendurchlauf nach dessen Zug gefragt wird.

Abschließend wird das neue board nochmal abgeschickt. Dabei kann zum
Beispiel eine erneute Ausgabe des Schachbretts mit dem aktualisierten
Zustand stattfinden oder beispielsweise bei der Nutzung einer Online-API
der gewählte Zug an die Schnittstelle gesendet werden.

Nach dem Ende des Spiels wird noch das Ergebnis ausgegeben. Der Code
dazu ist im folgenden Code Snippet zu sehen.

    \begin{Verbatim}[commandchars=\\\{\}]
{\color{incolor}In [{\color{incolor} }]:} \PY{n}{result} \PY{o}{=} \PY{n}{Tools}\PY{o}{.}\PY{n}{get\PYZus{}board\PYZus{}result}\PY{p}{(}\PY{n}{board}\PY{p}{)}
        \PY{k}{if} \PY{n}{result} \PY{o+ow}{is} \PY{l+m+mi}{1}\PY{p}{:}
            \PY{n}{players}\PY{p}{[}\PY{l+m+mi}{0}\PY{p}{]}\PY{o}{.}\PY{n}{print\PYZus{}win\PYZus{}msg}\PY{p}{(}\PY{p}{)}
            \PY{n}{players}\PY{p}{[}\PY{l+m+mi}{1}\PY{p}{]}\PY{o}{.}\PY{n}{print\PYZus{}loose\PYZus{}msg}\PY{p}{(}\PY{p}{)}
        \PY{k}{elif} \PY{n}{result} \PY{o+ow}{is} \PY{o}{\PYZhy{}}\PY{l+m+mi}{1}\PY{p}{:}
            \PY{n}{players}\PY{p}{[}\PY{l+m+mi}{1}\PY{p}{]}\PY{o}{.}\PY{n}{print\PYZus{}win\PYZus{}msg}\PY{p}{(}\PY{p}{)}
            \PY{n}{players}\PY{p}{[}\PY{l+m+mi}{0}\PY{p}{]}\PY{o}{.}\PY{n}{print\PYZus{}loose\PYZus{}msg}\PY{p}{(}\PY{p}{)}
        \PY{k}{else}\PY{p}{:}
            \PY{n}{players}\PY{p}{[}\PY{l+m+mi}{0}\PY{p}{]}\PY{o}{.}\PY{n}{print\PYZus{}draw\PYZus{}msg}\PY{p}{(}\PY{p}{)}
            \PY{n}{players}\PY{p}{[}\PY{l+m+mi}{1}\PY{p}{]}\PY{o}{.}\PY{n}{print\PYZus{}draw\PYZus{}msg}\PY{p}{(}\PY{p}{)}
\end{Verbatim}

    Dabei wird zuerst das Ergebnis an Hand der \texttt{get\_board\_result}
Funktion aus der Hilfsklasse "Tools" ermittelt. Diese sieht wie folgt
aus:

    \begin{Verbatim}[commandchars=\\\{\}]
{\color{incolor}In [{\color{incolor} }]:} \PY{k}{if} \PY{n}{board}\PY{o}{.}\PY{n}{is\PYZus{}variant\PYZus{}loss}\PY{p}{(}\PY{p}{)}\PY{p}{:}
            \PY{k}{return} \PY{o}{\PYZhy{}}\PY{l+m+mi}{1} \PY{k}{if} \PY{n}{board}\PY{o}{.}\PY{n}{turn} \PY{o}{==} \PY{n}{chess}\PY{o}{.}\PY{n}{WHITE} \PY{k}{else} \PY{l+m+mi}{1}
        \PY{k}{elif} \PY{n}{board}\PY{o}{.}\PY{n}{is\PYZus{}variant\PYZus{}win}\PY{p}{(}\PY{p}{)}\PY{p}{:}
            \PY{k}{return} \PY{l+m+mi}{1} \PY{k}{if} \PY{n}{board}\PY{o}{.}\PY{n}{turn} \PY{o}{==} \PY{n}{chess}\PY{o}{.}\PY{n}{WHITE} \PY{k}{else} \PY{o}{\PYZhy{}}\PY{l+m+mi}{1}
        \PY{k}{elif} \PY{n}{board}\PY{o}{.}\PY{n}{is\PYZus{}variant\PYZus{}draw}\PY{p}{(}\PY{p}{)}\PY{p}{:}
            \PY{k}{return} \PY{l+m+mi}{0}
        \PY{c+c1}{\PYZsh{} Checkmate.}
        \PY{k}{if} \PY{n}{board}\PY{o}{.}\PY{n}{is\PYZus{}checkmate}\PY{p}{(}\PY{p}{)}\PY{p}{:}
            \PY{k}{return} \PY{o}{\PYZhy{}}\PY{l+m+mi}{1} \PY{k}{if} \PY{n}{board}\PY{o}{.}\PY{n}{turn} \PY{o}{==} \PY{n}{chess}\PY{o}{.}\PY{n}{WHITE} \PY{k}{else} \PY{l+m+mi}{1}
        \PY{c+c1}{\PYZsh{} Draw claimed.}
        \PY{k}{if} \PY{n}{board}\PY{o}{.}\PY{n}{can\PYZus{}claim\PYZus{}draw}\PY{p}{(}\PY{p}{)}\PY{p}{:}
            \PY{k}{return} \PY{l+m+mi}{0}
        \PY{c+c1}{\PYZsh{} Seventyfive\PYZhy{}move rule or fivefold repetition.}
        \PY{k}{if} \PY{n}{board}\PY{o}{.}\PY{n}{is\PYZus{}seventyfive\PYZus{}moves}\PY{p}{(}\PY{p}{)} \PY{o+ow}{or} \PY{n}{board}\PY{o}{.}\PY{n}{is\PYZus{}fivefold\PYZus{}repetition}\PY{p}{(}\PY{p}{)}\PY{p}{:}
            \PY{k}{return} \PY{l+m+mi}{0}
        \PY{c+c1}{\PYZsh{} Insufficient material.}
        \PY{k}{if} \PY{n}{board}\PY{o}{.}\PY{n}{is\PYZus{}insufficient\PYZus{}material}\PY{p}{(}\PY{p}{)}\PY{p}{:}
            \PY{k}{return} \PY{l+m+mi}{0}
        \PY{c+c1}{\PYZsh{} Stalemate.}
        \PY{k}{if} \PY{o+ow}{not} \PY{n+nb}{any}\PY{p}{(}\PY{n}{board}\PY{o}{.}\PY{n}{generate\PYZus{}legal\PYZus{}moves}\PY{p}{(}\PY{p}{)}\PY{p}{)}\PY{p}{:}
            \PY{k}{return} \PY{l+m+mi}{0}
        \PY{k}{return} \PY{l+m+mi}{0}
\end{Verbatim}

    Dabei wird jede mögliche Option, wie das Spiel zum Ende gekommen sein
kann, durchgegangen und anschließend das jeweilige Ergebnis
zurückgegeben. Eine 1 steht für einen Sieg für weiß, eine -1 für einen
Sieg für schwarz und eine 0 für ein Unentschieden beziehungsweise einen
Patt.

Nach Abfrage des Ergebnisses werden beide Spieler abhängig von diesem
entweder dazu aufgefordert eine Siesesbenachrichtigung auszugeben oder
aber eine Benachrichtigung über die Niederlage oder ein Unentschieden.

Zusätzlich zu dem Verwalten des Spiels ist es auch Aufgabe des
\texttt{ChessMaster}s die Historie aller Spiele zu pflegen. Dazu wird
zunächst eine Liste von Schachbrettern der Klasse \texttt{chess.Board}
angelegt. Nach jedem durchgeführten Zug wird der neue Zustand des boards
zu dieser hinzugefügt. Dabei wird das Schachbrett in der fen-Notation
gespeichert, die in einem String den exakten Zustand des Schachbretts
wiedergeben kann. Dabei wird jedoch nur der erste Teil dieser Notation
gespeichert, da dieser alleine bereits Aufschluss über die
Positionierungen gibt. Die darauf folgenden Teile sind zum Speichern des
Spielers, der am Zug ist, wie viele Züge bereits durchgeführt wurden und
weitere Informationen, die zum Bewerten in der Historie nicht notwendig
sind.

Das Speichern der Züge im Verlauf des Spiels sieht dann wie folgt aus

    \begin{Verbatim}[commandchars=\\\{\}]
{\color{incolor}In [{\color{incolor} }]:} \PY{n}{board} \PY{o}{=} \PY{n}{chess}\PY{o}{.}\PY{n}{Board}\PY{p}{(}\PY{p}{)}
        \PY{n}{turn\PYZus{}list} \PY{o}{=} \PY{n+nb}{list}\PY{p}{(}\PY{p}{)}
        
        \PY{k}{while} \PY{o+ow}{not} \PY{n}{board}\PY{o}{.}\PY{n}{is\PYZus{}game\PYZus{}over}\PY{p}{(}\PY{p}{)}\PY{p}{:}
            \PY{n}{current\PYZus{}player} \PY{o}{=} \PY{n}{players}\PY{p}{[}\PY{n+nb}{int}\PY{p}{(}\PY{o+ow}{not} \PY{n}{board}\PY{o}{.}\PY{n}{turn}\PY{p}{)}\PY{p}{]}
            \PY{n}{current\PYZus{}player}\PY{o}{.}\PY{n}{print\PYZus{}board}\PY{p}{(}\PY{n}{current\PYZus{}player}\PY{o}{.}\PY{n}{name}\PY{p}{,} \PY{n}{board}\PY{p}{)}
        
            \PY{n}{move} \PY{o}{=} \PY{n}{current\PYZus{}player}\PY{o}{.}\PY{n}{get\PYZus{}move}\PY{p}{(}\PY{n}{board}\PY{p}{)}
            \PY{n}{board}\PY{o}{.}\PY{n}{push}\PY{p}{(}\PY{n}{move}\PY{p}{)}
            \PY{n}{current\PYZus{}player}\PY{o}{.}\PY{n}{submit\PYZus{}move}\PY{p}{(}\PY{n}{move}\PY{p}{)}
            
            \PY{n}{turn\PYZus{}list}\PY{o}{.}\PY{n}{append}\PY{p}{(}\PY{n}{board}\PY{o}{.}\PY{n}{fen}\PY{p}{(}\PY{p}{)}\PY{o}{.}\PY{n}{split}\PY{p}{(}\PY{l+s+s2}{\PYZdq{}}\PY{l+s+s2}{ }\PY{l+s+s2}{\PYZdq{}}\PY{p}{)}\PY{p}{[}\PY{l+m+mi}{0}\PY{p}{]}\PY{p}{)}
\end{Verbatim}

    In Zeile 2 wird die Liste erstellt und in der letzten Zeile der Zustand
in diese Liste gespeichert. Nach Ende des Spiels muss diese Liste noch
in die bestehende Historie eingepflegt werden. Dies ist in folgendem
Code-Snippet zu sehen.

    \begin{Verbatim}[commandchars=\\\{\}]
{\color{incolor}In [{\color{incolor} }]:} \PY{k}{def} \PY{n+nf}{groom\PYZus{}board\PYZus{}history}\PY{p}{(}\PY{n+nb+bp}{self}\PY{p}{,} \PY{n}{final\PYZus{}board}\PY{p}{,} \PY{n}{turn\PYZus{}list}\PY{p}{)}\PY{p}{:}
            \PY{n}{victory\PYZus{}status} \PY{o}{=} \PY{n}{Tools}\PY{o}{.}\PY{n}{get\PYZus{}board\PYZus{}result}\PY{p}{(}\PY{n}{final\PYZus{}board}\PY{p}{)}
        
            \PY{n}{new\PYZus{}turn\PYZus{}dict} \PY{o}{=} \PY{n+nb}{dict}\PY{o}{.}\PY{n}{fromkeys}\PY{p}{(}\PY{n}{turn\PYZus{}list}\PY{p}{,} \PY{n}{victory\PYZus{}status}\PY{p}{)}
        
            \PY{n}{history} \PY{o}{=} \PY{n}{pd}\PY{o}{.}\PY{n}{read\PYZus{}csv}\PY{p}{(}\PY{n}{HISTORY\PYZus{}FILE\PYZus{}LOC}\PY{p}{)}
            \PY{n}{history\PYZus{}dict} \PY{o}{=} \PY{n+nb}{dict}\PY{p}{(}\PY{n+nb}{zip}\PY{p}{(}\PY{n+nb}{list}\PY{p}{(}\PY{n}{history}\PY{o}{.}\PY{n}{board}\PY{p}{)}\PY{p}{,} \PY{n+nb}{list}\PY{p}{(}\PY{n}{history}\PY{o}{.}\PY{n}{value}\PY{p}{)}\PY{p}{)}\PY{p}{)}
        
            \PY{n}{merged\PYZus{}history\PYZus{}dict} \PY{o}{=} \PY{p}{\PYZob{}} \PY{n}{k}\PY{p}{:} \PY{n}{new\PYZus{}turn\PYZus{}dict}\PY{o}{.}\PY{n}{get}\PY{p}{(}\PY{n}{k}\PY{p}{,} \PY{l+m+mi}{0}\PY{p}{)} \PY{o}{+} \PY{n}{history\PYZus{}dict}\PY{o}{.}\PY{n}{get}\PY{p}{(}\PY{n}{k}\PY{p}{,} \PY{l+m+mi}{0}\PY{p}{)} \PY{k}{for} \PY{n}{k} \PY{o+ow}{in} \PY{n+nb}{set}\PY{p}{(}\PY{n}{new\PYZus{}turn\PYZus{}dict}\PY{p}{)} \PY{o}{|} \PY{n+nb}{set}\PY{p}{(}\PY{n}{history\PYZus{}dict}\PY{p}{)} \PY{p}{\PYZcb{}}
            \PY{n}{merged\PYZus{}history} \PY{o}{=} \PY{n}{pd}\PY{o}{.}\PY{n}{DataFrame}\PY{p}{(}\PY{n+nb}{list}\PY{p}{(}\PY{n}{merged\PYZus{}history\PYZus{}dict}\PY{o}{.}\PY{n}{items}\PY{p}{(}\PY{p}{)}\PY{p}{)}\PY{p}{,} \PY{n}{columns}\PY{o}{=}\PY{p}{[}\PY{l+s+s1}{\PYZsq{}}\PY{l+s+s1}{board}\PY{l+s+s1}{\PYZsq{}}\PY{p}{,}\PY{l+s+s1}{\PYZsq{}}\PY{l+s+s1}{value}\PY{l+s+s1}{\PYZsq{}}\PY{p}{]}\PY{p}{)}
        
            \PY{n}{merged\PYZus{}history}\PY{o}{.}\PY{n}{to\PYZus{}csv}\PY{p}{(}\PY{n}{HISTORY\PYZus{}FILE\PYZus{}LOC}\PY{p}{)}
\end{Verbatim}

    Dabei wird zuerst das Ergebnis ermittelt mittels der bereits
vorgestellten Funktion \texttt{get\_board\_result}. Danach wird ein
Dictionary angelegt, das auf der einen Seite jeden gespeicherten Zustand
des einzupflegenden Spiels als Schlüssel enthält und auf der anderen
Seite als Wert den Ausgang des Spiels. Zur Erinnerung - dieser beträgt 1
bei Sieg von weiß, 0 bei Unentschieden/Patt und -1 bei Sieg von Schwarz.

Nachdem das Dictionary der neu einzupflegenden Züge angelegt ist, wird
die vorhandene Historie aus der entsprechenden Datei ausgelesen. Dies
wird mittels der in \texttt{panda} enthaltenden Funktion
\texttt{pd.read\_csv} durchgeführt. Anschließend wird auch daraus ein
Dictionary erstellt, indem die Zeile "board" als Schlüssel und die Zeile
"value" als Wert verwendet wird.

Nun werden die beiden Dictionaries zusammengefügt. Dazu wird für jeden
Schlüssel aus dem Dictionary der neu einzupflegenden Spielzüge oder dem
vorhandenen Historie-Dictionary der berechnete Wert aus ersterem auf den
vorhandenen Wert in der Spielhistorie aufaddiert. Dadurch erhalten wird
ein Dictionary, das alle Schachzustände aus der Historie sowie den neuen
Spielzuständen vereint und dessen Werte durch Addition kombiniert,
wodurch der neu berechnete Wert Aufschluss über den wahrscheinlichen
Sieger ausgehend von einem bestimmten Zustand geben kann.

Anschließend wird dieses Dictionary in ein Panda Dataframe umgewandelt,
wobei der Schlüssel für die Zeile "board" und der Wert für die Zeile
"value" verwendet wird, um das DataDrame anschließend wieder zu der
CSV-Datei zu speichern.

Dies ermöglicht nicht nur eine Verwaltung des Spiels, sondern
gleichzeitig auch eine Speicherung aller möglichen Zustände, die bei den
Evaluierungsfunktionen (beschrieben in INSERT) zur Hilfe genommen werden
können

    \section{Implementierung des Iterative Deepening}\label{implementierung-des-iterative-deepening}

Die zentrale Aufgabe der KI ist es, den best möglichen Zug für eine
gegebene Situation zu finden. Dabei geht diese so vor, dass sie alle
möglichen Züge durchgeht und die daraus entstehenden Zustände analysiert
und evaluiert. Dabei jedoch zählt nicht nur der direkt erreichbare
Zustand zählt, sondern auch die aus diesem Zustand erreichbaren Zustände
und so weiter. Aus diesem Grund wird immer bis zu einer bestimmten Tiefe
in die Züge hineingeschaut und die sich daraus ergebenden Zustände
evaluiert.

Um dies jedoch nicht fest immer bis zu einer bestimmten Tiefe durchgehen
zu lassen, sondern variabel anzupassen, je nachdem wie viele Züge von
dem gegebenen Zustand aus möglich sind, kann ein Zeitlimit dienen. Dabei
wird anfangs die Tiefe auf 1 gesetzt und dann mittels des
Minimax-Algorithmus die Züge evaluiert. Danach wird die Tiefe um 1
erhöht und erneut der Minimax-Algorithmus angewandt. Dies wird solange
wiederholt, bis die angegebene Zeit abgelaufen ist. Dieser Algorithmus
nennt sich "Iterative Deepening". Nähere Informationen dazu sind in
Kapitel INSERT aufgeführt.

Dem Algorithmus muss dazu der aktuelle Zustand sowie eine maximale Tiefe
mitgegeben werden. Ist diese erreicht bricht der Algorithmus ab,
unabhängig davon, ob das Zeitlimit überschritten ist oder nicht.
Zunächst müssen beim Ausführen dann einige Werte wie folgt festgelegt
werden.

    \begin{Verbatim}[commandchars=\\\{\}]
{\color{incolor}In [{\color{incolor} }]:} \PY{n}{depth} \PY{o}{=} \PY{l+m+mi}{1}
            
        \PY{n}{start\PYZus{}time} \PY{o}{=} \PY{n+nb}{int}\PY{p}{(}\PY{n}{time}\PY{o}{.}\PY{n}{time}\PY{p}{(}\PY{p}{)}\PY{p}{)}
        \PY{n}{end\PYZus{}time} \PY{o}{=} \PY{n}{start\PYZus{}time} \PY{o}{+} \PY{n+nb+bp}{self}\PY{o}{.}\PY{n}{time\PYZus{}limit}
        \PY{n}{current\PYZus{}time} \PY{o}{=} \PY{n}{start\PYZus{}time}
        
        \PY{n}{player} \PY{o}{=} \PY{n+nb}{bool}\PY{p}{(}\PY{n}{board}\PY{o}{.}\PY{n}{turn}\PY{p}{)}
        \PY{n+nb+bp}{self}\PY{o}{.}\PY{n}{best\PYZus{}possible\PYZus{}result} \PY{o}{=} \PY{n+nb+bp}{self}\PY{o}{.}\PY{n}{get\PYZus{}best\PYZus{}possible\PYZus{}result}\PY{p}{(}\PY{n}{board}\PY{p}{,} \PY{n}{player}\PY{p}{)}
\end{Verbatim}

    Zunächst wird die Starttiefe auf 1 festgesetzt. Danach wird die
Startzeit auf die aktuelle Zeit gesetzt und die Endzeit berechnet, indem
auf die Startzeit das Zeitlimit addiert wird. Zudem wird der erste Wert
für die aktuelle Zeit auf die Startzeit festgelegt.

Anschließend wird für den Spieler, der aktuell am Zug ist, berechnet,
was das bestmöglich zu erreichende Resultat ist. Dies wird mit der
Funktion \texttt{get\_best\_possible\_result} durchgeführt. Dies ist
dazu gut, um finale Zustände dahingehend zu evaluieren, ob diese für den
Nutzer die best mögliche Option ist (Sieg oder Unentschieden wenn Sieg
nicht mehr möglich ist) und dementsprechend zu bewerten. Die Funktion
sieht wie folgt aus.

    \begin{Verbatim}[commandchars=\\\{\}]
{\color{incolor}In [{\color{incolor} }]:} \PY{k}{def} \PY{n+nf}{get\PYZus{}best\PYZus{}possible\PYZus{}result}\PY{p}{(}\PY{n+nb+bp}{self}\PY{p}{,} \PY{n}{board}\PY{p}{,} \PY{n}{player}\PY{p}{)}\PY{p}{:}
            \PY{k}{if} \PY{n}{player} \PY{o+ow}{and} \PY{n}{board}\PY{o}{.}\PY{n}{has\PYZus{}insufficient\PYZus{}material}\PY{p}{(}\PY{n}{chess}\PY{o}{.}\PY{n}{WHITE}\PY{p}{)}\PY{p}{:}
                \PY{k}{return} \PY{l+m+mi}{0}
            \PY{k}{if} \PY{o+ow}{not} \PY{n}{player} \PY{o+ow}{and} \PY{n}{board}\PY{o}{.}\PY{n}{has\PYZus{}insufficient\PYZus{}material}\PY{p}{(}\PY{n}{chess}\PY{o}{.}\PY{n}{BLACK}\PY{p}{)}\PY{p}{:}
                \PY{k}{return} \PY{l+m+mi}{0}
            \PY{k}{if} \PY{n}{player} \PY{o+ow}{and} \PY{o+ow}{not} \PY{n}{board}\PY{o}{.}\PY{n}{has\PYZus{}insufficient\PYZus{}material}\PY{p}{(}\PY{n}{chess}\PY{o}{.}\PY{n}{WHITE}\PY{p}{)}\PY{p}{:}
                \PY{k}{return} \PY{l+m+mi}{1}
            \PY{k}{if} \PY{o+ow}{not} \PY{n}{player} \PY{o+ow}{and} \PY{o+ow}{not} \PY{n}{board}\PY{o}{.}\PY{n}{has\PYZus{}insufficient\PYZus{}material}\PY{p}{(}\PY{n}{chess}\PY{o}{.}\PY{n}{BLACK}\PY{p}{)}\PY{p}{:}
                \PY{k}{return} \PY{o}{\PYZhy{}}\PY{l+m+mi}{1}
\end{Verbatim}

    Dabei muss der Funktion der aktuelle Zustand sowie der Spieler, der an
der Reihe ist, mitgegeben werden. Ist der aktuelle Spieler der der
weißen Figuren (player == True) und hat weiß unzureichende Materialien
für einen Sieg, so ist der bestmögliche Zustand ein Patt. Das gleiche
Ergebnis wird zurückgegeben, wenn der Spieler der der schwarzen Figuren
ist (player == False) und schwarz unzureichende Materialien hat.

Ist der Spieler jedoch weiß und er hat noch ausreichend Materialien, so
wird der Wert 1 zurückgegeben, da ein Sieg noch erreichbar ist. Genauso
wird für den schwarzen Spieler der Wert -1 zurückgegeben, falls er noch
ausreichende Materialien besitzt, da dieser noch einen Sieg erreichen
kann und der Wert -1 für einen Sieg von Schwarz steht.

Nach dieser Abfrage wird der eigentliche Algorithmus des "Iterative
Deepening" durchgeführt.

    \begin{Verbatim}[commandchars=\\\{\}]
{\color{incolor}In [{\color{incolor} }]:} \PY{n}{legal\PYZus{}moves} \PY{o}{=} \PY{n+nb}{list}\PY{p}{(}\PY{n}{board}\PY{o}{.}\PY{n}{legal\PYZus{}moves}\PY{p}{)}
        \PY{k}{while} \PY{n}{current\PYZus{}time} \PY{o}{\PYZlt{}} \PY{n}{end\PYZus{}time} \PY{o+ow}{and} \PY{n}{depth} \PY{o}{\PYZlt{}}\PY{o}{=} \PY{n}{max\PYZus{}depth}\PY{p}{:}
            \PY{n}{move\PYZus{}val\PYZus{}dict} \PY{o}{=} \PY{p}{\PYZob{}}\PY{p}{\PYZcb{}}
        
            \PY{n}{best\PYZus{}value} \PY{o}{=} \PY{n+nb}{float}\PY{p}{(}\PY{l+s+s1}{\PYZsq{}}\PY{l+s+s1}{\PYZhy{}inf}\PY{l+s+s1}{\PYZsq{}}\PY{p}{)}
            \PY{n}{best\PYZus{}move} \PY{o}{=} \PY{n}{legal\PYZus{}moves}\PY{p}{[}\PY{l+m+mi}{0}\PY{p}{]}
        
            \PY{k}{for} \PY{n}{move} \PY{o+ow}{in} \PY{n}{legal\PYZus{}moves}\PY{p}{:}
                \PY{n}{tmp\PYZus{}board} \PY{o}{=} \PY{n}{chess}\PY{o}{.}\PY{n}{Board}\PY{p}{(}\PY{n+nb}{str}\PY{p}{(}\PY{n}{board}\PY{o}{.}\PY{n}{fen}\PY{p}{(}\PY{p}{)}\PY{p}{)}\PY{p}{)}
                \PY{n}{tmp\PYZus{}board}\PY{o}{.}\PY{n}{push}\PY{p}{(}\PY{n}{move}\PY{p}{)}
        
                \PY{n}{value} \PY{o}{=} \PY{n+nb+bp}{self}\PY{o}{.}\PY{n}{min\PYZus{}value}\PY{p}{(}\PY{n+nb}{str}\PY{p}{(}\PY{n}{tmp\PYZus{}board}\PY{o}{.}\PY{n}{fen}\PY{p}{(}\PY{p}{)}\PY{p}{)}\PY{p}{,} \PY{n}{player}\PY{p}{,} \PY{n+nb}{float}\PY{p}{(}\PY{l+s+s1}{\PYZsq{}}\PY{l+s+s1}{\PYZhy{}inf}\PY{l+s+s1}{\PYZsq{}}\PY{p}{)}\PY{p}{,} \PY{n+nb}{float}\PY{p}{(}\PY{l+s+s1}{\PYZsq{}}\PY{l+s+s1}{inf}\PY{l+s+s1}{\PYZsq{}}\PY{p}{)}\PY{p}{,} \PY{n}{depth} \PY{o}{\PYZhy{}} \PY{l+m+mi}{1}\PY{p}{,} \PY{n}{end\PYZus{}time}\PY{p}{)}
                \PY{n}{move\PYZus{}val\PYZus{}dict}\PY{p}{[}\PY{n}{move}\PY{p}{]} \PY{o}{=} \PY{n}{value}
        
                \PY{k}{if} \PY{n}{value} \PY{o}{==} \PY{n}{MAX\PYZus{}BOARD\PYZus{}VALUE}\PY{p}{:}
                    \PY{k}{return} \PY{n}{move}
                \PY{k}{if} \PY{n}{value} \PY{o}{\PYZgt{}} \PY{n}{best\PYZus{}value}\PY{p}{:}
                    \PY{n}{best\PYZus{}value} \PY{o}{=} \PY{n}{value}
                    \PY{n}{best\PYZus{}move} \PY{o}{=} \PY{n}{move}
        
            \PY{n}{legal\PYZus{}moves}\PY{o}{.}\PY{n}{sort}\PY{p}{(}\PY{n}{key}\PY{o}{=}\PY{n}{move\PYZus{}val\PYZus{}dict}\PY{o}{.}\PY{n}{get}\PY{p}{,} \PY{n}{reverse}\PY{o}{=}\PY{k+kc}{True}\PY{p}{)}
            \PY{n}{depth} \PY{o}{+}\PY{o}{=} \PY{l+m+mi}{1}
            \PY{n}{current\PYZus{}time} \PY{o}{=} \PY{n+nb}{int}\PY{p}{(}\PY{n}{time}\PY{o}{.}\PY{n}{time}\PY{p}{(}\PY{p}{)}\PY{p}{)}
\end{Verbatim}

    Dabei wird zunächst eine Liste aller legalen Züge erstellt. Dann wird
eine Schleife so lange durchlaufen, bsi entweder die Zeit abgelaufen ist
oder aber die maximale Tiefe erreicht ist.

In dieser Schleife wird ein Dictionary aller Züge mit ihren berechneten
Werte erstellt. Zudem werden Anfangswerte für die besten Züge und dessen
Wert festgelegt. Der Anfangswert des besten Zugs wird auf den ersten Zug
festgesetzt. Der Wert dieses wird auf den Wert "- Unendlich" gesetzt.

Nun wird über alle legalen Züge iteriert und für jeden Zug ein
temporärer "board" angelegt, das erreichbaren Zustand widerspiegelt. Nun
wird mittels des Minimax-Algorithmus der Wert dieses boards ermittelt.
Dabei wird als Alpha "- Unendlich" und als Beta "Unendlich" mitgegeben.
Was die Werte Alpha und Beta aussagen, wird im Kapitel INSERT
beschrieben. Zudem wird die Tiefe auf einen Wert festgelegt, der um
einen Wert geringer ist als die maximale Tiefe, da durch Aufruf der
Funktion in die erste Tiefe hineingegangen wurde. Zudem wird die Zeit
mitgegeben, zu der der Algorithmus enden muss, damit der Minimax
Algorithmus dementsprechend endet und das zeitlimit nicht überschreitet.

Nachdem der Minimax-Algorithmus fertig durchlaufen ist, wird der Wert
mit dem Zug zu dem Dictionary hinzugefügt. Gleicht der Wert dem
maximalen Wert für einen Zustand, ist also dementsprechend ein Sieg,
wird der Zug direkt zurückgegeben, da mit diesem dann auf jeden Fall ein
Sieg erreich werdne kann. Andernfalls wird der Wert verglichen, ob er
besser ist als der aktuelle Wert. Ist dies der Fall, so wird der neue
beste Wert auf den aktuell berechneten festgelegt, ebenso wie der beste
Zug auf den der aktuellen Iteration gesetzt wird.

Nachdem alle Züge durchlaufen wurden, wird die Liste aller legalen Züge
an Hand der berechneten Werte sortiert, damit im nächsten Durchlauf die
Züge in dieser Reihenfolge durchlaufen werden. Dies verbessert den
Durchsatz beim Alpha Beta Pruning, wie in Kapitel INSERT beschrieben und
garantiert zudem, dass der beste Wert der vergangenen Runde gewählt
wird, falls der Minimax-Algorithmus beim Durchlaufen der nächst tieferen
Tiefe das Zeitlimit erreicht, bevor die Runde komplett evaluiert werden
konnte.

Abschließend wird noch die Tiefe um 1 erhöht und die aktuelle Zeit auf
die Systemzeit gesetzt, damit an Hand dieser entschieden werden kann, ob
der Algorithmus noch weiter durchlaufen darf.

Nachdem dann die Zeit abgelaufen ist und alle Züge in der für die
angegebenen Zeit maximalen Tiefe evaluiert wurden, wird der best
mögliche Zug zurückgegeben. Dieser wird dann von der Ki ausgeführt.

    \section{Implementierung des Minimax-Algorithmus mit
Alpha-Beta-Pruning}\label{implementierung-des-minimax-algorithmus-mit-alpha-beta-pruning}

Um den bestmöglichen Zug zu erkennen, wird beim Minimax Algorithmus
jeder Zug bis zu einer gewissen Tiefe betrachtet. Dabei wird unter der
Prämisse gehandelt, dass auch der Gegner stets den besten Zug macht.
Dies führt dazu, dass der bestmöglichste Zug ausgewählt wird (max), der
erreichbar ist, wenn der Gegner mit dem für ihn jeweils besten Zug
antwortet, der für den Spieler somit der schlechteste ist (min). Genauer
ist dies in Kapitel INSERT erklärt.

Die Umsetzung dabei erfolgt in zwei Funktionen - \texttt{min\_value} und
\texttt{max\_value}. Erstere berechnet dabei den schlecht möglichsten
Ausgang für den Spieler aus einer bestimmten Position, also den besten
Ausgang für den Gegner. Letztere berechnet den best möglichsten Ausgang.
Beide Funktionen sind sehr ähnlich aufgebaut und in folgendem
Code-Snippet zu sehen.

    \begin{Verbatim}[commandchars=\\\{\}]
{\color{incolor}In [{\color{incolor} }]:} \PY{k}{def} \PY{n+nf}{min\PYZus{}value}\PY{p}{(}\PY{n+nb+bp}{self}\PY{p}{,} \PY{n}{board\PYZus{}fen}\PY{p}{,} \PY{n}{player}\PY{p}{,} \PY{n}{alpha}\PY{p}{,} \PY{n}{beta}\PY{p}{,} \PY{n}{depth}\PY{p}{,} \PY{n}{time\PYZus{}limit}\PY{p}{)}\PY{p}{:}
            \PY{n}{board} \PY{o}{=} \PY{n}{chess}\PY{o}{.}\PY{n}{Board}\PY{p}{(}\PY{n}{board\PYZus{}fen}\PY{p}{)}
            \PY{n}{v} \PY{o}{=} \PY{n+nb}{float}\PY{p}{(}\PY{l+s+s1}{\PYZsq{}}\PY{l+s+s1}{inf}\PY{l+s+s1}{\PYZsq{}}\PY{p}{)}
        
            \PY{k}{if} \PY{n}{board}\PY{o}{.}\PY{n}{is\PYZus{}game\PYZus{}over}\PY{p}{(}\PY{p}{)} \PY{o+ow}{or} \PY{n}{depth} \PY{o}{==} \PY{l+m+mi}{0}\PY{p}{:}
                \PY{k}{return} \PY{n+nb+bp}{self}\PY{o}{.}\PY{n}{evaluate\PYZus{}board}\PY{p}{(}\PY{n}{board}\PY{p}{,} \PY{n}{player}\PY{p}{)}
            \PY{k}{if} \PY{n+nb}{int}\PY{p}{(}\PY{n}{time}\PY{o}{.}\PY{n}{time}\PY{p}{(}\PY{p}{)}\PY{p}{)} \PY{o}{\PYZgt{}}\PY{o}{=} \PY{n}{time\PYZus{}limit}\PY{p}{:}
                \PY{k}{return} \PY{n+nb}{float}\PY{p}{(}\PY{l+s+s2}{\PYZdq{}}\PY{l+s+s2}{\PYZhy{}inf}\PY{l+s+s2}{\PYZdq{}}\PY{p}{)}
        
            \PY{k}{for} \PY{n}{move} \PY{o+ow}{in} \PY{n}{board}\PY{o}{.}\PY{n}{legal\PYZus{}moves}\PY{p}{:}
                \PY{n}{tmp\PYZus{}board} \PY{o}{=} \PY{n}{chess}\PY{o}{.}\PY{n}{Board}\PY{p}{(}\PY{n}{board\PYZus{}fen}\PY{p}{)}
                \PY{n}{tmp\PYZus{}board}\PY{o}{.}\PY{n}{push}\PY{p}{(}\PY{n}{move}\PY{p}{)}
                \PY{n}{v} \PY{o}{=} \PY{n+nb}{min}\PY{p}{(}\PY{n}{v}\PY{p}{,} \PY{n+nb+bp}{self}\PY{o}{.}\PY{n}{max\PYZus{}value}\PY{p}{(}\PY{n+nb}{str}\PY{p}{(}\PY{n}{tmp\PYZus{}board}\PY{o}{.}\PY{n}{fen}\PY{p}{(}\PY{p}{)}\PY{p}{)}\PY{p}{,} \PY{n}{player}\PY{p}{,} \PY{n}{alpha}\PY{p}{,} \PY{n}{beta}\PY{p}{,} \PY{n}{depth} \PY{o}{\PYZhy{}}\PY{l+m+mi}{1}\PY{p}{,} \PY{n}{time\PYZus{}limit}\PY{p}{)}\PY{p}{)}
                \PY{k}{if} \PY{n}{v} \PY{o}{\PYZlt{}}\PY{o}{=} \PY{n}{alpha}\PY{p}{:}
                    \PY{k}{return} \PY{n}{v}
                \PY{n}{beta} \PY{o}{=} \PY{n+nb}{min}\PY{p}{(}\PY{n}{beta}\PY{p}{,} \PY{n}{v}\PY{p}{)}
            \PY{k}{return} \PY{n}{v}
\end{Verbatim}

    Neben dem akteullen Zustand des Spiels in fen Notation und dem Spieler,
für den es den Zustand zu evaluieren gilt, wird außerdem eine Tiefe
sowie ein Zeitlimit mitgegeben sowie die Werte Alpha und Beta. Alpha und
Beta sind dabei, wie in Kapitel INSERT erklärt, dazu da, um den
Minimax-Algorithmus zu beschleunigen, indem nicht jeder mögliche Zustand
betrachtet wird. Durch diese Werte fallen nämlich solche weg, die direkt
als irrelevant betrachtet werden können, da ohnehin bereits ein
besserrer (im Fall von \texttt{min\_value}) bzw. schlechterer (im Fall
von \texttt{max\_value}) Wert gefunden wurde.

Nachdem das Schachbrett über die fen-Notation erstellt wurde, wird
überprüft, ob das Spiel bereits vorbei ist oder die mitgegebene Tiefe
auf 0 liegt. In beiden Fällen wird der aktuelle Zustand direkt evaluiert
und zurückgegeben. Ansonsten wird geprüft, ob das übergebene Zeitlimit
bereits erreicht wurde. In dem Fall wird der maximal negative Wert
zurück gegeben, damit dieser Zug keine weitere Beachtung mehr bei der
endgültigen Auswahl findet.

Ist auch dies nicht der Fall, werden alle von dem gegebenen Zustand aus
erreichbaren Zustände durchgegangen. Dazu wird zunächst ein temporäres
Schachbrett - hier board - erstellt und der Zug auf diesem ausgeführt.
Dies ist dann der neue, erreichbare Zustand. Dieser wird dann in die
nächst tiefere Iteration gegeben, bei der nun der maximale Wert gesucht
wird. Dabei wird auch der Spieler mitgegeben sowie die Werte Alpha und
Beta und das Zeitlimit ebenso wie die um eins reduzierte Tiefe. Ist
dabei ein Wert dabei, der kleiner ist als das aktuelle v, wird dieser
Wert als das neue v genommen. Andernfalls bleibt v beim akteullen Wert.

Anschließend wird überprüft, ob der Wert v kleiner ist als der aktuelle
Alpha Wert. Ist dies der Fall, muss der Pfad keine weitere Beachtung
finden und es kann direkt v zurück gegeben werden. ist dies nicht der
Fall, so wird Beta auf v gesetzt, falls dieser Wert kleiner als das
aktuelle Beta ist. Nach Durchgang aller legalen Züge wird dann der sich
aus all diesen Iterationen ergebende Wert v zurück gegeben.

Die \texttt{max\_value} Funktion läuft similar ab, mit dem einzigen
Unterschied, dass hier der maximale statt der minmale Wert gesucht wird
und dementsprechend die Vergleiche sowie Startwerte angepasst sind.
Außerdem gibt dieser bei der um eins tieferen Iteration in die
\texttt{min\_value} Funktion, an der Stelle, an der diese in die
\texttt{max\_value} Funktion gibt. Der restliche Aufbau bleibt
unverändert.

    \begin{Verbatim}[commandchars=\\\{\}]
{\color{incolor}In [{\color{incolor} }]:} \PY{k}{def} \PY{n+nf}{max\PYZus{}value}\PY{p}{(}\PY{n+nb+bp}{self}\PY{p}{,} \PY{n}{board\PYZus{}fen}\PY{p}{,} \PY{n}{player}\PY{p}{,} \PY{n}{alpha}\PY{p}{,} \PY{n}{beta}\PY{p}{,} \PY{n}{depth}\PY{p}{,} \PY{n}{time\PYZus{}limit}\PY{p}{)}\PY{p}{:}
            \PY{n}{board} \PY{o}{=} \PY{n}{chess}\PY{o}{.}\PY{n}{Board}\PY{p}{(}\PY{n}{board\PYZus{}fen}\PY{p}{)}
            \PY{n}{v} \PY{o}{=} \PY{n+nb}{float}\PY{p}{(}\PY{l+s+s1}{\PYZsq{}}\PY{l+s+s1}{\PYZhy{}inf}\PY{l+s+s1}{\PYZsq{}}\PY{p}{)}
        
            \PY{k}{if} \PY{n}{board}\PY{o}{.}\PY{n}{is\PYZus{}game\PYZus{}over}\PY{p}{(}\PY{p}{)} \PY{o+ow}{or} \PY{n}{depth} \PY{o}{==} \PY{l+m+mi}{0}\PY{p}{:}
                \PY{k}{return} \PY{n+nb+bp}{self}\PY{o}{.}\PY{n}{evaluate\PYZus{}board}\PY{p}{(}\PY{n}{board}\PY{p}{,} \PY{n}{player}\PY{p}{)}
            \PY{k}{if} \PY{n+nb}{int}\PY{p}{(}\PY{n}{time}\PY{o}{.}\PY{n}{time}\PY{p}{(}\PY{p}{)}\PY{p}{)} \PY{o}{\PYZgt{}}\PY{o}{=} \PY{n}{time\PYZus{}limit}\PY{p}{:}
                \PY{k}{return} \PY{n+nb}{float}\PY{p}{(}\PY{l+s+s2}{\PYZdq{}}\PY{l+s+s2}{inf}\PY{l+s+s2}{\PYZdq{}}\PY{p}{)}
        
            \PY{k}{for} \PY{n}{move} \PY{o+ow}{in} \PY{n}{board}\PY{o}{.}\PY{n}{legal\PYZus{}moves}\PY{p}{:}
                \PY{n}{tmp\PYZus{}board} \PY{o}{=} \PY{n}{chess}\PY{o}{.}\PY{n}{Board}\PY{p}{(}\PY{n}{board\PYZus{}fen}\PY{p}{)}
                \PY{n}{tmp\PYZus{}board}\PY{o}{.}\PY{n}{push}\PY{p}{(}\PY{n}{move}\PY{p}{)}
                \PY{n}{v} \PY{o}{=} \PY{n+nb}{max}\PY{p}{(}\PY{n}{v}\PY{p}{,} \PY{n+nb+bp}{self}\PY{o}{.}\PY{n}{min\PYZus{}value}\PY{p}{(}\PY{n+nb}{str}\PY{p}{(}\PY{n}{tmp\PYZus{}board}\PY{o}{.}\PY{n}{fen}\PY{p}{(}\PY{p}{)}\PY{p}{)}\PY{p}{,} \PY{n}{player}\PY{p}{,} \PY{n}{alpha}\PY{p}{,} \PY{n}{beta}\PY{p}{,} \PY{n}{depth} \PY{o}{\PYZhy{}}\PY{l+m+mi}{1}\PY{p}{,} \PY{n}{time\PYZus{}limit}\PY{p}{)}\PY{p}{)}
                \PY{k}{if} \PY{n}{v} \PY{o}{\PYZgt{}}\PY{o}{=} \PY{n}{beta}\PY{p}{:}
                    \PY{k}{return} \PY{n}{v}
                \PY{n}{alpha} \PY{o}{=} \PY{n+nb}{max}\PY{p}{(}\PY{n}{alpha}\PY{p}{,} \PY{n}{v}\PY{p}{)}
            \PY{k}{return} \PY{n}{v}
\end{Verbatim}

    Mit diesem Algorithmus wird ein Baum aus allen Pfaden erstellt, an Hand
der der beste Zug ermittelt werden kann wie in Kapitel INSERT
beschrieben. Da es sich jedoch nur selten um Endzustände handelt, für
die eine Bewertung trivial erfolgen kann, ist eine Evaluierung der
Zustände nötig. Diese wird im nächsten Kapitel beschrieben.

    \section{Evaluierung eines gegebenen
Schachbretts}\label{evaluierung-eines-gegebenen-schachbretts}

    Um gegebene Zustände auch mitten im Spiel bewerten zu können, müssen
diese an Hand bestimmter Kriterien bewertet werden können, die über Sieg
oder Niederlage hinausgehen. Dazu gibt es verschiedene Ansätze, wie in
Kapitel INSERT erläutert. Einige davon wurden im Laufe des Projektes
umgesetzt und implementiert. Diese werden in diesem Kapitel erläutert.
Zunächst jedoch gilt es aufzuzeigen, wie die Evaluierung der Zustände im
generellen umgesetzt wird.

Zunächst wird zum Evaluieren eines Zustandes jener Zustand sowie der
Spieler, für den dieser zu evaluieren ist, in die Funktion
\texttt{evaluate\_board} gegeben, die die Evaluierung zentral verwaltet.

    \begin{Verbatim}[commandchars=\\\{\}]
{\color{incolor}In [{\color{incolor} }]:} \PY{k}{def} \PY{n+nf}{evaluate\PYZus{}board}\PY{p}{(}\PY{n+nb+bp}{self}\PY{p}{,} \PY{n}{board}\PY{p}{,} \PY{n}{player}\PY{p}{)}\PY{p}{:}
            \PY{n}{player\PYZus{}color} \PY{o}{=} \PY{n}{chess}\PY{o}{.}\PY{n}{WHITE} \PY{k}{if} \PY{n}{player} \PY{k}{else} \PY{n}{chess}\PY{o}{.}\PY{n}{BLACK}
        
            \PY{k}{if} \PY{n}{board}\PY{o}{.}\PY{n}{is\PYZus{}game\PYZus{}over}\PY{p}{(}\PY{p}{)}\PY{p}{:}
                \PY{n}{result} \PY{o}{=} \PY{n}{Tools}\PY{o}{.}\PY{n}{get\PYZus{}board\PYZus{}result}\PY{p}{(}\PY{n}{board}\PY{p}{)}
                \PY{k}{if} \PY{n}{result} \PY{o+ow}{is} \PY{n+nb+bp}{self}\PY{o}{.}\PY{n}{best\PYZus{}possible\PYZus{}result}\PY{p}{:}
                    \PY{k}{return} \PY{n}{MAX\PYZus{}BOARD\PYZus{}VALUE}
                \PY{k}{if} \PY{n}{result} \PY{o+ow}{is} \PY{o}{\PYZhy{}}\PY{l+m+mi}{1}\PY{p}{:}
                    \PY{k}{return} \PY{o}{\PYZhy{}}\PY{l+m+mi}{1} \PY{o}{*} \PY{n}{MAX\PYZus{}BOARD\PYZus{}VALUE}
        
            \PY{n}{evaluation\PYZus{}val} \PY{o}{=} \PY{l+m+mi}{0}
            \PY{k}{for} \PY{n}{func}\PY{p}{,} \PY{n}{value} \PY{o+ow}{in} \PY{n+nb+bp}{self}\PY{o}{.}\PY{n}{evaluation\PYZus{}funcs\PYZus{}dict}\PY{o}{.}\PY{n}{items}\PY{p}{(}\PY{p}{)}\PY{p}{:}
                \PY{k}{if} \PY{n}{value} \PY{o}{\PYZgt{}} \PY{l+m+mi}{0}\PY{p}{:}
                    \PY{n}{evaluation\PYZus{}val} \PY{o}{=} \PY{n}{evaluation\PYZus{}val} \PY{o}{+} \PY{n}{value} \PY{o}{*} \PY{n}{func}\PY{p}{(}\PY{n}{board}\PY{p}{,} \PY{n}{player\PYZus{}color}\PY{p}{)}
            \PY{k}{return} \PY{n}{evaluation\PYZus{}val}
\end{Verbatim}

    Dabei wird zunächst an Hand des Spielers die Farbe dieses ermittelt, die
später bei den einzelnen Evaluierungsfunktionen benötigt wird.

Dann wird für den Fall, dass der gegebene Zustand einem Endzustand
gleicht, das Ergebnis dieses ermittelt. Gleicht dieses dem bestmöglichen
Ergebnis, dass mittels der \texttt{get\_board\_result} Funktion zuvor
ermittelt wurde (siehe INSERT), so wird der maximale Wert (Unendlich)
für den Zusand zurückgegeben. Ist das Ergebnis des übermittelten
Zustands jedoch eine Niederlage, so wird der maximale Wert umgekehrt und
zurückgegeben (minus Unendlich).

Gleicht der Zustand keinem Endzustand, so werden bestimmte
Evaluationsfunktionen durchlaufen und aufaddiert. Dazu startet der Wert
bei 0 und für jede Evaluierungsfunktion wird das Ergebnis dieser
multipliziert mit einem festgelegten Faktor zu dem Gesamtwert
aufaddiert. Die Faktoren sowie die durchzuführenden
Evaluierungsfunktionen sind dabei abhängig vom Schwierigkeitsgrad der KI
sowie vom Spielstatus (Eröffnung, Mittelspiel, Endspiel).

Dabei werden zur Performanz-Steigerung jedoch nur Evaluierungsfunktionen
durchgegangen, dessen Faktor höher als 0 liegt. Der Grund dafür ist,
dass bestimmte Funktionne je nach Spielstatus leichter aus der
Evaluierung herausgenommen werden können, ohne, dass das gesamte
Dictionary angepasst werden muss und zudem keine unnötige Rechenzeit
durch Berechnung eines Werts benötigt wird, der im Endeffekt ohnehin
nicht zum Evaluierungswert aufaddiert wird.

Der daraus entstehende Evaluierungswert, der zurückgegeben wird, gibt
einen guten Aufschluss über den Wert des aktuellen Zustands. Dazu werden
verschiedene Evaluierungsfunktionen verwendet, wie in folgendem
Abschnitt zu sehen ist.

    \subsubsection{Materialbewertung}\label{materialbewertung}

Eine zentrale sowie einfache Bewertung ist dabei die Bewertung der
vorhandenen Materialien auf dem Spielfeld. Dabei werden alle Figuren der
jeweiligen Spieler zusammengezählt und je nach Figur mit einem Wert
multipliziert.

Dabei ist zunächst jedem Figurentyp ein Wert zuzuweisen. Üblicherweise
werden Bauern dabei 1 Punkt, Türmen 5 Punkte, Springern sowie Läufern
jeweils 3 Punkte und der Dame 9 Punkte zugeordnet. Diese werden dann
zusammengerechnet.

    \begin{Verbatim}[commandchars=\\\{\}]
{\color{incolor}In [{\color{incolor} }]:} \PY{k}{def} \PY{n+nf}{get\PYZus{}value\PYZus{}by\PYZus{}color}\PY{p}{(}\PY{n}{color}\PY{p}{)}\PY{p}{:}
            \PY{n}{attacked\PYZus{}pieces\PYZus{}value} \PY{o}{=} \PY{n+nb}{map}\PY{p}{(}\PY{k}{lambda} \PY{n}{piece\PYZus{}type} \PY{p}{:} \PY{n+nb}{len}\PY{p}{(}\PY{n}{board}\PY{o}{.}\PY{n}{pieces}\PY{p}{(}\PY{n}{piece\PYZus{}type}\PY{p}{,} \PY{n}{color}\PY{p}{)}\PY{p}{)} \PY{o}{*} \PY{n}{assign\PYZus{}piece\PYZus{}value}\PY{p}{(}\PY{n}{piece\PYZus{}type}\PY{p}{)}\PY{p}{,} \PY{n}{chess}\PY{o}{.}\PY{n}{PIECE\PYZus{}TYPES}\PY{p}{)}
            \PY{k}{return} \PY{n+nb}{sum}\PY{p}{(}\PY{n}{attacked\PYZus{}pieces\PYZus{}value}\PY{p}{)}
        
        \PY{k}{def} \PY{n+nf}{get\PYZus{}board\PYZus{}value}\PY{p}{(}\PY{p}{)}\PY{p}{:}
            \PY{n}{white\PYZus{}value} \PY{o}{=} \PY{n}{get\PYZus{}value\PYZus{}by\PYZus{}color}\PY{p}{(}\PY{n}{chess}\PY{o}{.}\PY{n}{WHITE}\PY{p}{)}
            \PY{n}{black\PYZus{}value} \PY{o}{=} \PY{n}{get\PYZus{}value\PYZus{}by\PYZus{}color}\PY{p}{(}\PY{n}{chess}\PY{o}{.}\PY{n}{BLACK}\PY{p}{)}
            
            \PY{k}{return} \PY{n}{white\PYZus{}value} \PY{o}{\PYZhy{}} \PY{n}{black\PYZus{}value} \PY{k}{if} \PY{n}{color} \PY{o+ow}{is} \PY{n}{chess}\PY{o}{.}\PY{n}{WHITE} \PY{k}{else} \PY{n}{black\PYZus{}value} \PY{o}{\PYZhy{}} \PY{n}{white\PYZus{}value}
\end{Verbatim}

    Um den Gesamtwert des Schachbretts zu berechnen muss zunächst der Wert
aller weißer Figuren berechnet werden und von diesem der Wert aller
schwarzen Figuren abgezogen werden. Je nach angegebenen Spieler wird für
diesen ein positiver Wert zurückgegeben, wenn das Spiel zu dessen
Gunsten verläuft und ein negativer, wenn dies nicht der Fall ist.

Damit die Werte der Spieler berechnet werden können, wird die Anzahl
aller Figurentypen der jeweiligen Farbe berechnet und diese mit dem Wert
der Figurentypen multipliziert. Am Ende werden die Ergebnisse für alle
Figurentypen zusammengezählt und zurückgegeben.

    \subsubsection{Materialbewertung attackierter
Figuren}\label{materialbewertung-attackierter-figuren}

Die Berechnung der attackierten Figuren ist ähnlich zu dem Vorgehen bei
der Berechnung des Brettwerts. Dabei werden erst die Werte, der vom
weißen Spieler attackierten Figuren berechnet und davon die Werte der
vom schwarzen Spieler attackierten Figuren abgezogen. Auch hierbei ist
ein positives Ergebnis zum Vorteil des angegebenen Spielers und ein
negativer Wert zum Vorteil des Gegenübers. Ebenso gilt umso höher der
Wert, desto deutlicher der Vorteil.

    \begin{Verbatim}[commandchars=\\\{\}]
{\color{incolor}In [{\color{incolor} }]:} \PY{k}{def} \PY{n+nf}{get\PYZus{}attacked\PYZus{}pieces\PYZus{}value\PYZus{}by\PYZus{}color}\PY{p}{(}\PY{n}{attacker\PYZus{}color}\PY{p}{,} \PY{n}{defender\PYZus{}color}\PY{p}{)}\PY{p}{:}
            \PY{n}{attackedSquares} \PY{o}{=} \PY{n+nb}{filter}\PY{p}{(}\PY{k}{lambda} \PY{n}{square} \PY{p}{:} \PY{n}{board}\PY{o}{.}\PY{n}{is\PYZus{}attacked\PYZus{}by}\PY{p}{(}\PY{n}{attacker\PYZus{}color}\PY{p}{,} \PY{n}{square}\PY{p}{)} \PY{o+ow}{and} \PY{o+ow}{not} \PY{n}{board}\PY{o}{.}\PY{n}{piece\PYZus{}at}\PY{p}{(}\PY{n}{square}\PY{p}{)} \PY{o+ow}{is} \PY{k+kc}{None} \PY{o+ow}{and} \PY{n}{board}\PY{o}{.}\PY{n}{piece\PYZus{}at}\PY{p}{(}\PY{n}{square}\PY{p}{)}\PY{o}{.}\PY{n}{color} \PY{o+ow}{is} \PY{n}{defender\PYZus{}color}\PY{p}{,} \PY{n}{chess}\PY{o}{.}\PY{n}{SQUARES}\PY{p}{)}
            \PY{n}{attackedPieces} \PY{o}{=} \PY{n+nb}{map}\PY{p}{(}\PY{k}{lambda} \PY{n}{square} \PY{p}{:} \PY{n}{board}\PY{o}{.}\PY{n}{piece\PYZus{}at}\PY{p}{(}\PY{n}{square}\PY{p}{)}\PY{o}{.}\PY{n}{piece\PYZus{}type}\PY{p}{,} \PY{n}{attackedSquares}\PY{p}{)}
            \PY{n}{value} \PY{o}{=} \PY{n+nb}{map}\PY{p}{(}\PY{n}{assign\PYZus{}piece\PYZus{}value}\PY{p}{,} \PY{n}{attackedPieces}\PY{p}{)}
            \PY{k}{return} \PY{n+nb}{sum}\PY{p}{(}\PY{n}{value}\PY{p}{)}
             
        \PY{k}{def} \PY{n+nf}{get\PYZus{}attacked\PYZus{}pieces\PYZus{}value}\PY{p}{(}\PY{p}{)}\PY{p}{:}
            \PY{n}{white\PYZus{}value} \PY{o}{=} \PY{n}{get\PYZus{}attacked\PYZus{}pieces\PYZus{}value\PYZus{}by\PYZus{}color}\PY{p}{(}\PY{n}{chess}\PY{o}{.}\PY{n}{WHITE}\PY{p}{,} \PY{n}{chess}\PY{o}{.}\PY{n}{BLACK}\PY{p}{)}
            \PY{n}{black\PYZus{}value} \PY{o}{=} \PY{n}{get\PYZus{}attacked\PYZus{}pieces\PYZus{}value\PYZus{}by\PYZus{}color}\PY{p}{(}\PY{n}{chess}\PY{o}{.}\PY{n}{BLACK}\PY{p}{,} \PY{n}{chess}\PY{o}{.}\PY{n}{WHITE}\PY{p}{)}
            
            \PY{k}{return} \PY{n}{white\PYZus{}value} \PY{o}{\PYZhy{}} \PY{n}{black\PYZus{}value}
\end{Verbatim}

    Um diese Werte der attackierten Figuren zu berechnen wird jedes Feld
durchgegangen. Daraus werden die Felder gefiltert, die von einer Figur
der Farbe des Verteidigers belegt sind und von einer Figur der
angreifenden Farbe attackiert werden können. Anschließend wird zu diesen
Feldern der Typ der Figur zugeordnet, die sich auf dem Feld befindet.
Daraufhin werden diesen ihre jeweiligen Werte zugeordnet und diese
abschließend summiert.

    \subsubsection{Positionsbewertung}\label{positionsbewertung}

Um die Positionen der einzelnen Figuren zu bewerten, werden zunächst für
jeden Figurentypen Matrizen benötigt, die über jedes Feld eine Aussage
über den Wert der Position der Figur geben. Diese sind in Kapitel INSERT
einsehbar. An Hand dieser Matrizen wird dann Aussage über den Wert
getroffen.

Dabei wird zunächst jedes Feld auf dem Schachbrett durchgegangen und die
darauf befindliche Figur berechnet. Dies wird mittels einer
verschachtelten Schleife gelöst, die zunächst alle Reihen durchgeht und
dann die einzelnen Felder in dieser Reihe.

    \begin{Verbatim}[commandchars=\\\{\}]
{\color{incolor}In [{\color{incolor} }]:} \PY{k}{def} \PY{n+nf}{get\PYZus{}board\PYZus{}positions\PYZus{}value\PYZus{}by\PYZus{}color}\PY{p}{(}\PY{n}{board}\PY{p}{,} \PY{n}{color}\PY{p}{)}\PY{p}{:}
            \PY{n+nb}{sum} \PY{o}{=} \PY{l+m+mi}{0}
            \PY{k}{for} \PY{n}{rank} \PY{o+ow}{in} \PY{n+nb}{range}\PY{p}{(}\PY{l+m+mi}{0}\PY{p}{,}\PY{l+m+mi}{8}\PY{p}{)}\PY{p}{:}
                \PY{k}{for} \PY{n}{file} \PY{o+ow}{in} \PY{n+nb}{range}\PY{p}{(}\PY{l+m+mi}{0}\PY{p}{,}\PY{l+m+mi}{8}\PY{p}{)}\PY{p}{:}
                    \PY{n}{piece} \PY{o}{=} \PY{n}{board}\PY{o}{.}\PY{n}{piece\PYZus{}at}\PY{p}{(}\PY{n}{chess}\PY{o}{.}\PY{n}{square}\PY{p}{(}\PY{n}{file}\PY{p}{,} \PY{n}{rank}\PY{p}{)}\PY{p}{)}
                    \PY{k}{if} \PY{p}{(}\PY{n}{piece} \PY{o+ow}{and} \PY{n}{piece}\PY{o}{.}\PY{n}{color} \PY{o}{==} \PY{n}{color}\PY{p}{)}\PY{p}{:}
                        \PY{n}{piece\PYZus{}pos\PYZus{}value} \PY{o}{=} \PY{n}{get\PYZus{}position\PYZus{}value\PYZus{}by\PYZus{}square}\PY{p}{(}\PY{n}{board}\PY{p}{,} \PY{n}{rank}\PY{p}{,} \PY{n}{file}\PY{p}{,} \PY{n}{color}\PY{p}{)}
                        \PY{n+nb}{sum} \PY{o}{+}\PY{o}{=} \PY{n}{piece\PYZus{}pos\PYZus{}value}
            \PY{k}{return} \PY{n+nb}{sum}
\end{Verbatim}

    Nachdem die Figur ermittelt wurde wird diese, falls diese der
angegebenen Farbe angehört, gemeinsam mit den Werten für Reihe und
Spalte an die Funktion \texttt{get\_position\_value\_by\_square}
übergeben. Nachdem diese den Wert zurückgegben hat wird dies zu der
bisherigen Summe aufaddiert und am Ende die Summe aller Figuren zurück
gegeben.

In der Funktion \texttt{get\_position\_value\_by\_square} wird mittels
der Matrix der Wert der Figur an der gegebenen Position ermittelt.

    \begin{Verbatim}[commandchars=\\\{\}]
{\color{incolor}In [{\color{incolor} }]:} \PY{k}{def} \PY{n+nf}{get\PYZus{}position\PYZus{}value\PYZus{}by\PYZus{}square}\PY{p}{(}\PY{n}{board}\PY{p}{,} \PY{n}{rank}\PY{p}{,} \PY{n}{file}\PY{p}{,} \PY{n}{color}\PY{p}{)}\PY{p}{:}
            \PY{n}{piece\PYZus{}type} \PY{o}{=} \PY{n}{board}\PY{o}{.}\PY{n}{piece\PYZus{}type\PYZus{}at}\PY{p}{(}\PY{n}{chess}\PY{o}{.}\PY{n}{square}\PY{p}{(}\PY{n}{file}\PY{p}{,} \PY{n}{rank}\PY{p}{)}\PY{p}{)}
            \PY{n}{piece\PYZus{}matrix} \PY{o}{=} \PY{n}{assign\PYZus{}piece\PYZus{}matrix}\PY{p}{(}\PY{n}{piece\PYZus{}type}\PY{p}{)} \PY{k}{if} \PY{n}{color} \PY{o}{==} \PY{n}{chess}\PY{o}{.}\PY{n}{BLACK} \PY{k}{else} \PY{n}{np}\PY{o}{.}\PY{n}{flip}\PY{p}{(}\PY{n}{assign\PYZus{}piece\PYZus{}matrix}\PY{p}{(}\PY{n}{piece\PYZus{}type}\PY{p}{)}\PY{p}{)}
            \PY{n}{piece\PYZus{}pos\PYZus{}value} \PY{o}{=} \PY{n}{piece\PYZus{}matrix}\PY{p}{[}\PY{n}{rank}\PY{p}{,}\PY{n}{file}\PY{p}{]}
            \PY{k}{return} \PY{n}{piece\PYZus{}pos\PYZus{}value}
\end{Verbatim}

    Dazu wird zunächst der Typ der Figur auf dem gegebenen Feld ermittelt.
Dann wird die dazu passende Matrix bestimmt und entsprechend gespiegelt,
falls die Farbe des angegebenen Spielers weiß sein sollte, damit die
Matrix mit den Positionen aus der Sicht des Spielers übereinstimmt.

Schlussendlich wird der Wert in der Matrix über die Reihe und Spalte
ermittelt und zurück gegeben.

    \subsubsection{Königszonen-Sicherheit}\label{kuxf6nigszonen-sicherheit}

Ein weiterer, wichtiger Wert ist die Sicherheit des Königs bemessen an
Hand der Figuren, die dessen Zone angreifen.

Dazu wird diese Zone berechnet und dann alle Figuren, die diese Zone
angreifen ermittelt. Mittels der in Kapitel INSERT vorgestellten
Berechnung wird dann der Wert des Angriffs auf die Königszone berechnet.

    \begin{Verbatim}[commandchars=\\\{\}]
{\color{incolor}In [{\color{incolor} }]:} \PY{k}{def} \PY{n+nf}{calculate\PYZus{}king\PYZus{}zone\PYZus{}safety}\PY{p}{(}\PY{n}{board}\PY{p}{,} \PY{n}{color}\PY{p}{)}\PY{p}{:}
            \PY{n}{attacker\PYZus{}color} \PY{o}{=} \PY{n}{chess}\PY{o}{.}\PY{n}{WHITE} \PY{k}{if} \PY{n}{color} \PY{o}{==} \PY{n}{chess}\PY{o}{.}\PY{n}{BLACK} \PY{k}{else} \PY{n}{chess}\PY{o}{.}\PY{n}{BLACK}
            \PY{n}{king\PYZus{}zone} \PY{o}{=} \PY{n}{calculate\PYZus{}king\PYZus{}zone}\PY{p}{(}\PY{n}{board}\PY{p}{,} \PY{n}{color}\PY{p}{)}
            \PY{n}{attackers} \PY{o}{=} \PY{n}{get\PYZus{}attackers\PYZus{}by\PYZus{}squares}\PY{p}{(}\PY{n}{board}\PY{p}{,} \PY{n}{king\PYZus{}zone}\PY{p}{,} \PY{n}{attacker\PYZus{}color}\PY{p}{)}
            \PY{n}{attack\PYZus{}weight} \PY{o}{=} \PY{n}{get\PYZus{}king\PYZus{}attack\PYZus{}weight}\PY{p}{(}\PY{n+nb}{len}\PY{p}{(}\PY{n}{attackers}\PY{p}{)}\PY{p}{)}
            \PY{n}{value\PYZus{}of\PYZus{}attack} \PY{o}{=} \PY{l+m+mi}{0}
            \PY{k}{for} \PY{n}{attacker} \PY{o+ow}{in} \PY{n}{attackers}\PY{p}{:}
                \PY{n}{value\PYZus{}of\PYZus{}attack} \PY{o}{+}\PY{o}{=} \PY{n}{get\PYZus{}king\PYZus{}attack\PYZus{}constants}\PY{p}{(}\PY{n}{attacker}\PY{o}{.}\PY{n}{piece\PYZus{}type}\PY{p}{)}
            
            \PY{k}{return} \PY{p}{(}\PY{n}{value\PYZus{}of\PYZus{}attack} \PY{o}{*} \PY{n}{attack\PYZus{}weight}\PY{p}{)} \PY{o}{/} \PY{l+m+mi}{1000}
\end{Verbatim}

    Dabei wird zuerst die Farbe des Angreifers berechnet, indem die gegebene
Farbe umgekehrt wird. Danach wird mittels der Funktion
\texttt{calculate\_king\_zone} die Königszone berechnet. Diese Funktion
gibt eine Menge von Feldern zurück, die der Königszone angehören.

    \begin{Verbatim}[commandchars=\\\{\}]
{\color{incolor}In [{\color{incolor} }]:} \PY{k}{def} \PY{n+nf}{calculate\PYZus{}king\PYZus{}zone}\PY{p}{(}\PY{n}{board}\PY{p}{,} \PY{n}{color}\PY{p}{)}\PY{p}{:}
            \PY{n}{king\PYZus{}zone} \PY{o}{=} \PY{n}{chess}\PY{o}{.}\PY{n}{SquareSet}\PY{p}{(}\PY{p}{)}
            \PY{n}{king\PYZus{}rank}\PY{p}{,} \PY{n}{king\PYZus{}file} \PY{o}{=} \PY{n}{get\PYZus{}piece\PYZus{}position}\PY{p}{(}\PY{n}{board}\PY{p}{,} \PY{n}{chess}\PY{o}{.}\PY{n}{Piece}\PY{p}{(}\PY{n}{chess}\PY{o}{.}\PY{n}{KING}\PY{p}{,} \PY{n}{color}\PY{p}{)}\PY{p}{)}
        
            \PY{n}{rank\PYZus{}range} \PY{o}{=} \PY{n+nb}{range}\PY{p}{(}\PY{l+m+mi}{0}\PY{p}{,}\PY{l+m+mi}{4}\PY{p}{)} \PY{k}{if} \PY{n}{color} \PY{o}{==} \PY{n}{chess}\PY{o}{.}\PY{n}{WHITE} \PY{k}{else} \PY{n+nb}{range}\PY{p}{(}\PY{o}{\PYZhy{}}\PY{l+m+mi}{3}\PY{p}{,} \PY{l+m+mi}{1}\PY{p}{)}
            \PY{k}{for} \PY{n}{rank\PYZus{}summand} \PY{o+ow}{in} \PY{n}{rank\PYZus{}range}\PY{p}{:}
                \PY{k}{if} \PY{p}{(}\PY{n}{king\PYZus{}rank} \PY{o}{+} \PY{n}{rank\PYZus{}summand}\PY{p}{)} \PY{o+ow}{in} \PY{n+nb}{range}\PY{p}{(}\PY{l+m+mi}{0}\PY{p}{,} \PY{l+m+mi}{8}\PY{p}{)}\PY{p}{:}
                    \PY{k}{for} \PY{n}{file\PYZus{}summand} \PY{o+ow}{in} \PY{n+nb}{range}\PY{p}{(}\PY{o}{\PYZhy{}}\PY{l+m+mi}{1}\PY{p}{,} \PY{l+m+mi}{2}\PY{p}{)}\PY{p}{:}
                        \PY{k}{if} \PY{p}{(}\PY{n}{king\PYZus{}file} \PY{o}{+} \PY{n}{file\PYZus{}summand}\PY{p}{)} \PY{o+ow}{in} \PY{n+nb}{range} \PY{p}{(}\PY{l+m+mi}{0}\PY{p}{,}\PY{l+m+mi}{8}\PY{p}{)}\PY{p}{:}
                            \PY{n}{king\PYZus{}zone}\PY{o}{.}\PY{n}{add}\PY{p}{(}\PY{n}{chess}\PY{o}{.}\PY{n}{square}\PY{p}{(}\PY{n}{king\PYZus{}file} \PY{o}{+} \PY{n}{file\PYZus{}summand}\PY{p}{,} \PY{n}{king\PYZus{}rank} \PY{o}{+} \PY{n}{rank\PYZus{}summand}\PY{p}{)}\PY{p}{)}
            
            \PY{k}{return} \PY{n}{king\PYZus{}zone}
\end{Verbatim}

    Um dies zu ermöglichen, wird zunächst die Position des Königs ermittelt.
Dabei werden alle Felder durchgegangen bis der König der entsprechenden
Farbe gefunden wurde. Dann wird Reihe sowie Spalte zurückgegeben.

Nun werden alle Felder der Königszone einzeln durchgegangen und zu der
Menge an Feldern hinzugeführt. Dabei werden bei den Spaltel alle Felder
bis 3 Felder in Richtung des Gegners durchgegangen und für jede Spalte
ein Feld links bis zu einem Feld rechts von dem König mitgezählt. Dabei
wird zuvor jeweils überprüft, ob sich die Spalte beziehungsweise die
Reihe noch au fdem Spielfeld befinden. Ist dies der Fall, wird das Feld
der Menge hinzugefügt und diese wird am Ende zurück gegeben.

Als nächstes wir ddann für diese Menge an Feldern ermittelt, welche
Figuren diese Felder angreifen. Dies wird mittels der
\texttt{get\_attackers\_by\_squares} Funktion durchgeführt.

    \begin{Verbatim}[commandchars=\\\{\}]
{\color{incolor}In [{\color{incolor} }]:} \PY{k}{def} \PY{n+nf}{get\PYZus{}attackers\PYZus{}by\PYZus{}squares}\PY{p}{(}\PY{n}{board}\PY{p}{,} \PY{n}{square\PYZus{}set}\PY{p}{,} \PY{n}{attacker\PYZus{}color}\PY{p}{)}\PY{p}{:}
            \PY{n}{attacker\PYZus{}dict} \PY{o}{=} \PY{p}{\PYZob{}}\PY{p}{\PYZcb{}}
            \PY{k}{for} \PY{n}{square} \PY{o+ow}{in} \PY{n}{square\PYZus{}set}\PY{p}{:}
                \PY{n}{attacker\PYZus{}square\PYZus{}set} \PY{o}{=} \PY{n}{board}\PY{o}{.}\PY{n}{attackers}\PY{p}{(}\PY{n}{attacker\PYZus{}color}\PY{p}{,} \PY{n}{square}\PY{p}{)}
                \PY{k}{for} \PY{n}{attacker\PYZus{}square} \PY{o+ow}{in} \PY{n}{attacker\PYZus{}square\PYZus{}set}\PY{p}{:}
                    \PY{n}{attacker\PYZus{}piece} \PY{o}{=} \PY{n}{board}\PY{o}{.}\PY{n}{piece\PYZus{}at}\PY{p}{(}\PY{n}{attacker\PYZus{}square}\PY{p}{)}
                    \PY{k}{if} \PY{o+ow}{not} \PY{p}{(}\PY{n}{attacker\PYZus{}piece}\PY{o}{.}\PY{n}{piece\PYZus{}type} \PY{o+ow}{is} \PY{n}{chess}\PY{o}{.}\PY{n}{PAWN} \PY{o+ow}{or} \PY{n}{attacker\PYZus{}piece}\PY{o}{.}\PY{n}{piece\PYZus{}type} \PY{o+ow}{is} \PY{n}{chess}\PY{o}{.}\PY{n}{KING}\PY{p}{)}\PY{p}{:}
                        \PY{n}{attacker\PYZus{}dict}\PY{p}{[}\PY{n}{attacker\PYZus{}piece}\PY{p}{]} \PY{o}{=} \PY{n}{attacker\PYZus{}dict}\PY{o}{.}\PY{n}{get}\PY{p}{(}\PY{n}{attacker\PYZus{}piece}\PY{p}{,} \PY{l+m+mi}{0}\PY{p}{)} \PY{o}{+} \PY{l+m+mi}{1}
            \PY{k}{return} \PY{n}{attacker\PYZus{}dict}
\end{Verbatim}

    Dabei wird jedes Feld einzeln durchgegangen und für alle jeweils eine
Menge an Figuren erstellt, die dieses Feld angreifen. Alle Angriefer
werden dann einzeln durchgegangen und dessen Figurentyp ermittelt. Wenn
dies weder ein Bauer noch ein König ist, wird die Figur einem Dictionary
von allen Angreifern hinzugefügt. Der Wert dieses Eintrags setzt sich
aus der Anzahl zusammen, wie viele Felder von dieser Figur angegriffen
werden. Dieses Dictionary wird nach Durchgang jedes Feldes
zurückgegeben.

Nachdem dieses Dictionary zurückgegeben wurde, wird die Gewichtung der
Attacke ermittelt. Dabei wird die Anzahl der verschiedenen Figuren, die
die Zone angreifen, zur Hilfe genommen und ein ensprechender Wert
zurückgegeben, wie in Kapitel INSERT beschrieben.

Abschließend wird für jeden Angreifer noch der Wert dieses ermittelt,
wie ebenfalls in Kapitel INSERT beschrieben, und diese alle aufaddiert.
Dieser Wird wird dann mit dem berechneten Gewicht multipliziert und
durch 1000 dividiert. Das Ergebnis aus dieser Berechnung wird dann
zurückgegeben und gibt einen Aufschluss über die Sicherheit des Königs.
Dies kann auch für die Sicherheit des gegnerischen Königs angewandt
werden, indem schlicht die Farbe des Angreifers auf die eigene Farbe
gesetzt wird.

    \subsubsection{Mobilität}\label{mobilituxe4t}

    \subsubsection{Bewertung an Hand angelegter
Historie}\label{bewertung-an-hand-angelegter-historie}

Besonders am Anfang des Spiels ist es oft unschlüssig, wie man ein
Schachbrett bewerten kann, da noch sehr viele Optionen des Spielverlaufs
offen sind. Dabei kann eine angelegte Historie helfen, die Aufschluss
über Siegchancen geben kann. Wie eine solche Historie angelegt werden
kann, wurde in Kapitel INSERT besprochen.

    \begin{Verbatim}[commandchars=\\\{\}]
{\color{incolor}In [{\color{incolor} }]:} \PY{k}{def} \PY{n+nf}{get\PYZus{}board\PYZus{}value\PYZus{}by\PYZus{}history}\PY{p}{(}\PY{n}{board}\PY{p}{,} \PY{n}{color}\PY{p}{)}\PY{p}{:}
            \PY{n}{dataset} \PY{o}{=} \PY{n}{pd}\PY{o}{.}\PY{n}{read\PYZus{}csv}\PY{p}{(}\PY{n}{HISTORY\PYZus{}FILE\PYZus{}LOC}\PY{p}{)}
            \PY{n}{row} \PY{o}{=} \PY{n}{dataset}\PY{o}{.}\PY{n}{loc}\PY{p}{[}\PY{n}{dataset}\PY{p}{[}\PY{l+s+s1}{\PYZsq{}}\PY{l+s+s1}{board}\PY{l+s+s1}{\PYZsq{}}\PY{p}{]} \PY{o}{==} \PY{n}{board}\PY{o}{.}\PY{n}{fen}\PY{p}{(}\PY{p}{)}\PY{o}{.}\PY{n}{split}\PY{p}{(}\PY{l+s+s2}{\PYZdq{}}\PY{l+s+s2}{ }\PY{l+s+s2}{\PYZdq{}}\PY{p}{)}\PY{p}{[}\PY{l+m+mi}{0}\PY{p}{]}\PY{p}{]}
            \PY{n}{value} \PY{o}{=} \PY{n}{row}\PY{p}{[}\PY{l+s+s1}{\PYZsq{}}\PY{l+s+s1}{value}\PY{l+s+s1}{\PYZsq{}}\PY{p}{]}\PY{o}{.}\PY{n}{item}\PY{p}{(}\PY{p}{)} \PY{k}{if} \PY{n+nb}{len}\PY{p}{(}\PY{n}{row}\PY{p}{[}\PY{l+s+s1}{\PYZsq{}}\PY{l+s+s1}{value}\PY{l+s+s1}{\PYZsq{}}\PY{p}{]}\PY{p}{)} \PY{o}{==} \PY{l+m+mi}{1} \PY{k}{else} \PY{l+m+mi}{0}
            \PY{k}{return} \PY{n}{value} \PY{k}{if} \PY{n}{color} \PY{o+ow}{is} \PY{n}{chess}\PY{o}{.}\PY{n}{WHITE} \PY{k}{else} \PY{o}{\PYZhy{}}\PY{l+m+mi}{1}\PY{o}{*}\PY{n}{value}
\end{Verbatim}

    Abgefragt werden kann diese, indem zunächst die Daten aus der Datei
geladen werden. Dann wird die Reihe abstrahiert, dessen Wert in der
Spalte "board" dem mitgegebenen Zustand (in fen Notation konvertiert)
gleicht. Dazu wird der entsprechende Wert ausgelesen und zurückgegeben.
Beim Zurückgeben wird der Wert noch negiert, falls der angegebene
Spieler der der schwarzen Figuren ist, da in der Historie die Werte aus
Sicht des Spielers der weißen Figuren gespeichert wird.


    % Add a bibliography block to the postdoc